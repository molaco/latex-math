\chapter{Representación Analítica de las funciones holomorfas}

\begin{note}
  Si $f$ holomorfa se puede representar localmente como una serie de potencias convergente, en particular, una serie de Taylor.
\end{note}

\section{Sucesiones y Series}

\begin{defn}[Sucesión Convergente]
  Sea $\{ z_{n} \}_{n \in \mathbb{N}}$ una sucesión de numeros complejos. Entonces, si
  \[
    \forall \epsilon > 0, \exists N \in \mathbb{N}: | z_{n} - z_{0} | < \epsilon, \quad \forall n \geq N,
  \]
  decimos que $\{ z_{n} \}_{n \in \mathbb{N}}$ converge a $z_{0}$ y lo denotamos $z_{n} \xrightarrow[]{ n \rightarrow \infty } z_{0}$.
\end{defn}

\begin{defn}[Serie Convergente]
  Sea $\sum_{n = 1}^{\infty} a_{n}$ una serie de números complejos. Entonces, si la sucesión de sumas parciales
  \[
    s_{n} = \sum_{k = 1}^{n} a_{k}
  \]
  converge a $S$, decimos que la serie $\sum_{n = 1}^{\infty} a_{n}$ converge a $S$ y lo denotamos $\sum_{n = 1}^{\infty} a_{n} = S$.
\end{defn}

¿AÑADIR TEST CONVERGENCE?

\begin{prop}
  Sea $\{ z_{n} \}_{n \in \mathbb{N}}$ una sucesión de números complejos. Entonces, 
  \[ 
    z_{n} \xrightarrow[]{ n \rightarrow \infty } z_{0} \Leftrightarrow 
    \begin{cases}
      \Re (z_{n}) \xrightarrow[]{ n \rightarrow \infty } \Re (z_{0}) \\
      \Im (z_{n}) \xrightarrow[]{ n \rightarrow \infty } \Im (z_{0})
    \end{cases} 
  \] 
\end{prop}

\begin{defn}[Convergencia absoluta]
  Sea $\sum_{n = 1}^{\infty} a_{n}$ una serie. Entonces, si $\sum_{n = 1}^{\infty} |  a_{n} |$ converge, decimos que $ \sum_{n = 1}^{\infty} a_{n}$ converge absolutamente.
\end{defn}

\begin{obs}
  $\sum_{n = 1}^{\infty} a_{n}$ converge absolutamente $\Rightarrow \sum_{n = 1}^{\infty} a_{n}$ converge.
\end{obs}

\begin{prop}[Producto de Series]
  Sean $\sum_{n = 0}^{\infty} a_{n}, \sum_{n = 0}^{\infty} b_{n}$ series con $a_{n}, b_{n} \in \mathbb{R}, \forall n \in \mathbb{N}$. Si
  \[ 
    c_{n} = \sum_{k = 0}^{n} b_{n -k} a_{k}
  \] 
  entonces,
  \[ 
    \sum_{n = 0}^{\infty} c_{n} = \big ( \sum_{n = 1}^{\infty} a_{n} \big ) \cdot ( \sum_{n = 1}^{\infty} b_{n} )
  \] 
\end{prop}

\begin{prop}
  Sean $\sum_{n = 0}^{\infty} a_{n}, \sum_{n = 0}^{\infty} b_{n}$ series con $a_{n}, b_{n} \in \mathbb{R}, \forall n \in \mathbb{N}$ abosolutamente convergentes. Entonces, $\sum_{n = 0}^{\infty} c_{n}$ es absolutamente convergente.
\end{prop}

\begin{defn}[Convergencia Puntual]
  Sea $f, f_{n}: \Omega \to \mathbb{C}$ funciones, $\{ f_{n} \}_{n \in \mathbb{N}}$ sucesión de funciones tal que $f_{n}(z) \xrightarrow[]{ n \rightarrow \infty } f(z), \forall z \in \Omega$. Entonces, $\{ f_{n} \}_{n \in \mathbb{N}} \xrightarrow[]{ n \rightarrow \infty } f$ puntualmente.
\end{defn}

\begin{obs}
  $\sum_{}^{} f_{n}(z) \xrightarrow[]{ n \rightarrow \infty } f(z), \forall z \in \Omega \Rightarrow$ $\sum_{}^{} f_{n} \xrightarrow[]{ n \rightarrow  \infty } f$.
\end{obs}

\begin{defn}[Convergencia Uniforme]
  Sea $f,f_{n}: \Omega \subset \mathbb{C} \to \mathbb{C}$ funciones, $\{ f_{n} \}_{n \in \mathbb{N}}$ sucesión de funciones. Entonces, si
  \[ 
    \forall \epsilon >0, \exists N \in \mathbb{N}: | f_{n}(z) - f(z) | < \epsilon, \quad \forall n \geq N, \forall z \in \Omega,
  \] 
  decimos que $\{ f_{n} \}_{n \in \mathbb{N}}$ converge converge uniformemente y lo denotamos $\{ f_{n} \}_{n \in \mathbb{N}} \xrightarrow[]{ n \rightarrow \infty } f$ uniformemente.
\end{defn}

\begin{obs}
  $N(\epsilon) \in \mathbb{N}$ no depende de $z \in \Omega$
\end{obs}

\begin{prop}
  Sea $f,f_{n}: \Omega \subset \mathbb{C} \to \mathbb{C}$ funciones, $\{ f_{n} \}_{n \in \mathbb{N}}$ sucesión de funciones tal que $\{ f_{n} \}_{n \in \mathbb{N}} \xrightarrow[]{ n \rightarrow \infty } f$ uniformemente en $\Omega$. Entonces,
  \[ 
    f_{n} \text{ continua } \forall n \in \mathbb{N} \Rightarrow f \text{ continua }
  \] 
\end{prop}

\begin{obs}
  $f$ no es continua $\Rightarrow \{ f_{n} \}$ no converge uniformemente.
\end{obs}

\begin{theo}[Weierstrass]
  Sea $f_{n}: \Omega \subset \mathbb{C} \to \mathbb{C}$ tal que
  \[
    \exists M_{n} : | f_{n}(z) | \leq M_{n}, \quad \forall n \in \mathbb{N}, \forall z \in \Omega.
  \]
  Si $\sum_{n = 1}^{\infty} M_{n}$ converge, entonces $\sum_{n = 1}^{\infty} f_{n}$ converge uniformemente en $ \Omega$.
\end{theo}

\begin{obs}
  $\sum_{n = 1}^{\infty} f_{n}$ converge uniformemente en $ \Omega$ $\Rightarrow \sum_{n = 1}^{\infty} | f_{n} |$ converge unifomemente en $\Omega$ (convergencia absoluta de $\sum_{}^{} f_{n}$).
\end{obs}

\begin{theo}
  Sea $\Omega \subset \mathbb{C}$ abierto, $f,f_{n}: \Omega \to \mathbb{C}$ funciones, $\{ f_{n} \}_{n \in \mathbb{N}}$ sucesión de funciones tal que $\{ f_{n} \}_{n \in \mathbb{N}} \xrightarrow[]{ n \rightarrow \infty } f$ uniformemente en $\Omega$ y $f_{n}$ holomorfa $\forall n \in \mathbb{N}$. Entonces, $f$ es holomnorfa.
\end{theo}

\begin{dem}
  $f_{n}$ holomorfa $\xRightarrow[]{ T. Cauchy } \int_{\partial{T}}^{} f_{n}(z) dz = 0$ y
  \[ 
    \int_{\partial{T}}^{} f_{n}(z) dz \xrightarrow[]{ n \rightarrow \infty } \int_{\partial{T}}^{} f(z) dz 
  \] 
  $\Rightarrow \int_{\partial{T}}^{} f(z) dz = 0 \Rightarrow f$ holomorfa.
\end{dem}

\begin{cor}
  Si $\{ f_{n} \} \xrightarrow[]{ n \rightarrow \infty } f$ uniformemente en $K$ compacto $\forall K \subset \Omega$, también se cumple el teorema anterior.
\end{cor}
