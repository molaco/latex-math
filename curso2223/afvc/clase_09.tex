\section{Ecuaciones de Cauchy-Riemann}

\begin{nota}
  Sea $f: \Omega \subset \mathbb{C} \to \mathbb{C}$ 
  \[ 
    f(x,y) = (u(x, y), v(x, y)) = u(x, y) + i v(x, y) ,
  \] 
  donde $u, v: \mathbb{R}^{2} \to \mathbb{R}^{2}$. Entonces, la matriz jacobiana de $f$ es 
  \[ J_{f}(x,y) =
    \begin{pmatrix}
       u_{x} & u_{y} \\
       v_{x} & v_{y}
    \end{pmatrix} 
  \] 
\end{nota} 

\begin{note}
  Queremos ver que significa qeu $ u, v$ sean diferenciables. Si derivamos $f$ en $z_{0} \in \Omega$ respecto de $x$ y $y$, parte real y parte imaginaria respectivamente, obtenemos dos expresiones de $f'(z_{0})$ que dan lugar a las ecuaciones de Cauchy-Riemann.
\end{note}

\begin{theo}[Ecuaciones Cauchy-Riemann]
  Sea $f: \Omega \subset \mathbb{C} \to \mathbb{C}$. Entonces $f'(z_{0})$ existe $\Leftrightarrow$ $f$ es diferenciable en $z_{0} = (x_{0}, y_{0}) \in \mathbb{R}^{2}$ con
  \[ 
    u_{x} = v_{y}, \; u_{y} = -v_{x} \text{ (Ecuaciones de C-R)} ,
  \] 
  es decir, si $\exists u_{x}, u_{y}, v_{x}, v_{y}$, son continuas en $\Omega$ y satisfacen las ecuaciones, entonces $f$ es analítica en $\Omega$.
\end{theo}

\begin{dem}
  \begin{enumerate}[label=(\roman*)]
    \item []
    \item [($\Rightarrow$)] En el límite
      \[ 
        f'(z_{0}) = \lim_{z \to z_{0}} \frac{f(z) - f(z_{0})}{z - z_{0}}
      \] 
      sustituimos $z = x + iy_{0}$
      \[ 
        \frac{f(z) - f(z_{0})}{z - z_{0}} = \frac{u(x, y_{0}) + i v(x, y_{0}) - u(x_{0},y_{0}) - i v(x_{0}, y_{0})}{x - x_{0}}
      \]
      \[ 
      = \frac{u(x, y_{0}) +  u(x_{0}, y_{0})}{x - x_{0}} + i \frac{v(x,y_{0}) - i v(x_{0}, y_{0})}{x - x_{0}}
      \] 
      donde $\frac{f(z) - f(z_{0})}{z - z_{0}} \xrightarrow[]{x \rightarrow x_{0}} f'(z_{0})$ implica
      \[ 
        \lim_{x \to x_{0}} \frac{u(x, y_{0}) +  u(x_{0}, y_{0})}{x - x_{0}} + i \frac{v(x,y_{0}) - i v(x_{0}, y_{0})}{x - x_{0}} 
      \]
      \[ 
        = \frac{\partial{u}}{\partial{x}}(x_{0}, y_{0}) + i \frac{\partial{v}}{\partial{x}}(x_{0}, y_{0}).
      \] 
      De manera análoga, si $z = x_{0} + i y$ entonces
      \[ 
      \lim_{y \to y_{0}} \frac{u(x_{0}, y) +  u(x_{0}, y_{0})}{i(y - y_{0})} +  \frac{v(x_{0},y) -  v(x_{0}, y_{0})}{(y - y_{0})} 
      \]
      \[ = \frac{1}{i}\frac{\partial{u}}{\partial{y}}(x_{0}, y_{0}) + \frac{\partial{v}}{\partial{y}}(x_{0}, y_{0})
      \]
      \[ 
         = - i\frac{\partial{u}}{\partial{y}}(x_{0}, y_{0}) + \frac{\partial{v}}{\partial{y}}(x_{0}, y_{0}).
      \] 
      Por tanto, $\exists f'(z_{0})$ y tiene el mismo valor independientemente de como $z$ se acerque a $z_{0}$
      \[ 
        f'(z_{0}) = \frac{\partial{u}}{\partial{x}} + i\frac{\partial{v}}{\partial{x}} =  - i\frac{\partial{u}}{\partial{y}} + \frac{\partial{v}}{\partial{y}}
      \] 
    \item [($\Leftarrow$)] A partir del teoremade Taylor
      \[ 
        u(x + s, y + t) = u(x, y) + \frac{\partial{u}}{\partial{x}}(x, y)s + \frac{\partial{u}}{\partial{y}}(x, y)t + R(s, t) 
      \] 
      donde $\frac{R(s,t)}{| h |} \xrightarrow[]{ z \rightarrow z_{0}} 0 $. También
      \[ 
        v(x + s, y + t) = v(x, y) + \frac{\partial{v}}{\partial{x}}(x, y)s + \frac{\partial{v}}{\partial{y}}(x, y)t + G(s, t) 
      \] 
      donde $\frac{G(s,t)}{| h |} \xrightarrow[]{ z \rightarrow z_{0}} 0 $. Entonces,
      \[ 
         f(z + h) = f(z) + \frac{\partial{u}}{\partial{x}}(x, y)s + \frac{\partial{u}}{\partial{y}}(x, y)t + R(h) 
      \] 
      \[ 
        + i\frac{\partial{v}}{\partial{x}}(x, y)s + i\frac{\partial{v}}{\partial{y}}(x, y)t + iG(h) 
      \] 
      \[
        = f(z) + \Big ( \frac{\partial{u}}{\partial{x}}(x, y) + i\frac{\partial{v}}{\partial{x}}(x, y) \Big )h + R(h) + iG(h)
      \] 
      Entonces,
      \[ 
        \frac{f(z + h) - f(z)}{h} = \Big ( \frac{\partial{u}}{\partial{x}}(x, y) + i\frac{\partial{v}}{\partial{x}}(x, y) \Big ) + \frac{R(h) + iG(h)}{h}
      \] 
      Por tanto,
      \[ 
        f'(z_{0}) = \frac{\partial{u}}{\partial{x}}(x_{0}, y_{0}) + i \frac{\partial{v}}{\partial{x}}(x_{0}, y_{0}) 
      \] 
      $\exists f'(z_{0})$ y es continua $\Rightarrow f(z)$ es anlítica.
  \end{enumerate}
\end{dem}

\begin{cor}
  Sea $f: \Omega \subset \mathbb{C} \to \mathbb{C}$ holomorfa, $\Omega$ abierto. Entonces, $f'(z) = 0, \forall z \in \Omega \Rightarrow f$ es constante.
\end{cor}

\begin{theo}
  Si $f(z)$ es diferenciable, entonces la matriz Jacobian $J_{f}: \mathbb{R}^{2} \to \mathbb{R}^{2}$ tiene determinante
  \[ 
    \det J_{f}(z) = | f'(z) |^{2} .
  \] 
\end{theo}

\begin{dem}
  \[ 
    J_{f}(x,y) =
    \begin{pmatrix}
       u_{x} & u_{y} \\
       v_{x} & v_{y}
    \end{pmatrix} 
  \] 
  Por tanto, el determinante es
  \[ 
    \det J_{f}(x,y) = u_{x} v_{y} - u_{y} v_{x} 
  \] 
  Si sustituimos las derivadas parciales de las ecuaciones de Cauchy nos queda
  \[ 
    \det J_{f}(x,y) = u_{x} u_{x} + v_{x} v_{x} 
  \] 
  \[ 
    = | u_{x} + i v_{x} |^{2} 
  \] 
  \[ 
    = | f'(z) |^{2} 
  \] 

\end{dem}
