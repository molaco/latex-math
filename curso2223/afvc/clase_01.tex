\part{Análisis Complejo}
\chapter{Preliminares}
\section{El Plano Complejo}

\begin{defn}[Plano Complejo]
  Definimos los números complejos como el conjunto $\mathbb{C} = \{ (a, b) : a, b \in \mathbb{R} \}$ junto con las operaciones suma y producto \[ (a, b) + (c, d) = (a + c, b + d) \] \[ (a, b) \cdot (c, d) = (ac - bd, bc + ad) \] 
\end{defn}

\begin{obs}
  $(\mathbb{C}, +, \cdot)$ es un cuerpo conmutativo. 
  \begin{enumerate}[label=(\roman*)]
    \item La identidad de la suma es $(0, 0)$ y la identidad del producto es $ (1,0)$.
    \item Se satisfacen la prorpiedad asociativa, la distributiba y la conmutativa.
    \item Todo elemento distinto de cero tiene inverso en $\mathbb{C}$.
  \end{enumerate}
\end{obs}

\begin{obs}
  Consideramos los números reales $\mathbb{R}$ como el subconjunto de los números complejos $\mathbb{C}$ de la forma $(a, 0)$.
  Dado $(a, b) \in \mathbb{C}$ podemos escribir $(a, b) = a(1, 0) + b (0, 1)$. Sea $i = (0, 1)$ entonces $(a, b) = a + ib$. Notese que $i² = (0, 1) \cdot (0, 1) = (-1, 0) \rightarrow 1 \in \mathbb{R}$.
\end{obs}

\begin{obs}
  La parte real de $z = a + ib \in \mathbb{C}$ es $ a$ y se denota $\Re(z) = a$. La parte imaginaria de $ z$ es $b$ y se denota $ \Im (z) = b$.
\end{obs}

\begin{defn}[Módulo]
  Sea $z = a+ib \in \mathbb{C}$, el módulo de $z$ es \[ |z| = \sqrt{a^{2} + b^{2}} \]
\end{defn}

\begin{obs}
  El módulo de un número complejo es la distancia desde el punto del plano hasta el origen.
\end{obs}

\begin{defn}[Conjugado]
  Sea $ z=a +ib \in \mathbb{C} $, el conjugado de $z$ es \[ \overline{z}=a-ib \] 
\end{defn}

\begin{obs}
  El conjugado de un número complejo es su simétrico respecto al eje de coordenadas.
\end{obs}

\begin{prop}
  Se verifican las siguientes propiedades:
  \begin{enumerate}[label=(\roman*)]
    \item $\overline{\overline{z}}=z$ y $\overline{z} = z \Leftrightarrow z \in \mathbb{R}$.
    \item $ z + \overline{z} = 2\Re(z)$ y $z - \overline{z} = 2 \Im(z)$.
    \item $ \overline{z+w}= \overline{z} + \overline{w}$ y $ \overline{-z} = -\overline{z}$
    \item $\overline{zw} = \overline{z} \cdot \overline{w}$ y si $z \neq 0$ entonces $\overline{z^{-1}} = \overline{z}^{-1}$
    \item $|z|^{2} = z \overline{z}$ y $z^{-1} = \frac{\overline{z}}{|z|^{2}},\ \forall z \neq 0.$
    \item $|zw| = |z||w|$, $|\frac{z}{w}|= \frac{|z|}{|w|} \text{ si $(w \neq 0)$}$ y $|z| = |\overline{z}|$
    \item $|z + w| \leq |z| + |w|$. Además, si $\exists t \geq 0 : z=tw$ se tiene $|z+w| = |z| + |w|$.
  \end{enumerate}
\end{prop}

\begin{obs}
  El módulo permite definir una distancia en el plano complejo $ d(z,w) = |z - w|$. De esta forma $\mathbb{C}$ y $\mathbb{R}$ son topológicamente iguales.
\end{obs}

\begin{defn}[Representación polar de un número complejo]
Sea $z = a + ib \in \mathbb{C}$, $z$ representa el punto $(a, b)$ en el plano, cuya expresión en coordenadas polares es $(r\cos\theta, r\sin \theta)$. Y escribimos \[z = r(\cos \theta + i \sin \theta) := re^{i \theta}\] donde $r=|z|$ y $\theta = \arg(z) = \arctg(\frac{b}{a})$.
\end{defn}

\begin{obs}
  Si $-\pi < \theta < \pi$ lo llamamos argumento principal y se denota $\Arg(z)$. El conjunto de todos los posibles argumentos de $z$ es $\{ Arg(z) + 2k\pi : k \in  \mathbb{Z}\}$.
\end{obs}

