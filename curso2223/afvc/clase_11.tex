\section{Aplicaciones Conformes}

\begin{defn}[Vector Tangente]
  Sea $\gamma(t) = x(t) + i y(t)$, $0 \leq t < 1$ una curva diferenciable parametrizada con $z_{0} = \gamma(0)$. Entonces, 
  \[ 
    \gamma'(0) = \lim_{t \to 0} \frac{\gamma(t) - \gamma(0)}{t} = x'(0) + i y'(0) 
  \] 
  es el vector tangente a $\gamma$ en $z_{0}$.
\end{defn}

\begin{defn}[Ángulo entre dos curvas]
  Definimos el ángulo entre dos curvas en $z_{0}$ como el ángulo entre sus vectores tangentes en $z_{0}$
\end{defn}

\begin{theo}
  Sea $\gamma: [0,1] \to \mathbb{C}$ una curva diferenciable parametrizada con $z_{0} = \gamma(0)$ y sea $f(z)$ una función diferenciable en $z_{0}$. Entonces la tangente de la curva $f(\gamma(t))$
  \[ 
    (f \circ \gamma)'(0) = f'(z_{0}) \gamma'(0).
  \] 
\end{theo}

\begin{defn}[Función Conforme]
  Sea $f: A \subset \mathbb{R}^{2} \to \mathbb{R}^{2}$ diferenciable y sean para dos curvas $\gamma_{1}, \gamma_{2}$ con $\gamma_{1}(0) = \gamma_{2}(0) = z_{0}$. Entonces, decimos que $f$ es conforme en $z_{0}$ si las curvas $(f \circ \gamma_{1})$, $(f \circ \gamma_{2})$ tienen $\gamma_{1}'(f(z_{0})) \neq 0$, $\gamma_{2}'(f(z_{0})) \neq 0$ y el ángulo entre $(f \circ \gamma_{1}' )(z_{0})$ y $(f \circ \gamma_{2}')(z_{0})$ es el mismo que el ángulo entre $\gamma_{1}'(z_{0})$ y $\gamma_{2}'(z_{0})$.
\end{defn}

\begin{obs}
  Una función conforme $f: D \to V$ es una función diferenciable con derivadas parciales continuas que es conforme $\forall z \in D$ e inyectiva.
\end{obs}
