\section{Cálculo de Residuos}

\begin{note}
  Si $f$ tiene una singularidad aislada en $z_{0}$, entonces $f$ admite desarrollo de Laurent en un entrono $U \setminus \{ z_{0} \}$ de $z_{0}$
  \[ 
    f(z) = \cdots + \frac{b_{2}}{(z - z_{0})^{2}} + \frac{b_{1}}{(z - z_{0})} + a_{0} + a_{1}(z - z_{0}) + \cdots 
  \] 
  donde $b_{1} = \Res (f, z_{0})$ es el residuo de $f$ en $z_{0}$
\end{note}

\begin{prop}
  Sea $f$ con singularidad aislada en $z_{0}$ y sea $k \geq 0$ el menor entero tal que $\exists \lim_{n \to \infty} (z - z_{0}) f(z)$. Entonces, $f(z)$ tiene un polo de orden $k$ en $z_{0}$. Sea $\phi(z) = (z - z_{0}) f(z)$, entonces $\phi$ se puede definir unicamente en $z_{0}$ tal que $\phi$ es analítica en $z_{0}$ y
  \[ 
    \Res(f, z_{0}) = \frac{\phi^{(k-1)}(z_{0})}{(k-1)!} 
  \] 
\end{prop}

\begin{theo}[de los Residuos]
  Sea $\Omega \subset \mathbb{C}$ simplemente conexo, $\{ z_{1}, \cdots, z_{N} \} \subset \Omega$, $f : \Omega \setminus \{ z_{1}, \cdots, z_{N} \} \to \mathbb{C}$ holomorfa, $\{ z_{1}, \cdots, z_{N} \}$ singularidades. Entonces, 
  \[ 
    \int_{\gamma^{+}}^{} f(z) dz = 2 \pi \sum_{k = 1}^{n} \Res (f, z_{k})
  \] 
\end{theo}

\begin{defn}[Residuos en el Infinito]
  Sea $F(z) = f(\frac{1}{z})$. Entonces, decimos que
  \begin{enumerate}[label=(\roman*)]
    \item $f$ tiene un polo de orden $k$ en $\infty$ si $F$ tiene un polo de orden $k$ en $0$,
    \item $f$ tiene un zero de orden $k$ en $\infty$ si $F$ tiene un zero de orden $k$ en $0$,
    \item $\Res(f, \infty) = -  \Res(\frac{1}{z^{2}}F(z), 0)$.
  \end{enumerate}
\end{defn}

\begin{prop}
  Si $a$ es un polo de orden $m$ de $f$, entonces
  \[
    \Res (f,a) = \frac{g^{(m-1)}(a)}{(m - 1)!}
  \]
  donde $g(z) = (z - a)^{m} f(z)$
\end{prop}
