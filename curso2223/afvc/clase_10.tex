\section{Función Inversa}

\begin{theo}[Función Inversa]
  Sea $f: \Omega \subset \mathbb{C} \to \mathbb{C}$ holomorfa, $z_{0} \in \Omega$ y $f'(z_{0}) \neq 0$. Entonces, existe un entorno $U \subset D: z_{0} \in U$ y un entorno de $V \subset \mathbb{C}: f(z_{0}) \in V$ tal que $f: U \to V$ es biyectiva y $f^{-1}$ es holomorfa con
  \[ 
    (f^{-1})'(f(z)) = \frac{1}{f'(z)} , z \in U.
  \] 
\end{theo}

\begin{dem}
\end{dem}

  Sea $J_{f}(x_{0}, y_{0})$ la matriz Jacobiana de $f$ en $z_{0} = (x_{0}, y_{0})$, por el Teorema 2.3 $\det(J_{f}(z_{0})) = | f'(z_{0}) |^{2} \neq 0$. Entonces, podemos aplicar el Teorema de la Función Inversa Real ya que $J: \mathbb{R}^{2}\to \mathbb{R}^{2}$. Solo falta ver que $\[ J_{f}(z) \]^{-1}$ cumple las ecuaciones de Cauchy-Riemann.
  \[ 
    J_{f}(x, y) = 
    \begin{pmatrix}
       u_{x} & u_{y} \\
       v_{x} & v_{y}
    \end{pmatrix}
  \]
  Entonces la matriz Jacobiana invera es
  \[ (J_{f}(x, y))^{-1} = \frac{1}{\det(J_{f})}
    \begin{pmatrix}
       v_{y} & -u_{y} \\
       -v_{x} & u_{x}
    \end{pmatrix}
  \] 
  y la matriz Jacobiana de la función inversa
  \[ 
    J_{f^{-1}}(x,y) =
    \begin{pmatrix}
       t_{x} & t_{y} \\
       s_{x} & s_{y}
    \end{pmatrix}
  \] 
  Entonce,
  \[ 
    t_{x} = \frac{1}{\det(J_{f})}v_{y} = \frac{1}{\det(J_{f})}u_{x},
  \] 
  \[ 
    s_{x} = -\frac{1}{\det(J_{f})} v_{x} = \frac{1}{\det(J_{f})} u_{y},
  \] 
  \[ 
    t_{y} =  \frac{1}{\det(J_{f})} v_{x},
  \] 
  \[ 
    s_{y} = \frac{1}{\det(J_{f})} v_{y}
  \] 
  las ecuaciones de Cauchy-Riemann se cumplen.

\begin{ejm}
  Sea $w = \log z$ la rama principal del logaritmo. Entonces, $w$ es continua y es la inversa de $z = e^{w}, -\pi < w < \pi$. Como $e^{w}$ es holomorfa con $(e^{w})'\neq 0$, podemos aplicar el Teorema de la Función Inversa. Por tanto, $\log z$ es holomorfa.
  \[ 
    z = e^{\log z} \Rightarrow 
  \] 
  \[ 
    1 = e^{\log z} \frac{d{}}{d{z}}(\log z) = z \frac{d{}}{d{z}}(\log z) \Rightarrow
  \] 
  \[ 
    \frac{d{}}{d{z}}(\log z) = \frac{1}{z}.
  \] 
\end{ejm}

\section{Funciones Harmónicas}

\begin{defn}[Ecuación de Laplace]
  La ecuación
  \[ 
    \frac{\partial{^{2}u}}{\partial{x_{1}^{2}}} + \cdots +  \frac{\partial{^{2}u}}{\partial{x_{m}^{2}}} = 0
  \] 
  se llama ecuación de Laplace.
\end{defn}

\begin{defn}[Laplaciano]
  El operador
  \[ 
    \Delta = \frac{\partial{^{2}}}{\partial{x_{1}^{2}}} + \cdots +  \frac{\partial{^{2}}}{\partial{x_{m}^{2}}}
  \] 
  se llama Laplaciano.
\end{defn}

\begin{obs}
  La ecuación de Laplace se escribe $\Delta u = 0$.
\end{obs}

\begin{defn}[Función Armónica]
  Las funciones que satisfacen la ecuación de Laplace se llaman funciones armónicas. Sea $u: A \to \mathbb{R}$, $u \in C^{2}$ tal que 
  \[ 
    \Delta u = \frac{\partial{^{2}u}}{\partial{x^{2}}} + \frac{\partial^2{u}}{\partial{y}^2} = 0 
  \] 
\end{defn}

\begin{theo}
  Si $f = u + i v$ es holomorfa y $u, v \in C^{2}$. Entonces, $u$ y $v$ son armónicas.
\end{theo}

\begin{obs}
  $u = \Re(f), v = \Im(f)$.
\end{obs}

\begin{dem}
  content
\end{dem}

\begin{defn}[Conjugado Armónico]
  Sea $u: D \subset \mathbb{R} \to \mathbb{R}$ armónica y $v$ armónica tal que $f = u + i v$ es holomorfa. Entonces, decimos que $v$ es el conjugado armónico.
\end{defn}

\begin{ejm}
  $f(z) = z^{2}$, $u = x^{2} + y^{2}$, $v = 2xy$.
\end{ejm}

\begin{theo}
  Sea $D$ un disco abierto o $D = \mathbb{R}^{2}$, $u: D \to \mathbb{R}$ armónica. Entonces, existe $v$ armónica conjugada.
\end{theo}

 \begin{dem}
   content
 \end{dem}

 \begin{cor}
   Toda función armónica es localmente la parte real de una función holomorfa.
 \end{cor}
