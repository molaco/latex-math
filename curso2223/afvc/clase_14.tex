\section{Consecuencias Teorema Cauchy}

\begin{theo}[Desigualdades de Cauchy]
  Sea $\Omega \subset \mathbb{C}$,  $D = \overline{D}(z_{0},R) \subset \Omega$, $f$ holomorfa en $D$. Si $| f(z) | \leq M, \forall z \in \partial{D}$, entonces
  \[ 
    \big | f^{(k)}(z_{0}) \big | \leq \frac{k!}{R^{k}}M, \; \forall k \in \mathbb{N}
  \] 
\end{theo}

\begin{dem}
  Por el teorema de Cauchy
  \[ 
    f_{(k)}(z_{0}) = \frac{k!}{2 \pi i} \int_{\gamma}^{} \frac{f(w)}{(w - z_{0})^{k+1}} dw 
  \] 
  \[ 
    \Rightarrow | f^{(k)}(z_{0}) | = \frac{k!}{2 \pi} \Big | \int_{\gamma}^{} \frac{f(w)}{(w - z_{0})^{k+1}} dw \Big |
  \] 
  Ahora,
  \[ 
    \Big | \frac{f(w)}{(w - z_{0})^{k+1}} \Big | \leq \frac{M}{R^{k+1}}
  \] 
  dado que $| w - z_{0} | = R, \forall w \in \partial D$. Entonces,
  \[ 
     | f^{(k)}(z_{0}) | \leq \frac{k!}{2 \pi} \cdot \frac{M}{R^{k+1}} \cdot L
  \] 
  donde $L$ es la longitud de $\gamma$.
\end{dem}

\begin{theo}[Liouville]
  Sea $f$ entera. Si $\exists M >0 : | f(z) |\leq M, \forall z \in \mathbb{C}$, entonces $f$ es constante.
\end{theo}

\begin{dem}
  Por las desigualdades de Cauchy con $k = 1$, $\forall z_{0} \in \mathbb{C}$ se tiene que
  \[
    | f'(z_{0}) | \leq \frac{M}{R},
  \]
  Entonces, si $R \rightarrow \infty$ tenemos que
  \[
    \frac{M}{R} \xrightarrow[]{ R \rightarrow \infty } 0
  \]
  Por tanto, $| f'(z_{0}) | = 0 \Rightarrow f'(z_{0}) = 0 \Rightarrow f$ es constante.
\end{dem}

\begin{theo}[Teorema Fundamental del Álgebra]
  Sea $a_{0}, \cdots, a_{n} \in \mathbb{C}$, $n \geq 1$, $a_{n} \neq 0$, $P(z) = a_{0} + a_{1} z + \cdots + a_{n} z^{n}$. Entonces, $\exists z_{0} \in \mathbb{C}: P(z_{0}) = 0$.
\end{theo}

\begin{dem}
  Sea $p(z_{0}) \neq 0, \forall z_{0} \in \mathbb{C}$. Entonces, $f(z) = \frac{1}{P(z)}$ es entera $\Rightarrow f(z)$ no es constante dado que $a_{n} \neq 0$. Basta ver que, por el teorema de Liouville, que $f(z)$ es acotada. \\

  Sea $M>0$, a partir de $P(z)$ por la desigualdad triangular
  \[ 
    | P(z) | \geq | a_{n} || z |^{n} - | a_{0} | - \cdots - | a_{n-1} || z |^{n-1} 
  \] 
  Sea $a = | a_{0} | + \cdots + | a_{n-1} |$. Si $z > 1$ entonces
  \[ 
    | P(z) | \geq | z |^{n-1} \Bigg ( | a_{n} || z | - \frac{| a_{0} |}{| z |^{n-1}} - \frac{| a_{1} |}{| z |^{n-2}} - \cdots - \frac{| a_{n-1} |}{1} \Bigg )
  \] 
  \[ 
    \geq | z |^{n-1}(| a_{n} || z | - a)
  \] 
  Sea $ K = \max \{ 1, \frac{M+ a}{| a_{n} |} \}$ entonces,  si $| z | > K \Rightarrow | P(z) | \geq M$. Por tanto, si $| z | > K \Rightarrow \frac{1}{| P(z) |} < \frac{1}{M}$. Pero si $z$ es tal que $| z | \leq K$, entonces $ \frac{1}{P(z)}$ es acotada y en valor absoluto por que es continua, es decir, $\exists L >0 : \frac{1}{| P(z) |} < \max \{ \frac{1}{M}, L \} \Rightarrow | f(z) |$ es acotada en $\mathbb{C}$.
\end{dem}
