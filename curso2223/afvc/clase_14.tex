\begin{theo}[Desigualdades de Cauchy]
  Sea $\Omega \subset \mathbb{C}$,  $D = \overline{D}(z_{0},R) \subset \Omega$, $f$ holomorfa en $D$. Si $| f(z) | \leq M, \forall z \in \partial{D}$, entonces
  \[ 
    \big | f^{(k)}(z_{0}) \big | \leq \frac{k!}{R^{k}}M, \; \forall k \in \mathbb{N}
  \] 
\end{theo}

\begin{dem}
  Por el teorema de Cauchy
  \[ 
    f_{(k)}(z_{0}) = \frac{k!}{2 \pi i} \int_{\gamma}^{} \frac{f(w)}{(w - z_{0})^{k+1}} dw 
  \] 
  \[ 
    \Rightarrow | f_{(k)}(z_{0}) | = \frac{k!}{2 \pi} \Big | \int_{\gamma}^{} \frac{f(w)}{(w - z_{0})^{k+1}} dw \Big |
  \] 
  Ahora,
  \[ 
    \Big | \frac{f(w)}{(w - z_{0})^{k+1}} \Big | \leq \frac{M}{R^{k+1}}
  \] 
  dado que $| w - z_{0} | = R, \forall w \in \partial D$. Entonces,
  \[ 
     | f_{(k)}(z_{0}) | \leq \frac{k!}{2 \pi} \cdot \frac{M}{R^{k+1}} \cdot L
  \] 
  donde $L$ es la longitud de $\gamma$.
\end{dem}

\begin{theo}[Liouville]
  Sea $f$ entera. Si $\exists M >0 : | f(z) |\leq M, \forall z \in \mathbb{C}$, entonces $f$ es constante.
\end{theo}

\begin{dem}
  content
\end{dem}

\begin{theo}[Teorema Fundamental del Álgebra]
  content
\end{theo}
