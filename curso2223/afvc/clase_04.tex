\section{Función Logaritmo}

\begin{defn}[Logaritmo]
  La función logaritmo se define como la inverasa de la función exponencial, 
  \[
    \log: \mathbb{C} \setminus \{ 0 \} \to \mathbb{C}
  \]
  \[
    z \mapsto \log(z) = w
  \]
  donde $\log(z) = w$ es la raíz de la ecuación $e^{w} = z$.
\end{defn}

\begin{obs}
  $e^{z} \neq 0, \forall z \in \mathbb{C} \Rightarrow$ el $0$ no tiene logaritmo.
\end{obs}

\begin{obs}
  Si $w = x + iy \neq 0$, $z = e^{w} = e^{x + iy}$ tiene soluciones 
  \[ 
  e^{x} = |z|, \ e^{iy} = \frac{w}{|w|} 
  \]
  donde la primera ecución tiene solución única $x = \log(|z|)$ y la segunda ecuación tiene inifinitas soluciones módulo $2\pi$.  
\end{obs}

\begin{obs}
  Distinguiendo la parte real y la parte imaginaria de $w$ podemos escribir 
  \[ 
  z = \log(z) = \log |z| + i\arg(z)
  \]
  dado que $e^{\log(z)} = e^{\log |z|} e^{i \arg(z)} = |z|e^{i \arg(z)} = z$.
\end{obs}

\begin{obs}
  La rama principal del logaritmo es
  \[ 
    \Log z = \log | z | + i \Arg(z), \quad z \neq 0
  \] 
  De esta manera, $\Log(z)$ es la inversa de $e^{w}$ con valores en $-\pi < \Im w \leq \pi$. 
\end{obs}

\begin{obs}
  Determinada la rama principal del logaritmo se tiene que
  \[ 
    \log(z) = \Log(z) + i \pi m, \quad m = 0, \pm 1, \pm 2, \cdots
  \] 
  para cualquier otra rama.
\end{obs}

\begin{defn}[Potencias]
  Sea $a,\alpha \in \mathbb{C}, a, \alpha \neq 0$
  \[ 
    a^{\alpha} = e^{\alpha \log(a)} 
  \] 
  
\end{defn}

\begin{obs}
  Si $\alpha = 0 \Rightarrow a^{0} = 1$.
\end{obs}

\begin{obs}
  En general, $a^{\alpha}$ tiene infinitos valores. Una excepción es $\alpha = n \Rightarrow a^{n} = e^{n \log(a)} = e^{\log(a)} \cdot \cdots \cdot e^{\log(a)} = a \cdot \cdots \cdot a$.
\end{obs}

\begin{prop}[Propiedades Potencias]
  El logaritmo verifica las siguientes propiedades:
  \begin{enumerate}[label=(\roman*)]
    \item $a^{-n} = \frac{1}{a^{n}}$
    \item $a^{\alpha + \beta} = a^{\alpha}a^{\beta}$ solo si fijamos el valor de $ \log(a)$
    \item $1^{\aplha} = e^{-2k\pi y}(\cos(2k\pi x) + i \sen(2k\pi x))$ donde $\alpha = x +iy$
  \end{enumerate}
\end{prop}

\begin{prop}
  \begin{enumerate}[label=(\roman*)]
    \item $f(z) = a^{z}$ es continua en $\mathbb{C}$ 
    \item Sea $\alpha \in \mathbb{C}, f(z) = z^{\alpha}$ es continua en el dominio de la rama del logaritmo.
  \end{enumerate}
\end{prop}
