\section{Teorema de Rouché}

\begin{theo}[Rouche]
  Sea $\Omega$ abierto simplemente conexo, $f,g$ holomorfas salvo en $\Omega$ salvo en los ceros y los polos, $\gamma$ curva cerrada en $\Omega$ que no pasa por ningún cero o polo de $f,g$. Si
  \[ 
    | g(z) | < | f(z) |, \quad \forall z \in \gamma,
  \] 
  entonces se tiene que
  \begin{enumerate}[label=(\roman*)]
    \item $\Delta \arg(f) = \Delta \arg(g)$,
    \item $Z_{f} - P_{f} = Z_{g} - P_{g}$.
  \end{enumerate}
\end{theo}

\begin{dem}
  content
\end{dem}

\section{Propiedades Funciones Armónicas}

\begin{prop}
  Sea $D$ un disco abierto, $u(x,y)$ un función armónica en $D$. Entonces, $\exists v(x,y)$ función en $D$ tal que $u + i v$ es analítica en $D$.
\end{prop}

\begin{theo}[Principio del Máximo]
  Sea $D \subset \mathbb{R}^{2}$ abierto conexo, $u : D \to \mathbb{R}$ armónica en $D$. Si $\exists z_{0} \in D : u(z) \leq u(z_{0})$, $\forall z \in D$, entonces $u$ es constante.
\end{theo}

\begin{dem}
  content
\end{dem}

\begin{theo}[Principio del Mínimo] 
  Sea $D \subset \mathbb{R}^{2}$ abierto, conexo y acotado, $u : D \to \mathbb{R}$ armónica tal que $u$ se puede extender con continuidad a la $\partial{D}$. Si $\exists m, M : m \leq u(z) \leq M, \forall z \in \partial{D}$, entonces
  \[ 
    m \leq u(z) \leq M, \quad \forall z \in D 
  \] 
\end{theo}
