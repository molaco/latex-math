\section{Integral de Funciones Complejas sobre Curvas}

\begin{defn}[Integral]
  Sea $h: [a, b] \subset \mathbb{R} \to \mathbb{C}$ un función compleja de una variable real y sean $u, v$ sus partes real e imaginaria respectivamente tal que $h(t) = u(t) + i v(t)$. Suponemos que $u, v$ son continuas. Entonces, llamamos la integral de $h$ a
  \[ 
    \int_{a}^{b} h(t) dt = \int_{a}^{b} u(t) dt + i \int_{a}^{b} v(t) dt,
  \] 
  donde las integrales de $u$ y $v$ tienen el sentido usual de cálculo unidimensional.
\end{defn}

\begin{defn}
  Sea $f$ continua y definida en un conjunto abierto $A \subset \mathbb{C}$, $\gamma: [a, b] \to \mathbb{C}$una curva diferenciable a trozos tal que $\gamma([a, b]) \subset A$. Entonces,
  \[ 
    \int_{\gamma}^{} f = \int_{\gamma}^{} f(z) dz = \sum_{i = 0}^{n-1} \int_{a_{i}}^{a_{i-1}} f(\gamma(t))\gamma'(t) dt
  \]
  es la integral de línea de $f$ a lo largo de $\gamma$.
\end{defn}

\begin{prop}
  Sea $ f(z) = u(x, y)+ i v(x, y)$, entonces
  \[ 
    \int_{\gamma}^{} f = \int_{\gamma}^{} [u(x, y)dx - v(x, y)dy] + i \int_{\gamma}^{} [u(x,y)dy + v(x, y)dx]
  \] 
\end{prop}

\begin{dem}
  \[ 
    f(\gamma(t))\gamma'(t) = [u(x(t), y(tb)) + i v(x(t), y(t))] \cdot [x'(t) + iy'(t)]
  \] 
  \[ 
    = [u(x(t), y(t))x'(t) - v(x(t), y(t))y'(t)] + i[v(x(t), y(t))x'(t) + u(x(t), y(t))y'(t)]
  \] 
  donde integrado sobre $[a_{i}, a_{i +1}]$ tenemos la expresión requerida.
\end{dem}

\begin{defn}[Reparametrización]
  Sea $\gamma: [a, b] \to \mathbb{C}$ una curva diferenciable a trozos. Una curva diferenciable a trozos $\overline{\gamma} [\overline{a}, \overline{b}] \to \mathbb{C}$ se llama reparametrización de $\gamma$ si $\exists \alpha: [a, b] \to [\overline{a}, \overline{b}]$ con $\alpha'(t) > 0, \alpha(a) = \overline{a}$ y $\alpha(b)= \overline{b}$ tal que $\gamma(t) = \overline{\gamma}(\alpha(t))$.
\end{defn}

\begin{prop}
  Si $\overline{\gamma}$ es una reparametrización de $\gamma$, entonces
  \[ 
    \int_{\gamma}^{} f = \int_{\overline{\gamma}}^{} f 
  \] 
  para $f: \Omega \subset \mathbb{C} \to \mathbb{C}$ donde $ \gamma([a, b]) \subset \Omega$.
\end{prop}

\begin{prop}
  Sean $f,g$ funciones continuas, $c_{1}, c_{2} \in \mathbb{C}$, $\gamma_{1}, \gamma_{2}, \gamma$ curvas diferenciables, entonces
  \begin{enumerate}[label=(\roman*)]
    \item $\int_{\gamma}^{} (c_{1} f + c_{2} g) = c_{1} \int_{\gamma}^{} f + c_{2} \int_{\gamma}^{}  g$,
    \item $\int_{-\gamma}^{} f = - \int_{\gamma}^{} f$,
    \item $\int_{\gamma_{1} + \gamma_{2}}^{} f = \int_{\gamma}^{} f + \int_{\gamma}^{} f$.
  \end{enumerate}
\end{prop}

\begin{theo}
  Sea $\gamma$ un curva diferenciable a trozos. Si $h(z)$ es una función continua en $\gamma$, entonces
  \[ 
    \Big | \int_{\gamma}^{} h(z) dz \Big | \leq \int_{\gamma}^{} | h(z) | | dz | .
  \] 
  Además, si $\gamma$ tiene longitud $L$ y $| h(z) | \leq M$ en $\gamma$, entonces
  \[ 
    \Big | \int_{\gamma}^{} h(z) dz \Big | \leq ML .
  \] 
\end{theo}

\begin{obs}
  $\int_{\gamma}^{} | h(z) | | dz | = \int_{a}^{b} | f(\gamma(t)) || \gamma'(t) | dt.$ 
\end{obs}

\begin{dem}
  Sea $g: [a, b] \to \mathbb{C}$, entonces
  \[ 
    \Re \Bigg ( \int_{a}^{b} g(t) dt \Bigg ) = \int_{a}^{b} \Re (g(t)) dt 
  \] 
  dado que $ \int_{a}^{b} g(t) dt = \int_{a}^{b} u(t) dt + i \int_{a}^{b} v(t) dt = u(t) + i v(t)$. \\

  Sea $\int_{a}^{b} g(t) dt = r e^{i \theta}$, entonces $r = \int_{a}^{b} e^{-i \theta} g(t) dt$
  \[ 
    \Rightarrow r = \Re (r) = \int_{a}^{b} \Re (e^{-i \theta} g(t)) dt 
  \] 
  como $\Re (e^{-i \theta} g(t) ) \leq | e^{-i \theta} g(t) | = | g(t) |, \text{ ya que } | e^{-i \theta} | = 1$, entonces tenemos que $\int_{a}^{b} \Re (e^{-i \theta} g(t)) dt \leq \int_{a}^{b} | g(t) | dt$
  \[ 
    \Rightarrow \Bigg | \int_{a}^{b} g(t) dt \Bigg | = r \leq \int_{a}^{b} | g(t) | dt. 
  \] 
  Y usando $| z z' | = | z || z' |$ tenemos que
  \[ 
    | \int_{\gamma}^{} f | = |  \int_{a}^{b} f(\gamma(t))\gamma'(t) dt | \leq \int_{a}^{b} | f(\gamma(t))\gamma'(t) | dt = \int_{a}^{b} | f(\gamma(t)) || \gamma'(t) | dt
  \] 
\end{dem}

\begin{theo}[Fundamental del Cálculo]
  Sea $\gamma: [0,1] \to \mathbb{C}$ una curva diferenciable a trozos, $\Omega \subset \mathbb{C}$ abierto tal que $\gamma([0,1]) \subset \Omega$, $F: \Omega \to \mathbb{C}$ función holomorfa con $F'$ continua. Entonces,
  \[ 
    \int_{\gamma}^{} F'(z) dz = F(\gamma(1)) - F(\gamma(0)) 
  \] 
\end{theo}

\begin{obs}
  Si $\gamma(0) = \gamma(1)$, entonces $\int_{\gamma}^{} F'(z) dz = 0$
\end{obs}

\begin{dem}
  \[ 
    \int_{\gamma}^{} F'(z) dz = \int_{0}^{1} F'(\gamma(t))\gamma'(t) dt = \int_{0}^{1} (F \circ \gamma)'(t) dt = F(\gamma(1)) - F(\gamma(0))
  \] 
\end{dem}

\begin{cor}
  Si $\gamma$ es una curva cerrada, entonces
  \[ 
    \int_{\gamma}^{} f(z) dz= 0 
  \] 
\end{cor}

\begin{theo}
  Si $\Omega \subset \mathbb{C}$ es abierto convexo, y $f: \Omega \to \mathbb{C}$es continua, entonces $f$ tiene primitica en $ \Omega$ si y solo si 
  \[ 
    \int_{\partial T}^{} f(z) dz 
  \] 
  para $T \subset \Omega$ triángulo.
\end{theo}

\begin{dem}
  content
\end{dem}

\section{Teorema de Cauchy}

\begin{defn}[Teorema de Green]
  Sea $D \subset \mathbb{C}$ abierto conexo acotado tal que $\partial D$ es una una curva cerrada y simple, $\Omega$ abierto tal que $\overline{D} \subset \Omega$ y $P,Q: \Omega \to \mathbb{R}$ de clase $C^1$. Entonces,
  \[ 
    \int_{\patial D^+}^{} Pdx Qdy = \iint_{D}^{} (\frac{\partial{Q}}{\partial{x}} - \frac{\partial{P}}{\partial{y}})dxdy
  \] 
\end{defn}

\begin{theo}[Cauchy]
  Sea $D \subset \mathbb{C}$ abierto conexo acotado tal que su frontera es una curva simple cerrada, $\Omega$ abierto tal que $\overline{D} \subset \Omega$, $f: \Omega \to \mathbb{C}$ función holomorfa tal que $f'$ es continua. Entonces,
  \[ 
    \int_{\patial D }^{} f(z) dz = 0
  \] 
\end{theo}

\begin{dem}
  Sea $ f = u + i v $,
  \[ 
    \int_{\gamma}^{} f = \int_{\gamma}^{} f(z) dz 
  \] 
  \[ 
    = \int_{\gamma}^{} (u + i v)(dx + dy) 
  \] 
  \[ 
    = \int_{\gamma}^{} (u dx - v dy) + i \int_{\gamma}^{} (u dy + v dx)
  \] 
  donde aplicando el teorema de Green, tenemos que
  \[ 
    \int_{\gamma}^{} f = \iint_{A}^{} \Bigg[ - \frac{\partial{v}}{\partial{x}} - \frac{\partial{u}}{\partial{y}} \Bigg] dxdy + i \iint_{A}^{} \Bigg[ \frac{\partial{u}}{\partial{x}} - \frac{\partial{v}}{\partial{y}} \Bigg] dx dy
  \] 
  a partir de las ecuaciones de Cauchy-Riemann ambos términos son nulos.
\end{dem}

\begin{theo}[Fórmula Integral de Cauchy]
  Sea $D \subset \mathbb{C}$ abierto, conexo y acotado tal que $\partial D$ es una curva simple cerrada, $\Omega$ abierto tal que $\overline{D} \subset \Omega$. Sea $f: \Omega \to \mathbb{C}$ un función holomorfa tal que $f'$ es continua. Entonces,
  \[ 
    f(z) = \frac{1}{2 \pi i} \int_{\partial D^+}^{} \frac{f(w)}{w - z} dw, \forall z \in D 
  \] 
\end{theo}

\begin{dem}
  Sea $z \in D, \epsilon >0, D_{\epsilon} = D \setminus \{ | w - z | \leq \epsilon \}$. La frontera $\partial D^+$ es la unión de $\partial D$ y $\{ | w - z | = \epsilon \}$ con orientación positiva. \\

  Dado que $ \frac{f(z)}{w-z}$ es holomorfa para $w \in D_{\epsilon}$, por el teorema de Cauchy tenemos
  \[ 
    \int_{\partial D^+}^{} \frac{f(w)}{w - z} dw = 0 
  \] 
  separando la frontera e invirtiendo la orientación se tiene
  \[ 
    \Rightarrow \int_{|  w - z | = \epsilon}^{}  \frac{f(z)}{w-z} dw = \int_{\partial D}^{} \frac{f(w)}{w - z} dw 
  \] 
  si escribimos $w = z + \epsilon e^{i \theta}, dw = i \epsilon e^{i \theta} dw$, obtenemos
  \[ 
    \int_{0}^{2 \pi} f(z + \epsilon e^{i \theta}) \frac{d \theta}{2 \pi} =  \frac{1}{2 \pi i} \int_{\patial D}^{} \frac{f(w)}{w - z} dw
  \] 
  por el teorema del valor medio para la funciones armónicas, la integral de la izquierda coincide con $f(z)$.
\end{dem}

\begin{theo}[Fórmula Integral de Cauchy para las Derivadas]
  Sea $D \subset \mathbb{C}$ abierto, conexo y acotado talque $\partial D$ es unacurva simple cerrada, $\Omega \subset \mathbb{C}$ abierto tal que $\overline{D} \subset \Omega$, $f: \Omega \to \mathbb{C}$ función holomorfa tal que $ f'$ es continua. Entonces,
  \[ 
    f^{(n)}(z) = \frac{n!}{2 \pi i} \int_{\partial D^+}^{} \frac{f(w)}{(w - z)^{n+1}} dw, \forall z \in D, \forall n \in \mathbb{N}.
  \] 
\end{theo}

%\begin{dem}
\begin{dem}
  Sea $z \in D$ entonces $dist (z, \partial D) = r > 0$, $f$ continua en $\partial D \Rightarrow \exists M>0 : | f(w) | < M, \forall w \in \partial D$ y $| \frac{1}{w - z} | \leq \frac{1}{r} $
  \[ 
    | \frac{f(w)}{w-z} |  \leq \frac{M}{r}, \forall w \in \partial D
  \] 
  y dado que $\frac{d{}}{d{z}}(\frac{f(w)}{w -z}) = \frac{f(w)}{w-z}^2$ entonces, por el teorema de derivación bajo el signo integral tenemos que
  \[ 
    f'(z) = \frac{1}{2 \pi i} \int_{\partial D}^{} \frac{f(w)}{(w-z)^2} dz
  \] 
  donde usando inducción y el teorema de derivación bajo el signo integral podemos ver que se cumple para las derivadas de orden $n$.
%\end{dem}
\end{dem}

\begin{cor}
  Sea $f: \Omega \to \mathbb{C}$ función holomorfa, $f': \Omega \to \mathbb{C}$ continua. Entonces $f$ es infinitamente derivable.
\end{cor}

\begin{theo}[Morera]
  Sea $f: \Omega \to \mathbb{C}$ función continua y $\int_{\partial T}^{} f(z) dz = 0, \forall T \subset \Omega$ triángulo. Entonces, $f$ es infinitamente derivable.
\end{theo}

\begin{dem}
  (Teorema fundamental del cálculo) $\Rightarrow f$ tiene primitiva, es decir, $\exists F: f = F', F$ holomorfa en $D$ y $F' = f$  continua. Entonces, por el corolario anterior $F$ infinitamente derivable $\Rightarrow F'$ infinitamente derivable.
\end{dem}
