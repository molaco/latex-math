\chapter{Función Característica}

\section{Función generatriz}

\begin{defn}[Función Generatriz Discreta]
  Sea $X$ v.a. discreta con función de masa $p_{X}$ y soporte $D_{X}$. Entonces, la función generatriz de $X$ es 
  \[ 
    G(s) = E [ s^{X} ] = \sum_{i = 1}^{\infty} s^{x_{i}} p_{i}
  \] 
  siempre que $\sum_{i = 1}^{\infty} |s^{x_{i}}| p_{i} < \infty$.
\end{defn}

\begin{defn}[Función Generatriz Continua]
  Sea $X$ v.a. continua con función de densidad $f$. Entonces, la función generatriz de $X$ es
  \[ 
    G(s) = E [s^{X}] = \int_{- \infty}^{+ \infty} s^{x}f(x) dx
  \] 
  siempre que $\int_{- \infty}^{+ \infty} |s^{X}|f(x) dx$.
\end{defn}

\begin{prop}[Propiedades Función Generatriz]
  La derivada de orden $n$ de la función generatriz es
  \[ 
    G^{(n)}(s) = \sum_{x = n}^{\infty} \binom{x}{n} n! f(x) s^{x - n}
  \] 
  que cumple lo siguiente
  \begin{enumerate}[label=(\roman*)]
    \item $G(0) = P(X = 0) = f(0)$,
    \item $\frac{1}{n!}G^{(n)}(0) = P(X = n) = f(n)$,
    \item $E(X) = G'(1)$,
    \item $G^{(n)}(1) = E [ X(X - 1) \cdots (X - n + 1) ]$
  \end{enumerate}
\end{prop}

\section{Función Generatriz de Momentos}

\begin{defn}[Función Generatriz de Momentos]
  Llamaremos función generatriz de momentos respecto al origen de la variable aleatoria $X$ con función de distribución $F_{X}$ a
  \[ 
    M(\theta) = E [e ^{\theta X}] = \int_{\mathbb{R}}^{} e ^{\theta X} dF_{X}(x)
  \] 
\end{defn}

\subsection{Función Generatriz de Momentos Discreta}

\begin{defn}[Función Generatriz de Momentos Discreta]
  Sea $X$ v.a. discreta con función de masa $p_{X}$. Lamamos función generatriz de momentos a la función
  \[ 
    M(\theta) = \sum_{i = 0}^{\infty} e ^{\theta x_{i}} p_{i} 
  \] 
  siempre que $\sum_{i = 0}^{\infty} |e ^{\theta x_{i}}| p_{i} < \infty$.
\end{defn}

\subsection{Función Generatriz de Momentos Continua}

\begin{defn}[Función Generatriz de Momentos Continua]
  Sea $X$ v.a. continua con función de densidad $f$. Lamamos función generatriz de momentos a la función
  \[ 
    M(\theta) = \int_{-\infty}^{+\infty} e ^{\theta x} f(x) dx
  \] 
  siempre que $\int_{-\infty}^{+\infty} |e ^{\theta x}| f(x) dx$.
\end{defn}

\subsection{Propiedades Función Generatriz De Momentos}

\begin{prop}
  A partir del desarrolo de Taylor de $M(\theta)$ en $\theta = 0$ se tiene
  \[ 
    M(\theta) = \sum_{j = 0}^{\infty} \frac{E(X^{j})}{j!}\theta^{j} 
  \] 
  donde $M^{(j)}(0) = E(X^{j})$.
\end{prop}

\section{Función Característica}

\begin{defn}[Función Característica]
  Sea $(\Omega, \mathcal{A}, P )$ espacio de probabilidad, $X$ v.a. con función de distribución $F_{X}$. Se llama función característica de $X$ a
  \[ 
    \varphi(t) = E [e^{itx}] = \int_{\mathbb{R}}^{} e^{itx} dF_{X}(x)
  \] 
  \[ 
    = \int_{\mathbb{R}}^{} \cos(tx) dF_{X}(x) + i \int_{\mathbb{R}}^{} \sen(tx) dF_{X}(x)
  \] 
\end{defn}

\subsection{Propiedades Función Característica}

\begin{prop}
  Sea $X$ v.a. con función de distribución $F_{X}$. Entonces, la función característica satisface
  \begin{enumerate}[label=(\roman*)]
    \item $\varphi$ existe $\forall t \in \mathbb{R}$,
    \item $\varphi(0) = 1$,
    \item $| \varphi(t) | \leq 1, \forall t \in \mathbb{R}$,
    \item $\varphi(t)$ es uniformemente continua,
    \item Si $Y = aX + b$, entonces $\varphi_{Y}(t) = e^{itb} \cdot \varphi_{X}(at)$.
  \end{enumerate}
\end{prop}

\begin{dem}
  content
\end{dem}

\section{Problema de los Momentos}

\begin{theo}
  Supuesto que los momentos de la variable aleatoria $X$ existen
  \[ 
    \alpha_{n} = \int_{\mathbb{R}}^{} x^{n} dF(x) 
  \] 
  Entonces,
  \begin{enumerate}[label=(\roman*)]
    \item $\exists \varphi^{(n)}(0) = i^{n} \alpha_{n}$
    \item $\exists \varphi^{(n)}(t) = i^{n} \int_{\mathbb{R}}^{} e^{itx} x^{n} dF(x)$
  \end{enumerate}
\end{theo}

\begin{dem}
  content
\end{dem}

\begin{theo}
  \begin{enumerate}[label=(\roman*)]
    \item []
    \item Si $\exists \varphi^{(2n)}(t)$ entonces $\exists \alpha_{2n} = \frac{\varphi^{(2n)}(0)}{i^{2n}}$
    \item Si $\exists \varphi^{(2n - 1)}(t)$ entonces $\exists \alpha_{2n - 2} = \frac{\varphi^{(2n -2)}(0)}{i^{2n - 2}}$
  \end{enumerate}
\end{theo}

\section{Teorema de Inversión}

\begin{theo}[de Inversión]
  Sea $X$ v.a. con función de distribución $F_{X}$ y $\varphi(t)$ su función característica, entonces
  \[ 
    f(x) = \frac{1}{2 \pi} \int_{+\infty}^{-\infty} e^{-itx} \varphi(t) dt
  \] 
\end{theo}
