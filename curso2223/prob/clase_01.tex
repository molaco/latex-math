\chapter{Espacio de Probabilidad}
\section{Experimentos aleatorios}

\begin{defn}[Experimento Determinista]
  Experimeto cuyo desarrolo es previsible con certidumbre y sus resultados están perfectamente determinados una vez fijadas las condiciones del mismo.
\end{defn}

\begin{ejm}
  Averiguar el espacio recorrido por un cuerpo en caída libre en el vacío al cabo de cierto tiempo $t$, donde se sabe que $x = \frac{1}{2}g t^2$ con $g$ la gravedad de la Tierra.
\end{ejm}

\begin{defn}[Experimento Aleatorio]
  Experimento en contexto de incertidumbre. Se caracterizan porque su desarrolo no ese previsible con certidumbre.
\end{defn}

\begin{ejm}
  Lanzar un dado.
\end{ejm}

\section{Espacio Muestral}

\begin{defn}[Espacio Muestral]
  Dado un experimento aleatorio, $\Omega$ es el conjunto de todos los posibles resultados del experimento. Decimos que $\Omega$ es el espacio muestral del experimento y los elementos de $\Omega$ se llaman sucesos elementales.
\end{defn}

\begin{ejm}
  Dado el experiemento "Lanzar un dado y obtener un $6$", el espacio muestral es $\Omega = \{ 1, 2, 3, 4, 5, 6 \}$. Si consideramos "Lanzar un dado y obtener un número par", el espacio muestral sería $\Omega = \{ \text{ par }, \text{ impar } \}$.
\end{ejm}

\subsection{Tipos de Espacios Muestrales}

\begin{defn}[Espacio Muestral Finito]
  Sea $\Omega$ un espacio muestral. Entonces, decimos que $\Omega$ es finito si tiene un número finito de elementos.
\end{defn}

\begin{ejm}
  Lanzar un dado.
\end{ejm}

\begin{defn}[Espacio Muestral Infinito Numerable]
  Sea $\Omega$ un espacio muestral. Entonces, decimos que $\Omega$ es infinito numerable si tiene un número infinito y numerable de elementos.
\end{defn}

\begin{ejm}
  Lanzar una moneda hasta obtener cara por primera vez. Aquí debemos considerar que se puede dar el caso en el que no se obtenga nunca cara y tiremos la moneda infinitas veces.
\end{ejm}

\begin{defn}[Espacio Muestral Continuo]
  Sea $\Omega$ un espacio muestral. Entonces, decimos que $\Omega$ es continuo si no hay discontinuidades o cambios abrutos entre los elementos del espacio muestral.
\end{defn}

\begin{ejm}
  El nivel del agua de un pantano entre los tiempos $t_{1}, t_{2}$. El espacio muestral $\Omega = \{  f_{t} : t \in [t_{1}, t_{2}] \}$.
\end{ejm}

\section{Sucesos}

\begin{note}
  Sea $\mathcal{A} \subset \Omega$. Decimos que se ha presentado el suceso $A \subset \mathcal{A}$ si el resultado del experiemento ha sido $w \in A$, un suceso elemental contenido en $\mathcal{A}$.
\end{note}

\section{Sucesiones de Conjuntos}

\begin{defn}[Sucesión de Conjuntos]
  Sea $\Omega$ espacio muestral, $f: \mathbb{N} \to \mathcal{P}(\Omega)$ una aplicación. Decimos que $f$ es una sucesión de conjuntos y la repesentamos $\{ A_{n} \}_{n \in \mathbb{N}} \subset \mathcal{P}(\Omega)$.
\end{defn}

\section{Límites de una sucesión de conjuntos}

\begin{defn}[Límite Inferior]
  Sea $\Omega$ espacio muestral, $\{ A_{n} \}_{n \in \mathbb{N}} \subset \mathcal{P}(\Omega)$ sucesión de conjuntos. Entoces, el límite inferior de $\{ A_{n} \}_{n \in \mathbb{N}}$ es el conjunto de puntos de $\Omega$ cuyos elementos pertenecen a todos los $A_{n}$ excepto a lo sumo a un número finito de ellso. $\lim \inf A_{n}$.
\end{defn}

\begin{defn}[Límite Superior]
  Sea $\Omega$ espacio muestral, $\{ A_{n} \}_{n \in \mathbb{N}} \subset \mathcal{P}(\Omega)$ sucesión de conjuntos. Entoces, el límite superior de $\{ A_{n} \}_{n \in \mathbb{N}}$ es el conjunto de puntos de $\Omega$ cuyos elementos pertenecen a infinitos $A_{n}$. Y se denota $\lim \sup A_{n}$.
\end{defn}

\begin{obs}
  $A \in \{ A_{2n} \}_{n \in \mathbb{N}} \Rightarrow A \in \lim \sup A_{n}$ pero $A \not \in \lim \inf A_{n}$
\end{obs}

\begin{prop}
  Sea $\Omega$ espacio muestral, $\{ A_{n} \}_{n \in \mathbb{N}} \subset \mathcal{P}(\Omega)$ una sucesión de conjuntos. Entonces,
  \begin{enumerate}[label=(\roman*)]
    \item $\lim \inf A_n = \bigcup_{k=1}^{\infty} \bigcap_{n=k}^{\infty} A_{n}$,
    \item $\lim \sup A_n = \bigcap_{k=1}^{\infty} \bigcup_{n=k}^{\infty} A_{n}$.
  \end{enumerate}
\end{prop}

\begin{dem}
  \begin{enumerate}[label=(\roman*)]
    \item []
    \item
      \begin{enumerate}[label=(\roman*)]
        \item [($\Rightarrow$)]  Sea $w \in \lim \inf A_n$. Entonces, $\exists k \in \mathbb{N} : w \in A_{n}, \forall n \geq k$. Por tanto,
          \[ 
            w \in \bigcap_{n = k}^{\infty} A_{n} \Rightarrow w \in \bigcup_{n = 1}^{\infty} \bigcap_{n = k}^{\infty} A_{n}
          \] 
        \item [($\Leftarrow$)] Sea $w \in \bigcup_{n = 1}^{\infty} \bigcap_{n = k}^{\infty} A_{n}$. Entonces, $\exists k \in \mathbb{N} : w \in \bigcap_{n = k}^{\infty} A_{n} \Rightarrow w \in A_{k} \cap A_{k+1} \cap \cdots \Rightarrow w$ pertenece a infinitos $A_{n}$ salvo a lo sumo a un número finito de ellos.
      \end{enumerate}
    \item 
      \begin{enumerate}[label=(\roman*)]
        \item [($\Rightarrow$)] Sea $w \in \lim \sup A_n$. Entonces, $w \in A_{n}, \forall n \in \mathbb{N}$
          \[
            \Rightarrow w \in \bigcup_{n = k}^{\infty} A_{n} \Rightarrow w \in \bigcap_{n = 1}^{\infty} \bigcup_{n = k}^{\infty} A_{n}.
          \]
        \item [($\Leftarrow$)] Sea $w \in \bigcap_{n = 1}^{\infty} \bigcup_{n = k}^{\infty} A_{n}$. Entonces, $w \in \bigcup_{n = 1}^{\infty} A_{n} \Rightarrow w \in A_{n}, \forall n \in \mathbb{N} \Rightarrow w \in \lim \sup A_n$.
      \end{enumerate}
  \end{enumerate}
\end{dem}

\begin{prop}
  $\forall \{ A_{n} \}_{n \in \mathbb{N}} \subset \mathcal{P}(\Omega) \Rightarrow \lim \inf A_{n} \subset \lim \sup A_n$.
\end{prop}

\begin{dem}
  Sea $w \in \lim \inf A_{n}$. Entonces, $w \in \bigcup_{n = 1}^{\infty} \bigcap_{n = k}^{\infty} A_{n} \Rightarrow \exists k \in \mathbb{N} : w \in \bigcap_{n = k}^{\infty} A_{n} \Rightarrow w \in A_{n}, \forall n \geq k \Rightarrow w \in \bigcup_{n = k}^{\infty} A_{n} \Rightarrow w \in \bigcap_{n = 1}^{\infty} \bigcup_{n = k}^{\infty} A_{n} \Rightarrow w \in \lim \sup A_n$.
\end{dem}

\subsection{Sucesión de conjuntos convergente}

\begin{defn}[Covergencia]
  Sea $\Omega$ un espacio muestral, $\{ A_{n} \}_{n \in \mathbb{N}} \subset \mathcal{P}(\Omega)$ una sucesión. Entonces, decimos que $\{ A_{n} \}_{n \in \mathbb{N}}$ es convergente si y solo si $\lim \inf A_{n} = \lim \sup A_n$.
\end{defn}

\subsection{Sucesiones Monótonas}

\begin{defn}[Sucesión Monótona]
  Sea $\Omega$ un espacio muestral, $\{ A_{n} \}_{n \in \mathbb{N}} \subset \mathcal{P}(\Omega)$ una sucesión. Entonces, decimos que $\{ A_{n} \}_{n \in \mathbb{N}}$ es monótona creciente si y solo si $\forall n \in \mathbb{N}, A_{n} \subset A_{n+1}$. Y decimos que $\{ A_{n} \}_{n \in \mathbb{N}}$ es monótona decreciente si y solo si $\forall n \in \mathbb{N}, A_{n+1} \subset A_{n}$.
\end{defn}

\begin{nota}
  \begin{enumerate}[label=(\roman*)]
    \item []
    \item $\uparrow A_{n}$ sucesión monótona creciente,
    \item $\downarrow A_{n}$ sucesión monótona creciente.
  \end{enumerate}
\end{nota}

\begin{prop}
Sea $\Omega$ un espacio muestral, $\{ A_{n} \}_{n \in \mathbb{N}} \subset \mathcal{P}(\Omega)$ una sucesión monónota. Entonces, $\lim \inf A_{n} = \lim \sup A_n$.
\end{prop}

\begin{dem}
  \begin{enumerate}[label=(\roman*)]
    \item []
    \item Sea $\downarrow A_{n}$. Entonces, $A_{n+1} \subset A_{n} \Rightarrow$
      \[
         \lim \sup A_n = \bigcap_{k=1}^{\infty} \bigcup_{n=k}^{\infty} A_{n} = \bigcap_{k = 1}^{\infty} A_{k}
      \]
      y
      \[
        \lim \inf A_n = \bigcup_{k=1}^{\infty} \bigcap_{n=k}^{\infty} A_{n} = \bigcap_{n = 1}^{\infty} A_{n}
      \]
      Por tanto, $\lim \inf A_n  = \lim \sup A_n$.
    \item Sea $\uparrow A_{n}$. Entonces, $A_{n} \subset A_{n + 1} \Rightarrow$
      \[
         \lim \sup A_n = \bigcap_{k=1}^{\infty} \bigcup_{n=k}^{\infty} A_{n} = \bigcup_{n = 1}^{\infty} A_{n}
      \]
      y
      \[
        \lim \inf A_n = \bigcup_{k=1}^{\infty} \bigcap_{n=k}^{\infty} A_{n} = \bigcup_{k = 1}^{\infty} A_{k}
      \]
      Por tanto, $\lim \inf A_n  = \lim \sup A_n$.
 
  \end{enumerate}
\end{dem}

\section{Estructuras con Subconjuntos}

\subsection{Álgebra}

\begin{defn}[Álgebra]
  Dado el espacio total $\Omega$, una clase $\mathcal{Q} \subset \mathcal{P}(\Omega)$ tiene estructura de álgebra si y solo si
  \begin{enumerate}[label=(\roman*)]
    \item $\Omega, \emptyset \in \mathcal{Q}$,
    \item $\forall A \in \mathcal{Q}, A^{c} \in \mathcal{Q}$
    \item $\forall A, A' \in \mathcal{Q}, A \cap A' \in \mathcal{Q}$,
  \end{enumerate}
\end{defn}

\begin{defn}[$\sigma$-Álgebra]
  Dado el espacio total $\Omega$, una clase $\mathcal{Q} \subset \mathcal{P}(\Omega)$ tiene estructura de $\sigma$-álgebra si y solo si
  \begin{enumerate}[label=(\roman*)]
    \item $\Omega, \emptyset \in \mathcal{Q}$,
    \item $\forall A \in \mathcal{Q}, A^{c} \in \mathcal{Q}$
    \item $\forall \{ A_{j} \}_{j \in J} \subset \mathcal{Q}, \; \bigcap_{j \in J} A_{j} \in \mathcal{Q}$
  \end{enumerate}
\end{defn}

\section{Espacio Medibles}

\begin{defn}[Espacio Medible]
  Sea $\Omega$ espacio muestralm $\mathcal{A} \subset \mathcal{P}(\Omega)$ $\sigma$-álgebra. Entoces, al par $(\Omega, \mathcal{A})$ lo llamamos espacio medible. Los elementos de $\mathcal{A}$ se llaman conjuntos medibles.
\end{defn}

\section{Probabilidad}

\begin{defn}[Medida de Probabilida]
  Sea $(\Omega, \mathcal{A})$ un espacio medible, $P: \mathcal{A} \to \mathbb{R}$ aplicación. Entonces, se dice que $P$ es una medida de probabilidad si cumpe
  \begin{enumerate}[label=(\roman*)]
    \item $\forall A \in \Omega, P(A) \geq 0$,
    \item $P(\Omega) = 1$,
    \item $\forall \{ A_{j} \}_{j \in J} \subset \mathcal{A}: A_{i} \cap A_{j} = \emptyset, \forall j \neq i \Rightarrow$
      \[ 
        P \Big ( \bigcup_{n = 1}^{\infty} A_{n} \Big ) = \sum_{ n = 1 }^{\infty} P(A_{n}).
      \] 
  \end{enumerate}
\end{defn}

\begin{prop}[Propiedades Medida Probabilidad]
  \begin{enumerate}[label=(\roman*)]
    \item []
    \item $P(\emptyset) = 0$,
    \item (Aditividad finita) $\forall \{ A_{j} \}_{j \in J}$ familia finita con elementos disjuntos dos a dos, entonces
      \[ 
        P \Big ( \bigcup_{k = 1}^{n} A_{k} \Big ) = \sum_{ k = 1 }^{n} P(A_{k}),
      \] 
    \item $\forall A \in \mathcal{A}, P(A^c) = 1 - P(A)$,
    \item $\forall A,B \in \mathcal{A}: A \subset B, P(A) \leq P(B)$,
    \item $\forall A \in \mathcal{A}, P(A) \leq 1$,
    \item $\forall A,B \in \mathcal{A}, P(A \cap B) = P(A) + P(B) - P(A \cup B)$
    \item $\forall \{ A_{j} \}_{j \in J} \subset \mathcal{A}$,
      \[
        P \big (\bigcup_{i = 1}^{n} A_{j} \big) = \sum_{j = 1}^{\n} P(A_{j}) - \sum_{j_{1}, j_{1} = 1, j_{1} < j_{2}}^{\n} P(A_{j_{1}} \cap A_{j_{2}}) + \cdots + (-1)^{j+1} P \big ( \bigcap_{j = 1}^{n} A_{j} \big )
      \]
    \item $\forall A, B \in \mathcal{A}, P(A \cup B) \leq P(A) + P(B)$,
    \item $\forall \{ A_{j} \}_{j \in J} \subset \mathcal{A}$ finita
      \[ 
        P \Big ( \bigcup_{j = 1}^{n} A_{j} \Big ) \leq \sum_{j = 1}^{n} P(A_{j})
      \] 
    \item $\forall \{ A_{j} \}_{j \in J} \subset \mathcal{A}$
      \[ 
        P \Big ( \bigcup_{j = 1}^{\infty} A_{j} \Big ) \leq \sum_{j = 1}^{\infty} P(A_{j})
      \] 
    \item $\forall \{ A_{j} \}_{j \in J} \subset \mathcal{A}$,
      \[ 
        P \Big ( \bigcap_{j = 1}^{\infty} A_{j} \Big ) \geq 1 - \sum_{j = 1}^{\infty} P(A^{c}_{j})
      \] 
  \end{enumerate}
\end{prop}

\begin{dem}
  \begin{enumerate}[label=(\roman*)]
    \item Consideramos la sucesión $\{ A, \emptyset, \emptyset, \cdots \}$ con $A \in \mathcal{A}$. Entonces, $\bigcup_{n=1}^{\infty} A = A \cup \emptyset \cup \emptyset \cup \cdots = A$. Por tanto,
      \[ 
        P \Big ( \bigcap_{n = 1}^{\infty} A_{n} \Big ) = \sum_{n = 1}^{\infty} P(A_{n})
      \] 
      \[ 
        \Rightarrow P(A) + \sum_{n = 2}^{\infty} P(A_{n}) = P(A) 
      \]  
      entonces, $P(\emptyset) = 0$.
    \item Se toma la sucesión $\{ A_{1}, \cdots, A_{n}, \emptyset, \cdots \}$ donde $A_{j} \in \mathcal{A}, \forall j \in J$ disjuntos dos a dos. Como $\bigcup_{j = 1}^{\infty} A_{j} = \bigcap_{j = 1}^{n} A_{j} $, entonces
      \[ 
        P(\bigcup_{j = 1}^{\infty} A_{j}) = \sum_{j = 1}^{\infty} P(A_{j})
      \] 
      \[ 
        \Rightarrow P(\bigcup_{j = 1}^{n} A_{j}) = \sum_{j = 1}^{n} P(A_{j})
      \] 
    \item $P(A \cup A^{c}) = P(\Omega) = 1 \Leftrightarrow P(A) + P(A^{c}) = 1 \Leftrightarrow P(A^{c}) = 1 - P(A)$.
    \item Podemos escrbir $B = A \cup (B \setminus A)$. Entonces,
      \[ 
        P(B) = P(A) + P(B \setminus A) 
      \] 
      donde $P(B \setminus A) > 0$,
      \[ 
         \Rightarrow P(B) \geq P(A) 
      \] 
    \item Sea $A \in \mathcal{A} \subset \mathcal{P}(\Omega)$, entonce $A \subset \Omega$. Por tanto, $P(A) \geq P(\Omega) = 1$.
    \item 
    \item 
    \item 
    \item 
    \item 
    \item 
    \item 
  \end{enumerate}
\end{dem}

\section{Espacio de Probabilidad}

\begin{defn}[Espacio de Probabilidad]
  Sea $\Omega$ espacio muestra, $\mathcal{A} \subset \mathcal{P}(\Omega)$ $\sigma$-álgebra, $P$ medida de probabilidad. Entonces, a la terna $(\Omega, \mathcal{A}, P)$ se le llama espacio de probabilidad. Los elementos de $\mathcal{A}$ se llaman sucesos.
\end{defn}

\section{Continuidad Secuencial de la Probabilidad}

\begin{theo}
  Sea $(\Omega, \mathcal{A}, P)$ un espacio de probabilidad, $\{ A_{j} \}_{j \in J} \subset \mathcal{A}, \uparrow A_{j}$. Entonces, 
  \[ 
    P(\lim_{n \to \infty} A_{n}) = \lim_{n \to \infty} P(A_{n}).
  \] 
\end{theo}

\begin{dem}
  $A_{n} \uparrow \Rightarrow \lim_{n \to \infty} A_{n} = \bigcup_{n = 1}^{\infty} A_{n} $. Sea $A$ tal que
  \[
    A = A_{1} \cup \Bigg[ \bigcup_{j = 1}^{\infty} (A_{j+1} - A_{j}) \Bigg]
  \]
  entonces, $A$ es unión de conjuntos disjuntos. Aplicado la aditividad finita tenemos que 
  \[ 
    P(A) = P(A_{1}) + \sum_{j =1}^{\infty} (A_{j+1} - A_{j})  
  \] 
  \[ 
    = P(A_{1}) + \lim_{n \to \infty} \sum_{j = 1}^{n} (P(A_{j+1}) - P(A_{j})) 
  \] 
  \[ 
    = \lim_{n \to \infty} \big ( P(A_{1}) + P(A_{2}) - P(A_{1}) + P(A_{3}) - P(A_{2}) + \cdots + P(A_{n+1})- P(A_{n}) \big )
  \] 
  \[ 
    = \lim_{n \to \infty} P(A_{n+1}) = \lim_{n \to \infty} P(A_{n}) .
  \] 
\end{dem}

\begin{theo}
  Sea $(\Omega, \mathcal{A}, P)$ un espacio de probabilidad, $\{ A_{j} \}_{j \in J} \subset \mathcal{A}, \downarrow A_{j}$. Entonces, 
  \[ 
    P(\lim_{n \to \infty} A_{n}) = \lim_{n \to \infty} P(A_{n}).
  \] 
\end{theo}

\begin{dem}
  $A_{n} \downarrow \Rightarrow \exists \lim_{n \to \infty} A_{n} = \bigcap_{n=1}^{\infty} A_{n} = A$ y $A^{c}_{n} \uparrow \Rightarrow$ (por la proposición anterior) 
  \[ 
    P(\lim_{n \to \infty} A^c_{n}) = \lim_{n \to \infty} P(A^{c}_{n})  
  \] 
  donde $\lim_{n \to \infty} A^c_{n} = A^c$. \\ 

  Ahora,
  \[ 
    P(\lim_{n \to \infty} A_{n}) = P(A) = 1 - P(A^c) 
  \]
  \[ 
    = 1 - P(\lim_{n \to \infty} A_{n}^c)  
  \] 
  \[ 
    = 1 - \lim_{n \to \infty} P(A_{n}^c) 
  \]
  \[ 
    = 1 - \lim_{n \to \infty} \big\{ 1 - P(A_{n}) \big\}  
  \] 
  \[
    = 1 - 1 + \lim_{n \to \infty} P(A_{n}) = \lim_{n \to \infty} P(A_{n})
  \] 
\end{dem}

\section{Probabilidad Condicionada}

\begin{defn}[Probabilida Condicionada]
  Sea $(\Omega, \mathcal{A}, P )$ un espacio de probabilidad y se $ A \subset \mathcal{A}$ un suceso tal que $P(A) > 0$. Entonces, decimos que
  \[ 
     P(B | A) = \frac{P(A \cap B)}{P(A)} , P(A) > 0
  \] 
  es la probabilidad de $B$ condiconada por $A$.
\end{defn}

\subsection{Teorema del producto}

\begin{theo}[Regla multiplicación]
  Sea $(\Omega, \mathcal{A}, P )$ un espacio de probabilidad, $A,B \mathcal{A}: P(A), P(B) > 0 $. Entonces,
  \[ 
    P(A \cap B) = P(A) \cdot P(B | A)  \text{ y}
  \] 
  \[ 
    P(A \cap B) = P(B) \cdot P(A | B) 
  \] 
\end{theo}

\subsection{Teorema de Probabilidad Total}

\begin{theo}[Probabilidad Total]
  Sea $(\Omega, \mathcal{A}, P )$ espacio de probabilidad, $\{ A_{n} \}_{n \in \mathbb{N}}\subset \mathcal{A}: A_{i} \cap A_{j} = \emptyset, \forall i \neq j, \bigcup_{n = 1}^{\infty} A_{n} = \Omega $. Entonces, para $B \in \mathcal{A}$
  \[ 
    P(B) = \sum_{ j = 1 }^{\infty}  P(B | A_{j})\cdot P(A_{j})
  \] 
  donde $P(A_{j}) > 0, \forall j \in \{ 1, 2, \cdots \}$
\end{theo}

\begin{dem}
  \[
    P(B) = P( B \cap \Omega) 
  \]
  \[
    = P \Big ( B \cap \Big[ \bigcup_{i=1}^{\infty} A_{i} \Big] \Big ) 
  \]
  \[
    = P \Big (\bigcup_{i = 1}^{\infty} (B \cap A_{i}) \Big )
  \]
  \[ 
    = \sum_{i = 1}^{\infty} B \cap A_{i} 
  \] 
  \[ 
    = P(B | A_{i}) \cdot P(A_{i}) , \ \forall i \in \mathbb{N}.
  \] 
\end{dem}

\begin{theo}[de Bayes]
  Sea $(\Omega, \mathcal{A}, P )$ espacio de probabilidad, $\{ A_{n} \}_{n \in \mathbb{N}} \subset \mathcal{A}$ tal que $P(A_{i}) >0, \forall i \in \mathbb{N}, B \in \mathcal{A}: P(B) > 0$. Entonces,
  \[ 
    P(A_{i} | B) = \frac{P(A_{i})\cdot P(B | A_{i})}{\sum_{i = 1}^{\infty} P(A_{i}) P(B | A_{i})}, i \in \mathbb{N}.
  \] 
\end{theo}

\begin{dem}
  \[ 
    P(A_{i} | B ) = \frac{P(A_{i} \cap B)}{P(B)} 
  \] 
  usando la independencia de sucesos y el teorema de la probaibilidad total tenemos que
  \[ 
    P(A_{i} | B)= \frac{P(A_{i}) \cdot P(B | A_{i})}{\sum_{i = 1}^{\infty} P(A_{i}) \cdot P(B | A_{i})}, i \in \mathbb{N}.
  \] 
\end{dem}

\section{Independencia de Sucesos}

\begin{defn}
  Sea $(\Omega, \mathcal{A}, P )$ un espacio de probabilidad, $A, B \in \mathcal{A}$ con $P(B) > 0$. Entonces, $A$ y $B$ se dicen independientes si y solo si 
  \[ 
    P( A \cap B) = P(A) \cdot P(B) .
  \] 
\end{defn}

\begin{prop}
  Sea $(\Omega, \mathcal{A}, P )$ un espacio de probabilidad, $A, B \in \mathcal{A}$ tal que $A$ y $B$ son sucesos independientes. Entoces, 
  \[ 
    P(A | B) = P(A) \text{ si } P(B) > 0 \text{ y } 
  \] 
  \[ 
    P(B | A) = P(B) \text{ si } P(A) > 0.
  \] 
\end{prop}

\begin{prop}
  Sea $(\Omega, \mathcal{A}, P )$ un espacio de probabilidad, $A, B \in \mathcal{A}$ tal que $A$ y $B$ son sucesos independientes. Entonces, también lo son $A^{c}$ y $B^{c}$, $A^{}$ y $B^{c}$, $A^{c}$ y $B^{}$.
\end{prop}
