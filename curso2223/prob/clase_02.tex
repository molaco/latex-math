\chapter{Modelo Uniforme}

\section{Regla de Laplace}

\begin{prop}[Regla de Laplace]
  Sea $(\Omega, \mathcal{A}, P )$ espacio de probabilidad tal que el conjunto de sucesos elementales es finito, los sucesos elementales son incompatibles dos a dos y equiprobables. Entonces, si $A \in \mathcal{A}$
  \[ 
    P(A) = \frac{\text{ número de sucesos elementales a favor de A}}{\text{número de sucesos elemenetales de $\Omega$}} 
  \] 
  a este resultado lo llamamos Regla de Laplace
\end{prop}

\begin{dem}
  Sea $a_{1}, a_{2}, \cdots, a_{n}$ el conjunto de sucesos elementales asociados, entonces
  \[
    \Omega = a_{1} \cup a_{2} \cup \cdots \cup a_{n},
  \]
  por ser incompatibles dos a dos
  \[ 
    P(a_{1}) + P(a_{2}) + \cdots + P(a_{n}) = 1
  \] 
  y por ser equiprobables, es decir, $P(a_{i}) = \frac{1}{n}, \forall i \in \{  1, \cdots, n \}$. Si $A \in \mathcal{A}$ tal que $A = \bigcup_{j \in J} a_{j}$ donde $ J = \{ 1, \cdots, k \}, k \leq n$, entonces
  \[ 
    P(A) = P(a_{1}) + \cdots + P(a_{k}) = \frac{k}{n} 
  \] 
  Así, hemos obtenido la Regla de Laplace.

\end{dem}

\section{Población y Muestra}

\begin{note}
  Dentro del muestreo aleatorio se distingue que la selección sea sin remplazamiento o con remplazamiento.
\end{note}

\begin{defn}[Selección sin Remplazamiento]
  Se seleccionan $n$ elementos de la población, mediante $n$ extracciones sucesivas sin remplazamiento, asignando en cada una de ellas probabilidades iguales a los elementos no seleccionados en las anteriores. En, este caso, $n$ es menor o igual que el tamaño de la población.
\end{defn}

\begin{defn}[Selección con Remplazamiento]
  Se seleccionan $n$ elementos de la población, mediante $n$ extracciones sucesivas con reemplazamiento, asignando en cada una de ellas probabilidades iguales a todos los elementos de la población.
\end{defn}

\begin{note}
  Distinguimos muestras ordenadas y sin ordenar.
\end{note}

\section{Muestras Ordenadas}

\begin{nota}
  $(N)_{n} = N \cdot (N-1) \cdot \cdots \cdot (N- n +1), \forall n \leq N$.
\end{nota}

\begin{prop}
  Sea $A = \{ a_{1}, \cdots, a_{n} \}$ y $B = \{ b_{1}, \cdots, b_{m} \}$. Entonces, es posible formar $n \cdot m$ pares tales que $(a_{i}, b_{i})$ donde $a_{i} \in A, b_{j} \in B$
\end{prop}

\begin{obs}
  El par $(a_{i}, b_{j})$ y el par $(b_{j}, a_{i})$ son iguales.
\end{obs}

\begin{prop}
  Sea $A_{1}, A_{2}, \cdots, A_{k}$ con $n_{1}, n_{2}, \cdots, n_{k}$ elementos. Entonces el número de ordenaciones de la forma $(x_{1}, x_{2}, \cdots, x_{k}) : x_{i} \in A_{i}, i \in \{ 1, \cdots, k \}$ es $n_{1} \cdot n_{2} \cdot \cdots \cdot n_{k}$.
\end{prop}

\begin{cor}
  $k$ selecciones sucesivas con exactamente $n_{i}$ opciones posibles en el $i$-ésimo paso, producen $n_{1} \cdot \cdots \cdot n_{k}$ resultados diferentes posibles.
\end{cor}

\begin{theo}
  De una población de $N$ elementos se pueden seleccionar $N^{n}$ muestras diferentes con remplazamiento de tamaño $n$ y $(N)_{n}$ muestras diferentes sin reemplazamiento de tamaño $n$.
\end{theo}

\begin{theo}
  El número de ordenaciones diferentes de $N$ elementos es 
  \[ 
    N! = N \cdot (N -1) \cdot \cdots \cdot 2 \cdot 1 
  \] 
\end{theo}

\begin{theo}
  Si se realiza un muestreo aleatorio con remplazamiento de tamaño $n$ de una población con $N$ elementos, la probabilidad de que en la muestra no aparezca ningún elemento dos veces es 
  \[ 
    p = \frac{(N)_{n}}{N^{n}} = \frac{N \cdot (N -1) \cdot \cdots \cdot (N-n+1)}{N^{n}} 
  \] 
\end{theo}

\section{Subpoblaciones}

\begin{defn}[Subpoblación]
  Una Subpoblación de tamaño $n$ es  una muestra de tamaño $n$ extraída de una población de tamaño $N$, cuyos elementos extraidos no han considerado ningún orden.
\end{defn}

\begin{nota}
  \[
    \binom{N}{n} = \frac{N!}{n! \cdot (N-n)!}
  \]
\end{nota}

\begin{theo}
  De una población de $N$ elementos se pueden seleccionar $\binom{N}{n}$ subpoblaciones diferenetes de tmaño $n \leq N$.
\end{theo}

\begin{dem}
  El número de subpoblaciones posibles de tamaño $n$ de una población $N$ es el número de ordenaciones distintas de $n$ elementos que es $n!$. Además, de una población de $N$ elementos se pueden seleccionar $(N)_{n}$ muestras diferentes sin remplazamiento de tamaño $n$. Entonces,
  \[ 
    A = \frac{(N)_{n}}{n!}
  \] 
\end{dem}

\begin{ejm}
  Un equipo está compuesto por $7$ miembros y un club cuenta con $20$ miembros, se podran formar $\binom{20}{7}$ equipos diferentes.
\end{ejm}

\begin{theo}
  De una población de $N$ elementos se pueden seleccionar $\binom{N + n -1}{n}$ subpoblaciones diferentes de tamaño $n$, mediante un muestreo con remplazamiento.
\end{theo}

\section{Prticiones}

\begin{defn}[Partición]
  Una partición de tamaño $r$ de una población de tamaño $N$ es una división de la población en $r$ grupos ordenados de elementos desordenados donde el grupo $i$ contine $n_{i}$ elementos $\forall i \in \{ 1,2, \cdots, r \}$ y
  \[ 
    n_{1} + n_{2} + \cdots + n_{r} = N 
  \] 
\end{defn}

\begin{theo}
  El número de particiones diferentes de tamaño $r$ en las cuales se puede divir una población de $N$ elementos es
  \[ 
    \frac{N!}{n_{1}! \cdot n_{2}! \cdots n_{r}!} 
  \] 
  siendo $n_{i}$ el tamaño del grupo $i$, $i \in \{ 1, \cdots, r \}$.
\end{theo}

\begin{ejm}
  Se lanza un dado en 10 ocasiones. El número total de formas en las cuales se pueden obtener $3$ unos, ningún dos, 2 treses, ningún cuatro, 3 cincos y 2 seises es
  \[ 
    \frac{10!}{3! \cdot 0! \cdot 2! \cdot 0! \cdot 3! \cdot 2!} 
  \] 
\end{ejm}

%\section{Coeficientes Binomiales}

\section{Variaciones, Combinaciones y Permutaciones}

\subsection{Variaciones de $N$ elementos tomados de $n$ en $n$}

\begin{defn}[Variaciones sin repetición]
  Las variaciones de $N$ elementos tomados de $n$ en $n$ son los diferentes grupos que se pueden formar a partir de $N$ elementos, tomados de $n$ en $n$. Cada dos grupos difieren entre sipor algún elemento o por el orden.
  \[ 
    V_{N, n} = (N)_{n} = N \cdot (N - 1) \cdot (N - n +1)
  \] 
\end{defn}

\begin{obs}
  Es lo mismo que el número de muestras diferentes de tamaño $n$ seleccionadas mediante un muestreo sin remplazamiento de una poblaciçon de tamaño $N$.
\end{obs}

\subsection{Variaciones de $N$ elementos tomados de $n$ en $n$}

\begin{defn}[Variaciones con repetición] 
  Las variaciones repetición de $N$ elementos tomados de $n$ en $n$ son los diferentes grupos que se pueden formar a partir de $N$ elementos, tomados de $n$ en $n$, en los que pueden aparecer elementos repetidos y dos grupos son distintos entre sí, tiene distintos elementos o estan situados en distintos lugares.
  \[ 
    RV_{M,n}^{N} = N^n
  \] 
\end{defn}

\subsection{Permutaciones de $N$ elementos}

\begin{defn}[Permutación]
  Las Permutaciones de $N$ elementos diferentes son los distintos grupos que pueden formarse entrando en cada uno de llos lo $N$ elementos dados, difiriendo únicamente en el orden de sucesión de sus elementos.
  \[ 
    P_{N} = N! = N \cdot (N-1) \cdots 2 \cdot 1 
  \]  
\end{defn}

\subsection{Permutaciones con repetición}

\begin{defn}[Permutaciones con repetición]
  Las permutaciones con repetición de $r$ elementos distintos tales que el elemento $i$ aparece $n_{i}$ veces $\forall i \in \{ 1, 2, \cdots, r \}$ con $ \sum_{i = 1}^{r} n_{r} = N$ es
  \[ 
    \frac{N!}{n_{1}! \cdot n_{2}! \cdots n_{r}!} 
  \] 
\end{defn}

\subsection{Combinaciones de $N$ elementos tomados de $n$ en $n$}

\begin{defn}[Combinaciones sin repetición]
  Son los diferente grupos que se pueden formar con $n$ elementos en cada uno, donde por lo menos cada uno tiene un elemento distinto. No se tiene en cuenta el órden en la disposición.
  \[ 
    C_{N,n} = \binom{N}{n} = \frac{N!}{n! \cdot (N-n)!}
  \] 
\end{defn}

\subsection{Combinaciones con repetición de $N$ elementos tomados de $n$ en $n$}

\begin{defn}[Combinaciones con repetición]
  Son la distintas disposiciones que se pueden formar tomando $n$ elementos de los $N$, entre lo cuales puden aparecer elementos repetidos, y dos disposicones serán distintas entre sí, si tienen distintos elementos. No se tiene en cuenta el órden en la disposición.
  \[ 
    RC_{N, n}  = \binom{N+ n -1}{n} = \binom{N + n -1}{N -1} = \frac{((N + n -1))!}{(N-1)! n!}
  \] 
\end{defn}
