\chapter{Vectores Aleatorios}

\section{Función Distribución en $\mathbb{R}^{2}$}

\begin{defn}[Distribución Propia]
  Sea $ (\mathbb{R}^{2}, \mathbb{B}_{2}) $, $F : \mathbb{R}^{2} \to \mathbb{R}$ función de distribución. Si se verifica que
  \begin{enumerate}[label=(\roman*)]
    \item $\forall a, b \in \mathbb{R}^{2} : a \leq b$ se tiene que $F(a, b] \geq 0$.
    \item $F$ es continua por la derecha, es decir, $\lim_{n \to \infty} F(x_{n}) = F(x)$, donde $( x_{n} )_{n \in \mathbb{N}} \subset \mathbb{R}^{2}$.
    \item $\lim_{x_{1}, x_{2} \rightarrow \infty} F(x_{1}, x_{2}) = 1$.
    \item $\lim_{x_{1} \rightarrow - \infty} F(x_{1}, x_{2}) = 0, \forall x_{2}$ y $\lim_{x_{2} \rightarrow - \infty} F(x_{1}, x_{2}) = 0, \forall x_{1}$
  \end{enumerate}
\end{defn}

\begin{theo}
  Sea $P$ probabilidad en $ (\mathbb{R}^{2}, \mathbb{B}_{2}) $, $F : \mathbb{R}^{2} \to \mathbb{R}$ definida por
  \[ 
    F(x_{1}, x_{2}) = P(-\infty, x].
  \] 
  Entonces, $F$ es una función de distribución propia.
\end{theo}

\begin{obs}
  A la función de distribución propia $F$ se le llama función de distribución asociada a la probabilidad $P$.
\end{obs}

\begin{defn}[Vector Aleatorio]
  Sea $(\Omega, \mathcal{A}, P )$ espacio de probabilidad. Entonces, la aplicación
  \[ 
    X = (X_{1}, \cdots, X_{n}) : (\Omega, \mathcal{A}, P ) \to \mathbb{R}^{n}
  \] 
  definida por
  \[ 
    X(\omega) = (X_{1}(\omega), \cdots, X_{n}(\omega)), \quad w \in \Omega
  \] 
  es una variable aleatoria $n$-dimensional si y solo si
  \[ 
    X^{-1}(B) = (X_{1}, X_{2}, \cdots, X_{n})\in(B) \in \mathcal{A}, \quad \forall B \in B_{n} 
  \] 
\end{defn}

\begin{theo}
  $X = (X_{1}, \cdots, X_{n}) $ es una variable aleatoria multidimensional si y solo si $X_{i}$ es variable aleatoria unidimensional $\forall i \in \{ 1, \cdots n \}$.
\end{theo}

\section{Ley de Probabilidad de un Vector Aleatorio}

\begin{prop}[Probabilidad Inducida por $X$ en $\mathbb{R}^n$]
  Sea $(\Omega, \mathcal{A}, P )$ espacio de probabilidad, $X = (X_{1}, \cdots, X_{n})$ vector aleatorio. Entonces, $X$ induce una probabilidad 
  \[ 
    \mathbb{P}_{X} \{ B \} = \mathbb{P} \{ X^{-1}(B) \} = \mathbb{P} \{ A \}
  \] 
  donde $X(A) = B, \forall B$. A la aplicación $\mathbb{P}_{X}$ la llamamos probabilidad inducida por $X$ en $\mathbb{R}^n$.
\end{prop}

\section{Función de Distribución Inducida por Variable Aleatoria Unidimensional}

\begin{theo}
  Sea $F : \mathbb{R}^{2} \to [0,1] $ una función. Entonces, $F$ es una función de distribución si y solo si se verifica
  \begin{enumerate}[label=(\roman*)]
    \item $F(-\infty, y) = 0 = F(x, -\infty) $.
    \item $F( + \infty, + \infty) = 1$.
    \item $F(x,y)$ es continua por la derecha.
    \item $F(x,y) \geq 0, \forall (x,y) \in \mathbb{R}^{2}$.
  \end{enumerate}
\end{theo}

\begin{obs}
  Este teorema se puede generalizar al caso $n$-dimensional.
\end{obs}

\section{Función de Masa en $\mathbb{R}^{2}$}

\begin{defn}[Función de Masa en $\mathbb{R}^{2}$]
  Sea $(X,Y)$ v.a. en $(\Omega, \mathcal{A}, P )$ espacio de probabilidad. Entonces, la función $p_{XY} : \mathbb{R}^{2} \to \mathbb{R}$ definida por
  \[ 
    p_{XY}(x,y) = \mathbb{P} \{ X = x, Y = y \}
  \] 
  \[ 
    = F(x,y) - F(x^{-}, y) - F(x, y^{-}) + F(x^{-}, y^{-}) \geq 0, \quad \forall (x,y) \in \mathbb{R}^{2}
  \] 
  se llama función de masa de $(X,Y)$.
\end{defn}

\begin{theo}
  Sea $(X,Y)$ v.a. con función de masa $p_{XY}$ y sea
  \[ 
    D_{XY} = \{ (x, y) \in \mathbb{R}^{2} : p_{XY}(x,y) > 0 \}.
  \] 
  Entonces, $D_{XY}$ es un conjunto numerable.
\end{theo}

\section{Variable Aleatoria Bidimensional Discreta}

\begin{defn}[Variable Aleatoria Bidimensional Discreta]
  Sea $(X,Y)$ v.a. con función de masa $p_{XY}$ y sea 
  \[ 
    D_{XY} = \{ (x,y) \in \mathbb{R}^{2} : p_{XY}(x,y) > 0 \}
  \] 
  donde $D_{XY} \neq \emptyset$ y $\sum_{(x,y) \in D_{XY}} p_{XY}(x,y) = 1$. Entonces, $(X, Y)$ se llama v.a. discreta y $D_{XY}$ es su soporte.
\end{defn}

\begin{defn}[Punto de Salto]
  Sea $(X, Y)$ v.a. discreta, entonces $\forall (x,y) \in \mathbb{R}^{2}$ tal que
  \[
    p_{XY}(x,y) > 0
  \]
  se llama punto de salto o discontinidad de salto. Y $D_{XY}$ es el conjunto de puntos de salto.
\end{defn}

\begin{defn}[Función de Distribución de Variable Aleatoria Bidimensional Discreta]
  Sea $(X, Y)$ v.a. discreta. Entonces,
  \[ 
    F(x,y) = \sum_{x_{i} \leq x, \ y_{i} \leq y} p_{XY}(x_{i}, y_{i})
  \]
  es su función de distribución.
\end{defn}

\begin{theo}
  Una colección de números no negativos $\{ p(x,y): x = 1, 2,\cdots ; y = 1, 2, \cdots \}$ que satisface
  \[ 
    \sum_{x = 1, y = 1}^{\infty} p(x, y) = 1
  \] 
\end{theo}

\section{Función De Densidad en $\mathbb{R}^{2}$}

\begin{defn}[Función de Densidad en $\mathbb{R}^{2}$]
  Una función $f : \mathbb{R}^{2} \to \mathbb{R}$ se llama función de densidad si 
  \begin{enumerate}[label=(\roman*)]
    \item $\forall (x,y) \in \mathbb{R}^{2}, f(x,y) \geq 0$.
    \item $f$ es integrable Riemann en $\mathbb{R}^{2}$.
    \item $\int_{- \infty}^{+ \infty} \int_{-\infty}^{+\infty} f(x,y) dx dy = 1$.
  \end{enumerate}
\end{defn}

\begin{theo}
  Sea $f$ un función de densidad en $\mathbb{R}^{2}$. Definimos la función
  \[ 
    F : \mathbb{R}^{2} \to \mathbb{R} 
  \] 
  \[ 
    (x,y) \mapsto F(x, y) = \int_{-\infty}^{x} \int_{-\infty}^{y} f(u,v) du dv
  \] 
  Entonces, $F$ es función de distribución en $\mathbb{R}^{2}$.
\end{theo}

\begin{theo}
  Sea $f$ función de densidad en $\mathbb{R}^{2}$ y $F$ función de distribución en $\mathbb{R}^{2}$ definida por
  \[ 
    F(x,y) = \int_{-\infty}^{x} \int_{-\infty}^{y} f(u,v) du dv, \quad \forall (x,y) \in \mathbb{R}^{2}
  \] 
  Si $f$ es continua y $F \in \mathcal{C}^{2}$, entonces
  \[ 
    \frac{\partial^2{F(x_{1},y_{1})}}{\partial{x} \partial{y}} = \frac{\partial^2{F(x_{1},y_{1})}}{\partial{y} \partial{x}} = f(x_{1}, y_{1})
  \] 
\end{theo}

\section{Variable Aleatoria Bidimiensional Continua}

\begin{defn}[Variable Aleatoria Bidimensional Continua]
  Sea $(X,Y)$ v.a. Si su función de distribución $F$ se puede expresar como
  \[ 
    F(x,y) = \int_{-\infty}^{x} \int_{-\infty}^{y} f(u,v) du dv, \quad \forall (x,y) \in \mathbb{R}^{2}.
  \] 
  Entonces, se dice que $(X,Y)$ es v.a. continua y
  \[ 
    C_{XY} = \{ (x,y) \in \mathbb{R}^{2} : f(x,y) > 0 \} 
  \] 
  se llama soporte continuo de $(X,Y)$.
\end{defn}

\begin{obs}
  Como $f$ es función de densidad se tiene que
  \[ 
    \iint_{C_{XY}}^{} f(u,v) dudv = 1 
  \] 
\end{obs}

\begin{theo}
  Si $f : \mathbb{R}^{2} \to \mathbb{R}$ es una función no negativa, integrable Riemann en $\mathbb{R}^{2}$ y
  \[ 
    \int_{-\infty}^{+ \infty} \int_{-\infty}^{+\infty} f(x,y) dx dy = 1 
  \] 
  entonces $f$ es la función de densidad de alguna variable aleatoria bidimensional continua.
\end{theo}

\section{Distribuciones Marginales}

\begin{theo}
  Sea $(X,Y)$ v.a. discreta con función de masa $p_{XY}$ y soporte $D_{XY}$. Entonces, $X$ es una v.a. unidimensional discreta con soporte $D_{X}$, e $Y$ es una v.a. unidimensional discreta cons soporte $D_{Y}$.
\end{theo}

\begin{defn}[Funciones de Densidad Marginales]
  Sea $f$ una función de densidad en $\mathbb{R}^{2}$. Entonces, las funciones
  \[ 
    f_{1} : \mathbb{R} \to \mathbb{R} 
  \] 
  \[ 
    u \mapsto \int_{-\infty}^{+\infty} f(u,v) dv 
  \] 
  \[ 
    f_{2} : \mathbb{R} \to \mathbb{R} 
  \] 
  \[ 
    v \mapsto \int_{-\infty}^{+\infty} f(u,v) du 
  \] 
  se llaman funciones de densidad marginales.
\end{defn}

\begin{obs}
  $f_{1}$ y $f_{2}$ son funciones de densidad.
\end{obs}

\begin{defn}[Función de Distribución Marginal]
  Sea $(X_{1},X_{2})$ v.a. y $F(x_{1},x_{2})$ su función de distribución. Entonces, llamamos función des distribución marginal respecto de $X_{i}$ a
  \[ 
    F_{i} = F(x_{i}, + \infty) = \mathbb{P} \{ X_{i} \leq x_{i}, X_{j} \in \mathbb{R} \}
  \] 
  \[ 
    = \mathbb{P} \{ (- \infty, x_{i}] \times \mathbb{R} \} 
  \] 
  \[ 
    = \lim_{x_{j} \to \infty} F(x_{i}, x_{j}) 
  \] 
  donde $i, j \in \{ 1, 2 \}, i \neq j$.
\end{defn}

\begin{theo}
  Sea $(X_{1}, X_{2})$ v.a. continua con función de densidad $f$ y soporte continuo $C_{X_{1} X_{2}}$. Entonces,
  \[ 
    C_{X_{i}} = \pi_{i}(C_{X_{1}, X_{2}})
  \] 
  es el soporte de $X_{i}$ con función de densidad marginal $f_{1}$.
\end{theo}

\section{Distribuciones Condicionadas}

\begin{defn}[Función de Masa Condicionada]
  Sea $(X,Y)$ v.a. discreta con función de masa $p_{XY}$ y soporte $D_{XY}$. Entonces, la función $p(x|b) : \mathbb{R} \to \mathbb{R}$ definida por
  \[ 
    p(x|b) =
    \begin{aligned}
      \begin{cases}
        \frac{p_{XY}(x, b)}{p_{Y}(b)}, \quad (x, b) \in D_{XY} \\
        0, \quad (x, b) \not \in D_{XY}
      \end{cases}
    \end{aligned} 
  \] 
  donde $b \in D_{y}$, es la función de masa de $X$ condicionada por $Y = b$.
\end{defn}

\begin{theo}
  La v.a. $X$ condicionada por $Y=b \in D_{Y}$ es una v.a. discreta con función de masa $p(x|b)$.
\end{theo}

\begin{defn}[Función de Densidad Condicionada]
  Se llama función de densidad de $X$ condicionada por $Y = y$ a la función $f_{X|Y}(x|y)$ no negativa que satisface
  \[ 
    F_{X|Y}(x,y) = \int_{- \infty}^{x} f_{X|Y}(t|y) dt, \quad \forall x \in \mathbb{R}
  \] 
\end{defn}

\begin{obs}
  $f_{X|Y}$ es función de densidad sobre $\mathbb{R}$.
\end{obs}

\begin{theo}
  Sea $(X,Y)$ v.a. continua con función de densidad $f$. Entonces,
  \begin{enumerate}[label=(\roman*)]
    \item $\forall (x,y) \in \mathbb{R}^{2} : f$ es continua y $f_{2}(y) \geq 0$ y $f_{2} es continua$ existe la función de densidad condicional de $X$ condicionada por $Y$ 
      \[ 
        f_{X|Y}(x | y) = \frac{f(x,y)}{f_{2}(y)}
      \] 
  \end{enumerate}
\end{theo}

\section{Independencia}

\begin{defn}[Sucesos Independientes Dos a Dos]
  Sea $(\Omega, \mathcal{A}, P )$, $\{ A_{j} \}_{j \in J}$ fml. de sucesos. Decimos que son independientes dos a dos si $\forall i, j \in J$ se tiene que 
  \[ 
    \mathbb{P} \{ A_{i} \cap A_{j} \} = \mathbb{P} \{ A_{i} \} \mathbb{P} \{ A_{j} \}
  \] 
\end{defn}

\begin{defn}[Sucesos Mutualmente Independientes]
  Sea $(\Omega, \mathcal{A}, P )$, $\{ A_{j} \}_{j \in J}$ fml. de sucesos. Decimos que son mutamente independientes si $\forall I \subset J : \card(I) \geq 2$ se tiene que 
  \[ 
    \mathbb{P} \{ \bigcap_{i \in I} A_{j} \} = \prod_{i \in I} \mathbb{P} \{ A_{i} \}
  \] 
\end{defn}

\begin{prop}
  Sea $(X,Y)$ v.a. Si $X$ e $Y$ son v.a. independientes, entonces $(X,Y)$ es v.a. independiente. En particular si $X, Y$ con v.a. discretas se tiene que 
  \[ 
    p_{XY}(x,y) = p_{X}(x) p_{Y}(y) ,  \quad \forall (x,y) \in \mathbb{R}^{2}.
  \] 
  Y si $X, Y$ son v.a. continuas se tiene que
  \[ 
    f_{XY}(x,y) = f_{X}(x) f_{Y}(y), \quad \forall (x,y) \in \mathbb{R}^{2}
  \] 
\end{prop}

\begin{obs}
  Si $X,Y$ son v.a. independientes con funciones de distribución $F_{X}, F_{Y}$, entonces
  \[
    F_{XY}(x,y) = F_{X}(x) F_{Y}(y)
  \]
  es la función de distribución de $(X,Y)$.
\end{obs}

\begin{obs}
  Si $X, Y$ son v.a. independientes entonces, 
  \[ 
    f_{X|Y}(x|y) = f_{X}(x) \quad \text{ y } \quad f_{Y|X}(y|x) = f_{Y}(y)
  \] 
  son las funciones de densidad condicionadas.
\end{obs}

\section{Transformaciones}

\begin{theo}
  Sea $(X,Y)$ v.a. continua con función de densidad $f$.
  \begin{enumerate}[label=(\roman*)]
    \item Sea la transformación
      \[ 
        z = g_{1}(x,y) ,
      \] 
      \[ 
        t = g_{2}(x,y) 
      \] 
      tal que existe la transformación inversa
      \[ 
        x = h_{1}(z,t) ,
      \] 
      \[ 
        y = h_{2}(z,t) 
      \] 
    \item Ambas Transformaciones son continuas.
    \item Existen todas as derivadas parciales.
    \item El jacobino de a transformación inversa
      \[ 
        J = \frac{\partial(x,y)}{\partial(z,t)} 
        = 
        \begin{vmatrix}
           x_{z} & x_{t} \\
           y_{z} & y_{t}
        \end{vmatrix} 
      \] 
      es distinto de cero.
  \end{enumerate}
  Entonces, la v.a. $(Z, T)$ es continua y tiene función de densidad
  \[ 
    l(z,t) = f(h_{1}(z,t), h_{2}(z,t)) \cdot | J | 
  \] 
\end{theo}

\subsection{Suma Variables Aleatorias}

\begin{prop}
  Sea $(X,Y)$ v.a. con función de densidad $f$. La variable aleatoria $Z = X + Y$ tiene función de densidad
  \[ 
    f_{Z}(z) = \int_{-\infty}^{+\infty} f(t, z - t) dt
  \] 
  y función de distribución
  \[ 
    F_{Z}(z) = \int_{-\infty}^{z} \int_{-\infty}^{+\infty} f(t, z - t) dt ds
  \] 
  Si $X$ e $Y$ son independientes
  \[ 
    f_{Z}(z) = \int_{-\infty}^{+\infty} f_{1}(t)f_{2}(z - t) dt
  \] 
  \[ 
    F_{Z}(z) = \int_{-\infty}^{z} \int_{-\infty}^{+\infty} f_{1}(t)f_{2}(z - t) dt
  \] 
  donde $f_{1}$ y $f_{2}$ son las funciones de densidad marginales.
\end{prop}

\subsection{Producto Variables Aleatorias}

\begin{prop}
  Sea $(X,Y)$ v.a. con función de densidad $f$. La variable aleatoria $Z = X \cdot Y$ tiene función de densidad
  \[ 
    f_{Z}(z) = \int_{-\infty}^{+\infty} f(t, \frac{z}{t}) \cdot \frac{1}{| t |} dt
  \] 
  y función de distribución
  \[ 
    F_{Z}(z) = \int_{-\infty}^{z} \int_{-\infty}^{+\infty} f(t, \frac{z}{t}) \cdot \frac{1}{| t |} dt ds
  \] 
  Si $X$ e $Y$ son independientes
  \[ 
    f_{Z}(z) = \int_{-\infty}^{+\infty} f_{1}(t)f_{2}(\frac{z}{t}) \cdot \frac{1}{| t |} dt
  \] 
  \[ 
    F_{Z}(z) = \int_{-\infty}^{z} \int_{-\infty}^{+\infty} f_{1}(t)f_{2}(\frac{z}{t}) \cdot \frac{1}{| t |} dt ds
  \] 
  donde $f_{1}$ y $f_{2}$ son las funciones de densidad marginales.
\end{prop}

\subsection{Cociente Variables Aleatorias}

\begin{prop}
  Sea $(X,Y)$ v.a. con función de densidad $f$. La variable aleatoria $Z = X / Y$ tiene función de densidad
  \[ 
    f_{Z}(z) = \int_{-\infty}^{+\infty} f(tz, t) | t | dt
  \] 
  y función de distribución
  \[ 
    F_{Z}(z) = \int_{-\infty}^{z} \int_{-\infty}^{+\infty} f(tz, t) | t | dt ds
  \] 
  Si $X$ e $Y$ son independientes
  \[ 
    f_{Z}(z) = \int_{-\infty}^{+\infty} f_{1}(tz)f_{2}(t) | t | dt
  \] 
  \[ 
    F_{Z}(z) = \int_{-\infty}^{z} \int_{-\infty}^{+\infty} f_{1}(tz)f_{2}(t) | t | dt ds
  \] 
  donde $f_{1}$ y $f_{2}$ son las funciones de densidad marginales.
\end{prop}

\section{Esperanza}

\begin{defn}[Esperaza Variable Aleatoria Bidimensional Discreta]
  Sea $(X,Y)$ v.a. con función de masa $p_{XY}$ y soporte $D_{XY}$. Si la serie
  \[ 
    \sum_{(x,y) \in D_{XY}} | g(x,y) | p_{XY}(x,y)
  \] 
  converge, entonces
  \[ 
    \mathbb{E} [ g(X,Y) ] = \sum_{(x,y) \in D_{XY}} | g(x,y) | p_{XY}(x,y)
  \] 
  es la esperanza de $g(X,Y)$.
\end{defn}

\begin{defn}[Esperaza Variable Aleatoria Bidimensional Discreta]
  Sea $(X,Y)$ v.a. continua con función de densidad $f$. Si la integral
  \[ 
    \int_{-\infty}^{+\infty} | g(x,y) | f(x,y) dx dy 
  \] 
  es finita, entonces
  \[ 
    \mathbb{E} [ g(X,Y) ] = \int_{-\infty}^{+\infty}  g(x,y)  f(x,y) dx dy 
  \] 
  es la esperanza de la v.a. $g(X,Y)$.
\end{defn}

\section{Momentos}

\begin{defn}[Momentos Respecto al Origen]
  Sea $(X,Y)$ v.a. El momento respecto al origen de orden $(k,l)$ es
  \[ 
    \alpha_{kl} = \mathbb{E} [ X^{k} Y^{t} ] 
  \] 
\end{defn}

\begin{defn}[Momento Respecto a la Media]
  Sea $(X,Y)$ v.a. El momento respecto a la media es de ordén $(k,l)$
  \[ 
    \mu_{kl} = \mathbb{E} [ (X - \alpha_{10})^{k} (Y - \alpha_{01})^{l} ]
  \] 
\end{defn}

\begin{defn}[Coeficiente de Correlación]
  El coeficiente de correlación se define como 
  \[ 
    \rho = \frac{\mu_{11}}{\sigma_{X} \sigma_{Y}} 
  \] 
  siendo $\sigma_{X} = \sqrt{\mu_{20}}$ y $\sigma_{Y} = \sqrt{\mu_{02}}$
\end{defn}

\section{Propiedades Esperanza}

\begin{prop}
  La esperanza verifica
  \begin{enumerate}[label=(\roman*)]
    \item $\mathbb{E} [ c ] = c, \forall c \in \mathbb{R}$
    \item Si $\mathbb{E} [ | X_{i} | ] < + \infty$ entonces
      \[
        \mathbb{E} [ \sum_{i = 1}^{n} a_{i} X_{i} ] = \sum_{i = 1}^{n} a_{i} \mathbb{E} [ X_{i} ]
      \]
    \item Si $\mathbb{E} [ | X_{i} | ] < + \infty$ y $\mathbb{E} [ | X_{i} |^{2} ] < + \infty$ entonces 
      \[ 
        \mathbb{E} [ X_{1} \cdot X_{2} ] = \mu_{11} + \alpha_{01} \cdot \alpha_{10}
      \] 
    \item Si $\mathbb{E} [ | X_{i} | ] < + \infty$ entonces
      \[ 
        \mathbb{E} [ \prod X_{i} ] = \prod \mathbb{E} [ X_{i} ]
      \] 
    \item $\mathbb{E} [ g_{1}(X) ] \leq \mathbb{E} [ g_{2}(X) ]$
    \item $| \mathbb{E} [ g(X) ] | \leq \mathbb{E} [ | g(X) | ]$
    \item $\mathbb{E} [ g_{1}(X) = g_{2}(X) ] = \mathbb{E} [ g_{1}(X) ] + \mathbb{E} [ g_{2}(X) ]$
  \end{enumerate}
\end{prop}

\section{Propiedades Varianza}

\begin{prop}
  La varianza verifica
  \begin{enumerate}[label=(\roman*)]
    \item 
    \item 
    \item 
    \item 
    \item 
    \item 
  \end{enumerate}
\end{prop}

\section{Función Característica}

\section{Coefieciente Correlación}

\section{Regresión}
