\chapter{Probabilida sobre la recta real}

\section{Probabilidad Sobre La Recta Real}

\begin{nota}
  Consideramos $(\mathbb{R}, B(\mathbb{R}), P)$ espacio de probabilidad.
\end{nota}

\subsection{Función de distribución en $\mathbb{R}$}

\begin{defn}[Función de distribución]
  Sea $F: \mathbb{R} \to \mathbb{R}$ tal que
  \begin{enumerate}[label=(\roman*)]
    \item $F$ es monótona no decreciente, es decir, $ \forall x_{1}, x_{2} \in \mathbb{R}: x_{1} < x_{2} \Rightarrow F(x_{1}) < F(x_{2})$.
    \item $F$ es continua por la derecha, $\lim_{h \to 0} F(x + h) = F(x)$
    \item $\lim_{x \to -\infty} F(x) = 0$,
    \item $\lim_{x \to +\infty} F(x) = 1$.
  \end{enumerate}
\end{defn}

\begin{theo}
  La función $F(x) = P \{ (- \infty, x] \}$ es función de distribución en $\mathbb{R}$.
\end{theo}

\begin{theo}
  Sea $F$ función de distribución en $\mathbb{R}$. Entonces, $F$ induce en $(\mathbb{R}, B(\mathbb{R}))$, espacio probabilizable, una probabilidad $P$ cuya función de distribución es $F$.
\end{theo}

\section{Probabilidad sobre $\mathbb{R}^n$}

\begin{defn}[Función de Distribución]
  Una función $F: \mathbb{R}^{n} \to \mathbb{R}$ se dice que es de distribución en $\mathbb{R}^{n}$ si y solo si
  \begin{enumerate}[label=(\roman*)]
    \item $\forall a, b \in \mathbb{R}^{n} :  a \leq b \Rightarrow F((a, b]) \geq 0$.
    \item $F$ continua por la derecha en cada variable, es decir, si $\{ x^{k} \}_{n \in \mathbb{N}} \downarrow : \{ x^{n} \}_{n \in \mathbb{N}} \rightarrow x$ con $x^{k} \in \mathbb{R}^{n}, \forall k \in \mathbb{N}$, entonces
    \[ 
      \lim_{n \to \infty} F(x^{n}) = F(x)
    \] 
    \item $\lim_{x_{i} \to -\infty} F(x_{1}, \cdots, x_{n}) = 0, \forall i \in \{ 1, \cdots, n \}$ y $F(+ \infty, \cdots, + \infty) = \lim_{x_{1}, \cdots, x_{n} \to + \infty} F(x_{1}, \cdots, x_{n}) = 1$.
  \end{enumerate}
\end{defn}
