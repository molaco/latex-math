\begin{ejr}[38]
  En la circunferencia $\mathbb{S}^{1} = \{ (x, y) \in \mathbb{R}^{2} : x^{2} + y^{2} = 1 \}$, con la topología usual restringida, se identifican los puntos diametralmente opuestos. Probar que el espacio cociente resultante es homeomorfo al obtenido a partir del intervalo $[0, 1]$ identificando los extremos.
\end{ejr}

\begin{sol}
%  La recta real proyectiva son el conjunto de rectas que pasan por el origen en $\mathbb{R}^{2}$. 
%
%  Sea $(\mathbb{R}^{2}, \mathcal{T}_{u})$, $\mathcal{R} : x \mathcal{R} y \Leftrightarrow x = \lambda y, \lambda \in \mathbb{R}$ relación de equivalencia en $\mathbb{R}^{2} \setminus \{ 0, 0 \}$ y $q : \mathbb{R}^{2} \setminus \{ 0, 0 \} \to \mathbb{R} \mathbb{P}_{1} : (x, y) \mapsto [(x, y)]_{\mathcal{R}}$. Entonces, 
https://math.stackexchange.com/questions/311196/homeomorphism-between-the-real-projective-line-and-a-circle


  El obtenido a partir del intervalo $[0, 1]$ identificando los extremos es homeomorfo a $\mathbb{S}^{1}$.
\end{sol}
