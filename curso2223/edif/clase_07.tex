\part{Estabilidad y Sistemas Autónomos}

\chapter{Estabilidad de Sistemas lineales}

\section{Sistemas lineales Homogéneos con coeficientes constantes}

\begin{defn}[Sistema Autónomo]
  content
\end{defn}

\begin{defn}[Punto de equilibrio]
  content
\end{defn}

\begin{defn}[Punto de equilibrio Hiperbólico]
  content
\end{defn}

\begin{defn}[Punto de equilibrio Atractor]
  content
\end{defn}

\begin{defn}[Punto de equilibrio Fuente]
  content
\end{defn}

\begin{obs}
  Si el origen es punto atractor o fuente, entonces es hiperbólico.
\end{obs}

\begin{defn}[Sistemas Topológicamente equivalentes]
  content
\end{defn}

\begin{defn}[Punto de Silla]
  content
\end{defn}

\begin{prop}[Caracterización de puntos de equilibrio hiperbólicos]
  content
\end{prop}

\begin{prop}[Soluciones Acotadas y Soluciones Periódicas]
  content
\end{prop}

\begin{defn}[Estabilidad de Soluciones]
  content
\end{defn}

\begin{defn}[Variedad Lineal Estable Local]
  Sea el sistema autónomo lineal $y'(t) = A(t) y$. Se conoce como variedad lineal estable local $(E_{s})$, variedad lineal inestable local $(E_{s})$ y variedad lineal central $(E_{c})$ a
  \[ 
    E_{s} = \langle \lambda_{1}, \cdots, \lambda_{m} \rangle
  \] 
  \[ 
    E_{u} = \langle \lambda_{m}, \cdots, \lambda_{m + n} \rangle
  \] 
  \[ 
    E_{c} = \langle \lambda_{m + n}, \cdots, \lambda_{m + n + k} \rangle
  \] 
  donde $\lambda_{i} \in \rho(A)$ tal que
  \[ 
    \begin{aligned}
      \begin{cases}
        \Re(\lambda_{i}) < 0, \quad 1 \leq i \leq m \\
        \Re(\lambda_{i}) > 0, \quad m \leq i \leq n + m \\
        \Re(\lambda_{i}) = 0, \quad m + n \leq i \leq m + n + k \\
      \end{cases}
    \end{aligned} 
  \] 
\end{defn}

\begin{defn}[Variedad Estable Global]
  Dado us sistema autónomo $y' = f(x) \in \mathbb{C}(\Omega)$ y $x_{\infty}$ punto de equilibrio. La variedad estable global de $y'$ es 
  \[ 
    W_{s}(x_{\infty})  = \{ x_{0} \in \Omega : \lim_{t \to +\infty} x(t; 0, x_{0}) = x_{\infty} \}
  \] 
\end{defn}

\begin{defn}[Variedad Inestable Global]
  Dado us sistema autónomo $y' = f(x) \in \mathbb{C}(\Omega)$ y $x_{\infty}$ punto de equilibrio. La variedad inestable global de $y'$ es 
  \[ 
    W_{s}(x_{\infty})  = \{ x_{0} \in \Omega : \lim_{t \to -\infty} x(t; 0, x_{0}) = x_{\infty} \}
  \] 
\end{defn}

\section{Sistemas Lineales Homogéneos con Coeficientes Variables}

\begin{obs}[Sistema Considerado]
  content
\end{obs}

\begin{obs}[Forma Integral De Las Soluciones De Un Sistema Lineal Homogéneo Con Coeficientes Variables]
  content
\end{obs}

\begin{prop}[Caracterización de Soluciones 1]
  content
\end{prop}

\begin{prop}[Caracterización de Soluciones 2]
  content
\end{prop}

\section{Sistemas Lineales No Homogéneos}

\begin{obs}[Sistema Considerado]
  content
\end{obs}

\begin{defn}[Sistema Lineal Asociado]
  content
\end{defn}

\begin{prop}[Caracterización de las Soluciones 1]
  content
\end{prop}

\begin{prop}[Caracterización de las Soluciones 2]
  content
\end{prop}

\section{Diagramas de Fases de Sistemas Planos}

Esquema EDO.

\chapter{Estabilidad de Sistemas no Lineales}

\section{Comportamiento Cualitativo De las Soluciones}

\begin{defn}
  Dado el problema del valor inicial
  \[ 
    \begin{aligned}
      \begin{cases}
        y' = f(t,y) \\
        y(t_{0}) = y_{0}
      \end{cases}
    \end{aligned} 
  \] 
  donde $f \in \mathcal{C}([ a, b ] \times \Omega ; \mathbb{R}^{n})$. Consideramos el sistema no lineal
  \[ 
    y' = f(t,y) 
  \] 
  Entonces, decimos que
  \begin{enumerate}[label=(\roman*)]
    \item $x \in \Omega$ es un punto de equilibrio de $y' = f(y)$ si $f(x) = 0$.
    \item Un punto de equilibrio es hiperbólico si
      \[ 
        \forall \lambda \in \rho(Df(x)), \Re (\lambda) \neq 0.
      \] 
    \item Un punto de equilibrio se denomina no hiperbólico si
      \[ 
        \exists \lambda \in \rho(Df(x)): \Re (\lambda) = 0.
      \] 
    \item El sistema $y' = Df(x) \cdot y$ es el sistema lineal asociado.
  \end{enumerate}
\end{defn}

\begin{defn}[Clasificación De Puntos De Equilibrio]
  Sea $x$ un punto de equilibrio hiperbólico del sistema no lineal. Considerando el sistema lineal asociado $Df(x)$, entonces este se clasifica de la misma forma que los puntos de equilibrio de un sistema lineal.
\end{defn}

\begin{obs}
  Un punto de equilibrio $x$ foco es asintóticamente estable ($\forall \lambda \in \rho(Df(x)), \Re(\lambda) < 0 $).
\end{obs}

\begin{obs}
  Un punto de equilibrio fuente o de silla es inestable.
\end{obs}

\section{Teorema de la Variedad Estable}

\begin{defn}[Variedad No Lineal Estable]
  Sea $y' = f(y)$ un sistema no lineal, entonces la variedades estables locales de $y'$ son las del sistema lineal asociado $y' = Df(x) y$.
\end{defn}

\begin{theo}[de la Variedad Estable]
  content
\end{theo}

\begin{ejm}[Cálculo de Variedades]
  Calcular las variedades estables e inestables de un sistema no lineal.
\end{ejm}

\section{Teorema de Hartman-Grobman}

\begin{obs}
  Bajo que condiciones los puntos de equilibrio de un sistema no lineal tienen el mismo comportamiento cualitativo que el sistema lineal asociado.
\end{obs}

\begin{theo}
  Es condición suficiente que $Df(x_{\infty})$ no tenga autovalores con parte real nula, es decir, que sea hiperbólico.
\end{theo}

\section{Teorema de Lyapunov}

\begin{obs}
  Consideramos la estabilidad de los puntos de equilibrio de un sistema no lineal
  \[ 
    y' = f(t,y), 
  \] 
  Si el punto de equilibrio es hiperbólico de terminamos la estabilidad según el signo de los autovalores del sistema lineal asociado $y' = Df(t) \cdot y$. Ahora, si el punto no es hiperbólico usamos el método de Lyapunov.
\end{obs}

\begin{theo}[de Lyapunov] 
  Sea $\dot{u} = f(u)$ us sistema autónomo no lineal $u_{\infty} \in \mathbb{R}^{\infty}$. Si existe $V : \mathbb{R}^{d} \to \mathbb{R}$ 
  \begin{enumerate}[label=(\roman*)]
    \item $V(x) = 0 \Leftrightarrow x = u_{\infty}$, 
    \item $V(x) > 0, \quad \forall x \neq u_{\infty}$,
  \end{enumerate}
  Entonces,
  \begin{itemize}
    \item $\nabla V(x) \cdot f(x) \leq 0, \quad \forall x \in \mathbb{R}^{d} \Rightarrow u_{\infty}$ es estable,
    \item $\nabla V(x) \cdot f(x) < 0, \quad \forall x \in \mathbb{R}^{d} \Rightarrow u_{\infty}$ es asintóticamente estable,
    \item $\nabla V(x) \cdot f(x) > 0, \quad \forall x \in \mathbb{R}^{d} \Rightarrow u_{\infty}$ es inestable.
  \end{itemize}
  En caso de que exista, la función $V(x)$ se llama función de Lyapunov.
\end{theo}
