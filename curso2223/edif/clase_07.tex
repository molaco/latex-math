\part{Estabilidad y Sistemas Autónomos}

\chapter{Estabilidad de Sistemas lineales}

\section{Sistemas lineales Homogéneos con coeficientes constantes}

\begin{defn}[Sistema Autónomo]
  content
\end{defn}

\begin{defn}[Punto de equilibrio]
  content
\end{defn}

\begin{defn}[Punto de equilibrio Hiperbólico]
  content
\end{defn}

\begin{defn}[Punto de equilibrio Atractor]
  content
\end{defn}

\begin{defn}[Punto de equilibrio Fuente]
  content
\end{defn}

\begin{obs}
  Si el origen es punto atractor o fuente, entonces es hiperbólico.
\end{obs}

\begin{defn}[Sistemas Topológicamente equivalentes]
  content
\end{defn}

\begin{defn}[Punto de Silla]
  content
\end{defn}

\begin{prop}[Caracterización de puntos de equilibrio hiperbólicos]
  content
\end{prop}

\begin{prop}[Soluciones Acotadas y Soluciones Periódicas]
  content
\end{prop}

\begin{defn}[Estabilidad de Soluciones]
  content
\end{defn}

\begin{defn}[Variedad Lineal Estable Local]
  Sea el sistema autónomo lineal $y'(t) = A(t) y$. Se conoce como variedad lineal estable local $(E_{s})$, variedad lineal inestable local $(E_{s})$ y variedad lineal central $(E_{c})$ a
  \[ 
    E_{s} = \langle \lambda_{1}, \cdots, \lambda_{m} \rangle
  \] 
  \[ 
    E_{u} = \langle \lambda_{m}, \cdots, \lambda_{m + n} \rangle
  \] 
  \[ 
    E_{c} = \langle \lambda_{m + n}, \cdots, \lambda_{m + n + k} \rangle
  \] 
  donde $\lambda_{i} \in \rho(A)$ tal que
  \[ 
    \begin{aligned}
      \begin{cases}
        \Re(\lambda_{i}) < 0, \quad 1 \leq i \leq m \\
        \Re(\lambda_{i}) > 0, \quad m \leq i \leq n + m \\
        \Re(\lambda_{i}) = 0, \quad m + n \leq i \leq m + n + k \\
      \end{cases}
    \end{aligned} 
  \] 
\end{defn}

\begin{defn}[Variedad Estable Global]
  Dado us sistema autónomo $y' = f(x) \in \mathbb{C}(\Omega)$ y $x_{\infty}$ punto de equilibrio. La variedad estable global de $y'$ es 
  \[ 
    W_{s}(x_{\infty})  = \{ x_{0} \in \Omega : \lim_{t \to +\infty} x(t; 0, x_{0}) = x_{\infty} \}
  \] 
\end{defn}

\begin{defn}[Variedad Inestable Global]
  Dado us sistema autónomo $y' = f(x) \in \mathbb{C}(\Omega)$ y $x_{\infty}$ punto de equilibrio. La variedad inestable global de $y'$ es 
  \[ 
    W_{s}(x_{\infty})  = \{ x_{0} \in \Omega : \lim_{t \to -\infty} x(t; 0, x_{0}) = x_{\infty} \}
  \] 
\end{defn}

\section{Sistemas Lineales Homogéneos con Coeficientes Variables}

\begin{obs}[Sistema Considerado]
  content
\end{obs}

\begin{obs}[Forma Integral De Las Soluciones De Un Sistema Lineal Homogéneo Con Coeficientes Variables]
  content
\end{obs}

\begin{prop}[Caracterización de Soluciones 1]
  content
\end{prop}

\begin{prop}[Caracterización de Soluciones 2]
  content
\end{prop}

\section{Sistemas Lineales No Homogéneos}

\begin{obs}[Sistema Considerado]
  content
\end{obs}

\begin{defn}[Sistema Lineal Asociado]
  content
\end{defn}

\begin{prop}[Caracterización de las Soluciones 1]
  content
\end{prop}

\begin{prop}[Caracterización de las Soluciones 2]
  content
\end{prop}

\section{Diagramas de Fases de Sistemas Planos}

Esquema EDO.

\chapter{Estabilidad de Sistemas no Lineales}

\section{Comportamiento Cualitativo De las Soluciones}

\begin{defn}
  Dado el problema del valor inicial
  \[ 
    \begin{aligned}
      \begin{cases}
        y' = f(t,y) \\
        y(t_{0}) = y_{0}
      \end{cases}
    \end{aligned} 
  \] 
  donde $f \in \mathcal{C}([ a, b ] \times \Omega ; \mathbb{R}^{n})$. Consideramos el sistema no lineal
  \[ 
    y' = f(t,y) 
  \] 
  Entonces, decimos que
  \begin{enumerate}[label=(\roman*)]
    \item $x \in \Omega$ es un punto de equilibrio de $y' = f(y)$ si $f(x) = 0$.
    \item Un punto de equilibrio es hiperbólico si
      \[ 
        \forall \lambda \in \rho(Df(x)), \Re (\lambda) \neq 0.
      \] 
    \item Un punto de equilibrio se denomina no hiperbólico si
      \[ 
        \exists \lambda \in \rho(Df(x)): \Re (\lambda) = 0.
      \] 
    \item El sistema $y' = Df(x) \cdot y$ es el sistema lineal asociado.
  \end{enumerate}
\end{defn}

\begin{defn}[Clasificación De Puntos De Equilibrio]
  Sea $x$ un punto de equilibrio hiperbólico del sistema no lineal. Considerando el sistema lineal asociado $Df(x)$, entonces este se clasifica de la misma forma que los puntos de equilibrio de un sistema lineal.
\end{defn}

\begin{obs}
  Un punto de equilibrio $x$ foco es asintóticamente estable ($\forall \lambda \in \rho(Df(x)), \Re(\lambda) < 0 $).
\end{obs}

\begin{obs}
  Un punto de equilibrio fuente o de silla es inestable.
\end{obs}

\section{Teorema de la Variedad Estable}

\begin{defn}[Variedad No Lineal Estable]
  Sea $y' = f(y)$ un sistema no lineal, entonces la variedades estables locales de $y'$ son las del sistema lineal asociado $y' = Df(x) y$.
\end{defn}

\begin{theo}[de la Variedad Estable]
  content
\end{theo}

\begin{ejm}[Cálculo de Variedades]
  Calcular las variedades estables e inestables de un sistema no lineal.
\end{ejm}

\section{Teorema de Hartman-Grobman}

\begin{obs}
  Bajo que condiciones los puntos de equilibrio de un sistema no lineal tienen el mismo comportamiento cualitativo que el sistema lineal asociado.
\end{obs}

\begin{theo}
  Es condición suficiente que $Df(x_{\infty})$ no tenga autovalores con parte real nula, es decir, que sea hiperbólico.
\end{theo}

\section{Teorema de Lyapunov}

\begin{obs}
  Consideramos la estabilidad de los puntos de equilibrio de un sistema no lineal
  \[ 
    y' = f(t,y), 
  \] 
  Si el punto de equilibrio es hiperbólico determinamos la estabilidad según el signo de los autovalores del sistema lineal asociado $y' = Df(t) \cdot y$. Ahora, si el punto no es hiperbólico usamos el método de Lyapunov.
\end{obs}

\begin{theo}[de Lyapunov] 
  Sea $\dot{u} = f(u)$ us sistema autónomo no lineal $u_{\infty} \in \mathbb{R}^{\infty}$. Si existe $V : \mathbb{R}^{d} \to \mathbb{R}$ 
  \begin{enumerate}[label=(\roman*)]
    \item $V(x) = 0 \Leftrightarrow x = u_{\infty}$, 
    \item $V(x) > 0, \quad \forall x \neq u_{\infty}$,
  \end{enumerate}
  Entonces,
  \begin{itemize}
    \item $\nabla V(x) \cdot f(x) \leq 0, \quad \forall x \in \mathbb{R}^{d} \Rightarrow u_{\infty}$ es estable,
    \item $\nabla V(x) \cdot f(x) < 0, \quad \forall x \in \mathbb{R}^{d} \Rightarrow u_{\infty}$ es asintóticamente estable,
    \item $\nabla V(x) \cdot f(x) > 0, \quad \forall x \in \mathbb{R}^{d} \Rightarrow u_{\infty}$ es inestable.
  \end{itemize}
  En caso de que exista, la función $V(x)$ se llama función de Lyapunov.
\end{theo}

\section{Teorema de Poincaré-Bendixson}

\begin{defn}[Órbita]
  Sea $y' = f(y)$ un sistema autónomo. Si $y$ es una solución en su intervalo máximo de existencia $(\alpha, \omega)$, entonces
  \[ 
    \{ y(t) : \alpha < t < \omega \}
  \] 
  se dice que es una órbita.
\end{defn}

\begin{obs}
  Si $\phi$ es solución de $y'$ entonces, $\{ \phi(t) : t \in (\alpha, \omega) \}$ es una órbita.
\end{obs}

\begin{prop}
  Dos órbitas son disjuntas o son iguales.
\end{prop}

\begin{prop}
  Si dos órbitas tienen un punto en común, entonces son idénticas.
\end{prop}

\begin{nota}
  Se denota $\phi(t, y_{0})$ a la solución única del sistema autónomo $y' = f(y)$ tal que $\phi(t_{0}, y_{0}) = y_{0}$. Es decir, $\phi(t, y_{0})$ es la solución asociada al valor inicial $y_{0}$.
\end{nota}

\begin{defn}[Positivamente Invariante]
  Un conjunto $S$ se dice que es positivamente invariante para el sistema $y' = f(y)$ si 
  \[ 
    \forall y_{0} \in S, \phi(t, y_{0}) \in S, \quad \forall t \in [0, \omega).
  \] 
\end{defn}

\begin{nota}
  Toda solución $\phi(t,y_{0})$ del sistema autónomo $y' = f(y)$ se corresponde con una órbita en el espacio de fases que denotamos $\gamma(y_{0})$, es decir,
  \[ 
    \gamma(y_{0}) = \{ \phi(t, x_{0}) : t \in (\alpha, \omega) \} 
  \] 
\end{nota}

\begin{defn}[$\omega$-límite]
  El conjunto de los puntos límite de una órbita $\phi(t, y_{0})$ en el intervalo $[0, +\infty)$ es
  \[ 
    \omega(\gamma(y_{0})) = \{ z \in \mathbb{R}^{n} : \exists ( t_{\lambda} )_{\lambda \in \Lambda} : t_{\lambda} \rightarrow +\infty \Rightarrow \phi(t_{\lambda}, y_{0}) \rightarrow z \}.
  \] 
\end{defn}

\begin{obs}
  Es el conjunto de puntos a los que tiende la solución $\phi(t,y_{0})$ cuando $t \rightarrow +\infty$.
\end{obs}

\begin{defn}[$\alpha$-límite]
  El conjunto de los puntos límite de una órbita $\phi(t, y_{0})$ en el intervalo $(-\infty, 0]$ es
  \[ 
    \omega(\gamma(y_{0})) = \{ z \in \mathbb{R}^{n} : \exists ( t_{\lambda} )_{\lambda \in \Lambda} : t_{\lambda} \rightarrow -\infty \Rightarrow \phi(t_{\lambda}, y_{0}) \rightarrow z \}.
  \] 
\end{defn}
\begin{obs}
  Es el conjunto de puntos a los que tiende la solución $\phi(t,y_{0})$ cuando $t \rightarrow -\infty$.
\end{obs}

\begin{prop}[Solución Periódica]
  Sea $y' = f(y)$ un sistema autónomo, $y(t)$ una solución tal que $y(0) = y(\omega)$ es una solución periódica.
\end{prop}

\begin{obs}
  Una solución $y(t)$ que satisface $y(0) = y(\omega)$ es una curva cerrada.
\end{obs}

\begin{defn}[Cíclo]
  Un cíclo es una solución periódica no constante.
\end{defn}

\begin{defn}[Ciclo-límite]
  Si un ciclo es el $\omega$-límite o $\alpha$-límite de una órbita distinta, entonces se llama ciclo-límite.
\end{defn}

\begin{defn}[Ciclo-límite estable]
  Si un cíclo es el $\omega$-límte de toda órbita cercana, entonces se llama ciclo-límite estable.
\end{defn}

\begin{theo}[Poincaré-Bendixson]
  Sea $y' = f(y)$ un sistema autónomo. Si $\phi(t,y)$ es una órbita acotada para $t \geq 0$ entonces, se cumple una de las siguientes
  \begin{itemize}
    \item $\omega(\gamma(y))$ es un ciclo,
    \item $\forall z \in \omega(\gamma(y))$, $\omega(\gamma(z))$ es un conjunto de uno o más puntos de equilibrio.   
  \end{itemize}
  El mismo resultado se cumple para órbitas negativas.
\end{theo}

\begin{prop}
  Sea $y' = f(y)$ un sistema autónomo, $\phi(t, y_{0})$ ciclo. Entonces, existe almenos un punto de equilibrio dentro del ciclo.
\end{prop}

\begin{obs}
  Si $D \subset \mathbb{R}^{2}$ tal que $\partial{D} = \{ \phi(t,y_{0}) : t \in [0, \omega) \}$, entonces $\exists y \in D : f(y) = 0$ es punto de equilibrio.
\end{obs}

\begin{defn}[Simplemente Conexo]
  Un conjunto $D \subset \mathbb{R}^{2}$ es simplemente conexo si es conexo y $\forall C$ curva cerrada tal que $C \subset \mathring{D}$ entonces $\mathring{C} \subset D$.
\end{defn}

\begin{defn}[Divergencia]
  Sea $F(x,y) = (f(x,y), g(x,y))$ la divergencia de $F$ es 
  \[ 
    \dv F(x,y) = f_{x}(x, y) + g_{y}(x,y)
  \] 
\end{defn}

\begin{theo}[Bendixon-Dulac]
  Sea $y' = f(y)$ un sistema autónomo tal que
  \[ 
    y' =
    \begin{aligned}
      \begin{cases}
        y_{1}' = f_{1}(y_{1}, y_{2}) \\
        y_{2}' = f_{2}(y_{1}, y_{2}) \\
      \end{cases}
    \end{aligned} 
  \] 
  donde $F(y_{1}, y_{2}) = (f_{1}(y_{1}, y_{2}), f_{2}(y_{1}, y_{2}))$. Suponemos $\exists g \in \mathcal{C}^{1}(D)$, con $D \subset \mathbb{R}^{2}$ simplemente conexo tal que
  \[ 
    \dv (g(y_{1},y_{2}) \cdot F(y_{1}, y_{2}))
  \] 
  es distinto de cero y tiene signo constante en $D$. Entonces, el sistema no tiene ningún cilco en $D$.
\end{theo}

\begin{obs}
  El criterio negativo de Bendixon se obtiene cuando $g \equiv 1$.
\end{obs}
