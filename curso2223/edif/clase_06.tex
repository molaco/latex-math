\section{Definiciones}

\begin{defn}[Sistema Autónomo]
  Un sistema de ecuaciones diferenciales se dice autónomo si no depende explícitamente de la variable temporal $t$. Un sistema autónomo es de la forma
  \[ 
     =
    \begin{aligned}
      \begin{cases}
        x_{1}' = a_{11} x_{1} + \cdots + a_{1n} x_{n} \\
        x_{2}' = a_{21} x_{1} + \cdots + a_{2n} x_{n} \\
        \vdots \\
        x_{n}' = a_{n1} x_{1} + \cdots + a_{nn} x_{n} \\
      \end{cases}
    \end{aligned} 
  \] 
  donde $a_{ij} \in \mathbb{R}$.
\end{defn}

\begin{defn}[Punto de Equilibrio]
  Sea el sistema 
  \[ 
    y'(t) = A y(t).
  \] 
  un sistema lineal autónomo. Se dice que $x \in \mathbb{R}^{n}$ es un punto de equilibrio del sistema si $A x = 0$.
\end{defn}

\begin{defn}[Punto de Equilibrio Hiperbólico]
  Un punto de equilibrio es un punto de equilibrio hiperbólico si 
  \[ 
    \forall \lambda \in \rho(A), \quad \Ree(\lambda) \neq 0
  \] 
\end{defn}

\begin{defn}[Estabilidad]
  Sea $\overline{u}$ un punto de equilibrio de $\dot{u} = f(u)$. Se dice que $\overline{u}$ es estable si
  \[
    \forall \epsilon > 0, \exists \delta(\epsilon) > 0, \delta(\epsilon) \leq \epsilon, \forall u_{0} : ||\overline{u} - u_{0} || \leq \lambda
  \]
  entonces, para $\varphi(t, u_{0})$ solución se verifica
  \[
    ||\varphi(t, u_{0}) - \overline{u}|| \leq \epsilon, \quad \forall t \geq 0.
  \]
\end{defn}

\begin{defn}[Estabilidad Asintótica]
  Un punto de equilibrio estable es asintóticamente estable si
  \[
    \exists \lambda : \lim_{t \to \infty} \varphi(t, u_{0}) = \overline{u}.
  \]
\end{defn}

\section{Estabilidad Sistemas Lineales}

\begin{note}
  Consideramos el sistema lineal
  \[ 
    \dot{u}(t) = A(t) u(t) + b(t) \\
  \] 
  donde $A \in C(\mathbb{R}; \mathbb{R}^{d \times d})$, $b \in C(\mathbb{R}; \mathbb{R}^{d \times d})$ y $u \in \mathbb{R}^{n}$.
\end{note}
