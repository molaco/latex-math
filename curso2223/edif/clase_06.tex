\section{Definiciones}

\begin{defn}[Sistema Autónomo]
  Un sistema de ecuaciones diferenciales se dice autónomo si no depende explícitamente de la variable temporal $t$. Un sistema autónomo es de la forma
  \[ 
     =
    \begin{aligned}
      \begin{cases}
        x_{1}' = a_{11} x_{1} + \cdots + a_{1n} x_{n} \\
        x_{2}' = a_{21} x_{1} + \cdots + a_{2n} x_{n} \\
        \vdots \\
        x_{n}' = a_{n1} x_{1} + \cdots + a_{nn} x_{n} \\
      \end{cases}
    \end{aligned} 
  \] 
  donde $a_{ij} \in \mathbb{R}$.
\end{defn}

\begin{defn}[Punto de Equilibrio]
  Sea $f \in C^{1}(\Omega)$. Se dice que $x_{0} \in \Omega$ es un punto de equilibrio del sistema $x' = f(x)$ si $f(x_{0}) = 0$. Si $x$ no es un punto de equilibrio, entonces decimos que $x$ es un punto regular.
\end{defn}

\begin{defn}[Punto de Equilibrio Hiperbólico]
  Un punto de equilibrio es un punto de equilibrio hiperbólico si 
  \[ 
    \forall \lambda \in \rho(A), \quad \Ree(\lambda) \neq 0
  \] 
\end{defn}

%\begin{defn}[Órbita]
%  La órbita positiva de un punto $x$ se define como
%  \[ 
%    O_{x}^{+} = \{ \varphi(t, x) : t \geq 0 \} 
%  \] 
%  y la órbita negativa como 
%  \[ 
%    O_{x}^{-} = \{ \varphi(t, x) : t \leq 0 \} 
%  \] 
%\end{defn}
%
%\begin{obs}
%  Si $O_{x} \cap O_{y} \neq \emptyset$, entonces $O_{x} = O_{y}$.
%\end{obs}
%
%\begin{defn}[Diagrama de Fases]
%  Llamamos diagrma de fases de $x' = f(x)$ a la partición de $\Omega$ en órbitas.
%\end{defn}
%
%\begin{prop}[Órbitas Periódicas]
%  Si una curva integral $t \mapsto \varphi(t,x)$ se corta a sí misma entonces es periódica.
%\end{prop}

\begin{defn}[Estabilidad]
  Sea $\overline{u}(t)$ un solución de $u' = f(t, x)$ definida en el intervalo $[\beta, \infty )$. Se dice que $\overline{u}$ es estable si, 
  \[ 
    \forall t_{0} \geq 0, \epsilon \geq 0, \exists \delta = \delta(\epsilon, t_{0}) > 0 : \forall u_{0}, | u_{0} - \overline{u}(t_{0}) | \leq \lambda 
  \] 
  \[ 
    \Rightarrow | u(t; t_{0}, u_{0}) - \overline{u}(t)| \leq \epsilon, \quad \forall t \geq t_{0}. 
  \] 
  Se dice que $\overline{u}(t)$ es asintóticamente estable si es estable y si, además 
  \[ 
    \exists \eta = \eta(t_{0}) > 0 : | u_{0} - \overline{u}(t_{0}) | < \eta \Rightarrow \lim_{t \to \infty} | u(t; t_{0}, u_{0}) - \overline{u}(t) | = 0.
  \] 
\end{defn}

\begin{defn}[Punto Equilibrio Atractor]
  content
\end{defn}

\begin{defn}[Punto Equilibrio Repulsor]
   content
 \end{defn}

\section{Estabilidad Sistemas Lineales}

\begin{note}
  Consideramos el sistema lineal
  \[ 
    \dot{u}(t) = A(t) u(t) + b(t) \\
  \] 
  donde $A \in C(\mathbb{R}; \mathbb{R}^{d \times d})$, $b \in C(\mathbb{R}; \mathbb{R}^{d \times d})$ y $u \in \mathbb{R}^{n}$. \\
\end{note}

\begin{note}[Estudio Estabilidad]
  Estudiar la estabilidad de la solución $u(t; t_0, u_0)$ del sistema $\dot{u} = Au + b$ equivale a estimar el creimiento de
  \[ 
    | u(t; t_0, u_0) - u(t; t_0, \overline{u}_{0}) |
  \] 
  donde $ | u_{0} - \overline{u}_{0} | < \epsilon$.
\end{note}

\begin{note}[Sistema Homogeneo Asociado]
  Sea el sistema lineal $\dot{u} = Au + b$, $u(t; t_0, u_0)$, $u(t; t_0, \overline{u_{0}})$ soluciones. Entonces,
  \[ 
    v(t) = u(t; t_0, u_0) - u(t; t_0, \overline{u_{0}})
  \] 
  es solución de $\dot{v(t)} = A(t) u $ sistema homogeneo.
\end{note}

\begin{prop}
   Sea $\dot{u} = Au + b$ sistema lineal entonces, las propiedades de estabilidad de la solución $u(t; t_0, u_0)$ conincide con las propiedade de estabilidad del punto de equilibrio $v(t)$ solución del sistema lineal homogeneo asociado $\dot{v} = Av$.
\end{prop}

\begin{theo}
  Sea $\dot{u} = Au + b$ un sistema lineal. Entonces, todas las soluciones tienen las mismas propiedades de estabilidad.
\end{theo}

\begin{obs}
  En general, la estabilidad y la acotación son dos conceptos que no están relacionados.
\end{obs}

\begin{ejm}
  Sea el sistema $\dot{u}(t) = 1$ y solución $u(t) = t + t_{0}$. Entonces, esta es estable pero no acotada.
\end{ejm}

\begin{theo}
  Sea el sistema lineal $\dot{{u}}(t) = Au(t) + b(t)$. Entonces,
  \begin{itemize}
    \item $\dot{{u}}(t) = Au(t) + b(t)$ es estable $\Leftrightarrow$ sus soluciones son acotadas, es decir, para toda solución $u(t; t_0, u_0)$
      \[ 
        \exists M > 0 : | u(t; t_0, u_0) | < M, 
      \] 
    \item $\dot{{u}}(t) = Au(t) + b(t)$ es asintóticamente estable $\Leftrightarrow$ todas sus soluciones convergen a cero, es decir, $\forall u(t; t_0, u_0)$ solución,
      \[ 
        \lim_{t \to \infty} u(t; t_0, u_0) = 0 .
      \] 
  \end{itemize}
\end{theo}

\begin{prop}
  El sistema $\dot{{u}}(t) = Au(t) + b(t)$ es asintóticamente estable $\Leftrightarrow$ para toda solución $u(t; t_0, u_0)$,
  \[ 
    \lim_{t \to +\infty} u(t; t_0, u_0) = 0.
  \] 
\end{prop}

%\begin{dem}
%  \begin{enumerate}[label=(\roman*)]
%    \item []
%    \item [$(\Rightarrow)$] Suponemos que $\forall u(t; t_0, u_0)$ solución del sistema $\dot{{u}} = Au + b$ se tiene que
%      \[ 
%        u(t; t_0, u_0) \xrightarrow[]{ t \rightarrow + \infty } 0,
%      \] 
%      entonces la soluciones son acotadas. Por tanto, el sistema es estable. Como toda solución converge a cero, se deduce que el sistema es asintóticamente estable.
%    \item [$(\Leftarrow)$] Suponemos que el sistema $\dot{u}(t) = A(t) u(t)$ es asintóticamente estable. Entonces,
%      \[ 
%        \exists \eta : | u_{0} | \leq \eta \Rightarrow \lim_{t \to +\infty} | u(t; t_0, u_0) | = 0.
%      \] 
%      Sea $u(t; t_0, u_0) \neq 0$ solución, entonces
%      \[ 
%        \omega(t) = \frac{u}{| \overline{u_{0}} |} \cdot u(t; t_0, u_0).
%      \] 
%
%  \end{enumerate}
%\end{dem}

\subsection{Pequeñas Perturbaciones}

\begin{theo}
  Sea el sistema lineal $\dot{{u}}(t) = [A + c(t)] u(t)$ donde $A \in \mathbb{M}_{d \times d}(\mathbb{R})$ con coeficientes constantes y $c \in C^{1}(\mathbb{R}; \mathbb{R}^{d \times d})$. Entonces,
  \begin{enumerate}[label=(\roman*)]
    \item $\dot{u} = A u(t)$ estable y
      \[
        \int_{t_{0}}^{+ \infty} ||c(s)|| ds < + \infty_{j}
      \]
      entonces $\dot{{u}}(t) = [A + c(t)] u(t)$ es estable.
    \item $\dot{u} = A u(t)$ es asintóticamente estable y
      \[
        ||c(t)|| \xrightarrow[]{ t \rightarrow +\infty }0,
      \]
      entonces $\dot{{u}}(t) = [A + c(t)] u(t)$ es asintóticamente estable.
  \end{enumerate}
\end{theo}

\begin{ejm}
  Probar que si $a > 0$ y $c \in \mathbb{R}$, entonces el sistema
  \[ 
    \ddot{x} + (a + \frac{c}{1 + t^{2}})x = 0
  \]
  es estable.
\end{ejm}
%
%\begin{sol}
%  \[ 
%    \begin{aligned}
%      \begin{cases}
%        u = x \\
%        v = \dot{x}
%      \end{cases}
%    \end{aligned}
%  \] 
%  \[ 
%    \Rightarrow
%    \begin{pmatrix}
%       \dot{u} \\
%       \dot{v}
%    \end{pmatrix}
%    =
%    \begin{pmatrix}
%       0 & 1 \\
%       -(a + \frac{c}{1 + t^{2}}) & 0
%    \end{pmatrix}
%    \begin{pmatrix}
%       u \\
%       v
%    \end{pmatrix}
%  \]
%  \[ 
%    = (A + c(t)) \begin{pmatrix}
%       u \\
%       v
%    \end{pmatrix} 
%  \] 
%  donde $A = 
%  \begin{pmatrix}
%     0 & 1 \\
%     -a & 0
%  \end{pmatrix}$
%  \[ 
%    \Rightarrow c(t) =
%    \begin{pmatrix}
%       0 & 1 \\
%       -(a + \frac{c}{1 + t^{2}}) & 0
%    \end{pmatrix} 
%  \] 
%\end{sol}

\begin{ejm}
  Probar que si $a, b > 0$ y $c \in \mathbb{R}$, entonces 
  \[ 
    \ddot{x} + a \dot{x} + (b + c e^{-t} \sen(t)) = 0
  \] 
  es asintóticamente estable.
\end{ejm}

\begin{theo}
  Sea $\dot{u} = (A + C(t))u$ un sistema con $A \in \mathbb{M}_{d \times d} (\mathbb{R})$ y $ C \in C(\mathbb{R}; \mathbb{R}^{d \times d})$. Entonces, si $\dot{u} = A u$ es asintóticamente estable y $||C(t)|| \xrightarrow[]{ t \rightarrow +\infty } 0$, entonces $\dot{u} = ( A + C(t)) u$ es asintóticamente estable.
\end{theo}

\section{Estabilidad sistemas no lineales. Linealización}

\begin{note}
  Estudiamos sistemas de la forma 
  \[ 
    \dot{u}(t) = f(t, u(t)) .
  \] 
  Estudiar la estabilidad de la solución
  \[ 
    u(t; t_0, u_0) 
  \] 
  equivale a estudiar la estabilidad de
  \[ 
    v(t) = u(t; t_0, \overline{u}_{0}) - u(t; t_0, u_0)
  \] 
  \[ 
     
    \dot{v}(t) = \dot{u}(t; t_0, \overline{u}_{0}) - \dot{u}(t; t_0, u_0)
  \] 
  \[ 
    = f(t, u(t; t_0, \overline{u}_{0})) - f(t, u(t; t_0, {u}_{0}))
  \] 
  que verifica
  \[  
    \dot{v}(t) = f(t, v(t) + u(t; t_0, {u}_{0})) - f(t, u(t; t_0, {u}_{0}))
  \] 
  \[ 
    g(t; v) .
  \] 
  Esta ecución se denomina ecución variacional.
\end{note}

\begin{prop}
  Estudiar la propiedades de estabilidad de $u(t; t_0, u_0)$ sol de $\dot{u} = f(t, u)$ es equivalente a estudiar las propiedades de estabilidad de $v(t; t_0, u_0)$ sol de $\dot{v}(t) = g(t; v)$.
\end{prop}

\begin{obs}
  Se tiene que $g(t, 0) = 0 \Rightarrow (t,0)$ es punto de equilibrio.
\end{obs}

\begin{prop}
  Si $| v(t) | \xrightarrow[]{ t \rightarrow +\infty }$ donde $v(t)$ es solución de
  \[ 
    \dot{v}(t) = A(t) v(t) + h(t, v) 
  \] 
  donde $A(t) = D_{u}f(u(t; t_0, u_0))$ y $h(t, v) \xrightarrow[]{ | v(t) | \rightarrow 0 }$. Entonces, las propiedades de estabilidad de $\dot{v}(t)$ equivalen a las propiedades de estabilidad de $\dot{u}(t)$.
\end{prop}

\begin{note}
  Cuando $A(t)$ es de coeficientes constantes podemos asegurar estabilidad asintótica e inestabilidad de las soluciones del sistema no lineal.
\end{note}

\begin{theo}[Linealización; Estabilidad Asintótica]
  Sea $\dot{u} = f(t, u(t))$ us sistema y $ u(t; t_0, u_0)$ solución de $\dot{u}$ tal que su matriz de Linealización sea de coeficientes constantes, es decir, 
  \[ 
    \dot{v} = Av + h(t, v),
  \] 
 y además, 
 \begin{enumerate}[label=(\roman*)]
  \item $\forall \lambda \in \rho(A), \Ree(\lambda) < 0$
  \item $h(t, v) $ es continua para $t \in (t^{*}, +\infty)$ con $t^{*} < t_{0}$ y $h \in C^{1}(v)$ respecto a $v$ en un entorno $V^{0} \subset \mathbb{R}^{d}$. Si suponemos que $| h(t, v) | = \rho(| v |)$ cuando $| v | \rightarrow 0$ uniformemente respecto a $t$.
 \end{enumerate}
\end{theo}
