%\begin{ejr}[7]
%  Sea $S_{1} \subset \mathbb{R}^{3}$ superficie, $f : V_{p} \subset S_{1} \to \mathbb{R}$ ecuación local de $S_{1}$ en un entorno abierto de $p \in \mathbb{R}^{3}$. Es decir, $S_{1} \cap V_{p} = f^{-1}(0) \cap S_{1}$ y $0$ es valor regular de $f$. Demostrar que $S_{1}$ y $S_{2}$ se cortan transversalmente en $p$ si y solo si $f|_{S_{2} \cap V_{p}}$ no tiene punto crítico en $p$.
%\end{ejr}

\begin{ejr}[6]
  Dos superficies regulares $S_{1}, S_{2}$ se cortan transversalmente si $T_{p}(S_{1}) \neq T_{p}(S_{2}), \forall p \in S_{1} \cap S_{2}$. Demostrar que si $S_{1}$ corta transversalmente a $S_{2}$, entonces $S_{1} \cap S_{2}$ es una superficie regular.
\end{ejr}

\begin{sol}
  Como toda superficie es localmente el grafo de una función diferenciable, $S_{1}$ viene dada por $f(x, y, z) = 0$ y $S_{2}$ viene dado por $g(x, y, z) = 0$ en un entorno de $p$, donde $0$ es un valor regular de $f$ y $g$. En este entorno de $p$, $S_{1} \cap S_{2}$ viene dado por la imagen inversa de $(0, 0)$ de la aplicación $F : \mathbb{R}^{3} \to \mathbb{R}^{2} : F(q) = (f(q), g(q))$. Dado que $S_{1}$ y $S_{2}$ se cortan transversalmente, los vcetores normales $(f_{x}, f_{y}, f_{z})$ y $(g_{x}, g_{y}, g_{z})$ son linealmente independientes. Por tanto, $(0, 0)$ es una valor regular de $F$ y $S_{1} \cap S_{2}$ es una curva regular.

  REVISAR
\end{sol}
