\begin{ejr}[Evaluación Continua]
  Sea $S \subset \mathbb{R}^{3}$ superficie. Demostrar que
  \begin{enumerate}
    \item Si $X : U \subset \mathbb{R}^{2} \to S$ es una parametrización y $h : V \subset \mathbb{R}^{2} \to U \subset \mathbb{R}^{2}$ es difeomorfismo, entonces $X \circ h : V \to S$ es parametrización.
    \item Sea $S' \subset \mathbb{R}^{3}$ superficie. Si $X : U \subset \mathbb{R}^{2} \to S$ es parametrización y $\phi : S \to S'$ difeomorfismo, entonces $\phi \circ X : U \to S'$ es parametrización de $S'$.
    \item $Y : U \subset \mathbb{R}^{2} \to S$ parametrización de $Y(U)$ $\Leftrightarrow Y$ es un difeomorfismo.
  \end{enumerate}
\end{ejr}

\begin{sol}
  \begin{enumerate}
    \item [] 
    \item Lo vemos usando la definición. Debemos comprobar que 
      \begin{enumerate}
        \item $X \circ h$ es diferenciable. \\

          $X$ es parametrización $\Rightarrow X$ es diferenciable y $h$ difeomorfismo $\Rightarrow h$ diferenciable. Por tanto, $X \circ h$ es diferenciable ya que la composición de aplicaciones diferenciables es diferenciable.
        \item $X \circ h$ es homeomorfismo. \\

          $X$ parametrización $\Rightarrow X$ homeomorfismo y $h$ difeomorfismo $\Rightarrow h$ homeomorfismo diferenciable con inversa diferenciable. Entoces, $X \circ h$ es homeomorfismo ya que la composición de homeomorfismos es homeomorfismo.
        \item $ d(X \circ h)_{p}$ es inyectiva. \\

          $X$ parametrización $ \Rightarrow (d X)_{q}$ es inyectiva y $h$ difeomorfismo $\Rightarrow (d h)_{p}$ es inyectiva (*). Como la composición de funciones inyectivas es inyectiva, entonces $d(X \circ h)$ es inyectiva.


      \end{enumerate}
      Por tanto, $X \circ h$ es parametrización.
    \item Usamos que $Y$ es parametrización $\Leftrightarrow Y$ es difeomorfismo. Como $X$ parametrización $\Rightarrow X$ difeomorfismo, entonces $\phi \circ X$ es difeomorfismo por ser composición de difeomorfismos. Por tanto, $\phi \circ X$ difeomorfismo $\Rightarrow \phi \circ X$ parametrización.
    \item 
      \begin{enumerate}[label=(\roman*)]
        \item [$(\Rightarrow)$] Si $Y : U \subset \mathbb{R}^{2} \to S$ es una parametrización, entonces
          \[ 
            Y^{-1} : Y(U) \to \mathbb{R}^{2}
          \] 
          es diferenciable. Además, $\forall p \in Y(U)$, $\forall Z : V \subset \mathbb{R}^{2} \to S$ parametrización,
          \[ 
            Y^{-1} \circ Z : Z^{-1}(W) \to Y^{-1}(W)
          \] 
          donde $W = Y(U) \cap Z(V)$, es diferenciable. Por tanto, $U$ y $Y(U)$ son difeomorfos.
        \item [$(\Leftarrow)$] Sea $S \subset \mathbb{R}^{3}$ superficies. Si $Y : U \subset \mathbb{R}^{2} \to S$ es difeomorfismo, entonces $Y$ es diferenciable, $Y$ es homeomorfismo y $(d Y)_{p}$ (*) es inyectiva. Por tanto, $Y$ es parametrización de $S$. \\
      \end{enumerate}
  \end{enumerate}

  (*) Veamos que $f : X \subset \mathbb{R}^{m} \to  Y \subset \mathbb{R}^{m}$ difeomorfismo $\Rightarrow (d f)_{p}$ isomorphismo, $p \in X : f(p) \in Y$. Si $f$ difeomorfismo, entonces $f$ tiene inversa $f^{-1}$. Ahora, 
\[ 
  (d I_{Y})_{f(p)} = d(f \circ f^{-1})_{f(p)} = (d f)_{p} \circ (d f^{-1})_{f(p)},
\] 
\[ 
  (d I_{X})_{p} = d(f^{-1} \circ f)_{p} = (d f^{-1})_{f(p)} \circ (d f)_{p} 
\] 
entonces, $(d f)_{p}$ es un isomorfismo.
\end{sol}
