
\begin{ejr}[35]
  Sea $\alpha  : I \subset \mathbb{R} \to \mathbb{R}^{3}$ curva p.p.a., $M: \mathbb{R}^3 \to \mathbb{R}^3$ movimiento rígido y $\beta = M \circ \alpha$ curva. Demostrar
  \begin{enumerate}[label=(\roman*)]
    \item $M$ conserva la orientación $\Rightarrow k_{\beta} = k_{\alpha}$, $\tau_{\beta} = \tau_{\alpha}$,
    \item $M$ invierte la orientación $\Rightarrow k_{\beta} = k_{\alpha}$, $\tau_{\beta} = - \tau_{\alpha}$.
  \end{enumerate}
\end{ejr}

\begin{sol}[35]
  Sea $\beta = (M  \circ \alpha)$ donde $\phi: \mathbb{R}^3 \to \mathbb{R}^3 : t \mapsto \phi(t) = At + \vec{v}$ es un movimiento rígido con $A$ matriz ortonormal asociada a la isometría y $\vec{v} \in \mathbb{R}^3$.
  
  Sea $\alpha$ p.p.a entonces,
  \[ 
    ||\beta'|| = ||(M \circ \alpha )'|| = ||A \alpha'|| = ||\alpha'|| = 1
  \] 
  $\beta$ es p.p.a.. Esto se debe a que
  \[ 
    d_{t} M = \frac{d{}}{d{t}} (At + \vec{v}) = A
  \]
  \[ 
    \Rightarrow d_{t} (M \circ \alpha) = \frac{d{}}{d{t}}(A \alpha(t) + \vec{v}) = A \alpha'(t)
  \] 
  y dado que $A$ es ortonormal, es decir, $A^{t} = A^{-1}$
  \[
    \Rightarrow ||A x|| = \sqrt{ Ax \cdot Ax } = \sqrt{ x \cdot A^{t}A } = ||x||, \; \forall x \in \mathbb{R}^{3}
  \]
  $\Rightarrow A$ conserva la norma. \\


  Para la curvatura de $\beta$, que es $k_{\beta} = ||\beta''||$, tenemos que
  \[ 
    k_{\beta} = ||(M \circ \alpha)''|| = ||(A \alpha + \vec{v})''|| = ||A \alpha''|| = ||\alpha''|| =  k_{\alpha}
  \]
  dado que $A$ es la matriz asociada a la ismoetría del movimiento rígido $M$, y conserva la norma. Entonces, $k_{\beta} = k_{\alpha} \Rightarrow$ la curvatura es invariante por movimiento rígido.  \\

  Y para la torsión de $\beta$ que es
  \[ 
   \tau_{\beta} = (\beta ' \times \beta '') \cdot \beta''' = (A \alpha' \times A \alpha'') \cdot A \alpha'''
  \] 
  \[ 
    = \det(A) A (\alpha' \times \alpha'') \cdot A \alpha'''
  \] 
  \[ 
    = \det(A) (\alpha' \times \alpha'') \cdot \alpha'''
  \] 
  \[ 
    = \det(A) \tau_{\alpha}
  \] 
  \[ 
    \pm \det(A) \tau_{\alpha} 
  \] 
%  donde
%  \[ 
%     det(A) =
%    \begin{cases}
%     1, \text{ si $A$ conserva la orientación }\\
%     -1, \text{ si $A$ invierte la orientación }\\
%    \end{cases} 
%  \] 
  esto se debe a que el producto vectorial bajo transformaciones de matrices obedece $(Ba) \times (Bb) = (\det(B))(B^{-1})^t (a \times b), B \in \mathcal{M}_{3 \times 3}, a, b \in \mathbb{R}^{3}$. Luego,  $A$ es ortogonal por ser la matriz asociada a una isometría linea $\Rightarrow (A^{t})^{-1} = A$. Y $\det(A) = \pm 1$ por ser $A$ matriz ortogonal.\\

  Por tanto, la torsión de $\beta$ es
  \[ 
    \tau_{\beta} =
    \begin{cases}
      \tau_{\alpha}, \text{ si $A$ conserva la orientación} \\
      -\tau_{\alpha}, \text{ si $A$ invierte la orientación}
    \end{cases}
  \] 
\end{sol}
