\chapter{Curvas}

\begin{ejr}[33]
  Sea $\alpha  : I \subset \mathbb{R} \to \mathbb{R}^{3}$ curva p.p.a., $M: \mathbb{R}^3 \to \mathbb{R}^3$ movimiento rígido y $\beta = M \circ \alpha$ curva. Demostrar
  \begin{enumerate}[label=(\roman*)]
    \item $M$ conserva la orientación $\Rightarrow k_{\beta} = k_{\alpha}$, $\tau_{\beta} = \tau_{\alpha}$,
    \item $M$ invierte la orientación $\Rightarrow k_{\beta} = - k_{\alpha}$, $\tau_{\beta} = \tau_{\alpha}$.
  \end{enumerate}
\end{ejr}

\begin{sol}
  Sea $\beta = M \alpha$ donde $M \in \mathcal{M}_{3 \times 3}(\mathbb{R})$. Entonces,
  \[ 
    k_{\beta} = ||\beta''|| = ||M \alpha''|| = ||M|| ||\alpha''|| = ||M|| k_{\alpha}
  \]
  donde
  \[ 
    ||M||  =
    \begin{cases}
      1, \text{ si $M$ conserva la orientación} \\
      -1, \text{ si $M$ invierte la orientación}
    \end{cases}
  \] 
  \[ 
    \Rightarrow k_{\beta} =
    \begin{cases}
      k_{\alpha}, \text{ si $M$ conserva la orientación} \\
      -k_{\alpha}, \text{ si $M$ invierte la orientación}
    \end{cases}
  \] 
  La torsión de $\beta$ es
  \[ 
   \tau_{\beta} = (\beta ' \times \beta '') \cdot \beta''' = (M \alpha' \times M \alpha'') \cdot M \alpha'''
  \] 
  \[ 
    = (\alpha' \times \alpha'') \cdot \alpha''' = \tau_{\alpha} 
  \] 
  Por tanto, la torsión es invariante ante isometrías.
\end{sol}
