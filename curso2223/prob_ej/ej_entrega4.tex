\section{Entrega 4}

\begin{ejr}[Ejercicio 2, Hoja 6]
  Estudiar, para cada una de las siguientes funciones, si es función característica de alguna variable aleatoria y, en caso afirmativo, determinar la función de probabilidad correspodiente.
  \begin{enumerate}[label=(\roman*)]
    \item $\varphi(t) = \sum_{n = 0}^{\infty} a_{n} \cos(n t)$ con $a_{n} \geq 0$ y $\sum_{n = 0}^{\infty} a_{n} = 1$,
    \item $\varphi(t) = \frac{1}{1 + a(1 - e^{it})}$ con $a > 0$.
  \end{enumerate}
\end{ejr}

\begin{sol}
  \begin{enumerate}[label=(\roman*)]
    \item []
    \item Sea $\varphi_{n}(t) = \cos(n t)$ función característica, entonces $\sum_{n = 0}^{\infty} a_{n} \varphi_{n}(t)$ es una combinación lineal convexa de $\{ \varphi_{n}, n \in \mathbb{N} \}$ numerable. Por tanto, $\varphi(t)$ también es función característica. 

      Para $X_{n}$ v.a. discreta con $D_{X_{n}} = \{ -n, n \}$ y función de masa
      \[ 
        p_{X_{n}} =
        \begin{aligned}
          \begin{cases}
            \frac{1}{n} \quad \text{ si } x \in D_{X} \\
            0 \quad \text{ si } x \not \in D_{X}
          \end{cases}
        \end{aligned}
      \] 
      Entonces, se tiene que
      \[
        \varphi_{n}(t) = \mathbb{E} [ e^{itX_{n}} ] 
      \]
      \[ 
        = \frac{e^{itn} + e^{-itn}}{2} = \cos(nt) 
      \] 
      Por tanto la función de distribución asociada a $\varphi_{n}$ es una distribución uniforme discreta $U(-n,n)$. 
    \item Aplicando el teorema de inversión 
      \[ 
        f(x) = \frac{1}{2 \pi} \int_{-\infty}^{\infty} e^{-itx} \frac{1}{1 + a(1 - e^{it})} dt
      \] 
  \end{enumerate}
\end{sol}
