

\section{Entrega 3}

\begin{ejr}[Ejercicio 2, Hoja 4]
  Sean $A$ y $B$ dos sucesos tales que $P(A) = \frac{1}{4}, P(B | A) = \frac{1}{2}$ y $P(A | B) = \frac{1}{4}$. Indicar si son ciertas o falsas las siguientes afirmaciones:
  \begin{enumerate}
    \item $A \subset B$,
    \item $A$ y $B$ independientes,
    \item $A^{c}$ y $B^{c}$ independientes,
    \item $A$ y $B$ incompatibles,
    \item $P(A^{c} | B^{c}) = \frac{1}{2}$,
    \item $P(A | B) + P(A | B^{c}) = 1$
  \end{enumerate}
\end{ejr}

\begin{sol}
  Sea $(\Omega, \mathcal{A}, P )$ donde $\mathcal{A} \subset \mathcal{P}(\Omega)$, $A, B \in \mathcal{A}$ tal que $P(A) = \frac{1}{4}, P(B | A) = \frac{1}{2}$ y $P(A | B) = \frac{1}{4}$.
  \begin{enumerate}[label=(\roman*)]
    \item (FALSO) Supongamos que $A \subset B$, entonces $A \cap B = A \Rightarrow P(A \cap B) = P(A)$ y

      \[
        P(B | A) = \frac{P(A \cap B)}{P(A)} = \frac{P(A)}{P(A)} = 1
      \]
      Es contradicción, dado que $P(B | A) = \frac{1}{2}$. Por tanto, $A \not \subset B$.

    \item (CIERTO) Sabemos que
      \[ 
        P(B | A) = \frac{P(A \cap B)}{ P(A)},
      \] 
      entonces
      \[ 
        P(B | A) \cdot P(A) = P(A \cap B) 
      \] 
      \[ 
        \Rightarrow P(A \cap B) = \frac{1}{4} \cdot \frac{1}{2} = \frac{1}{8} 
      \]  
      Ahora,
      \[ 
        P(A | B) = \frac{P(A \cap B)}{P(B)}
      \] 
      \[ 
        \Rightarrow P(B) = \frac{P(A \cap B)}{P(A | B)} = \frac{\frac{1}{8}}{\frac{1}{4}} = \frac{1}{2} 
      \] 
      Como $A$ y $B$ son independientes $\Leftrightarrow$
      \[ 
        P(A \cap B) = P(B) \cdot P(A) 
      \] 
      donde $P(A \cap B) = \frac{1}{8}$ y $P(A) \cdot P(B) = \frac{1}{4} \cdot \frac{1}{2} = \frac{1}{8}$. Deducimos que $A$ y $B$ son independientes.

    \item (CIERTO) Veamos que $A$ y $B$ independientes $\Rightarrow$ $A^{c}$ y $B^{c}$ independientes. Queremos ver que $P(A^{c} \cap B^{c}) = P(A^{c})\cdot P(B^{c})$.
      \[ 
        P(A^{c} \cap B^{c}) = P((A \cup B)^{c}) = 1 - P(A \cup B) = 1 - P(A) - P(B) + P(A \cap B)
      \] 
      donde $P(A \cap B) = P(A) \cdot P(B)$ por ser $A$ y $ B$ independientes,
      \[ 
        P(A^{c} \cap B^{c}) = 1 - P(A) - P(B) + P(A)\cdot P(B)
      \] 
      \[ 
        = 1 - P(A) - P(B)(1 - P(A)) = P(A^{c}) - P(B) \cdot P(A^{c}) 
      \]
      \[ 
        = P(A^{c})(1 - P(B^{c})) = P(A^{c}) \cdot P(B^{c}) 
      \] 

    \item (FALSO) Si $A$ y $B$ son incompatibles, $P(A \cap B) = 0$ pero $P(A \cap B) = \frac{1}{8}$.
    \item (FALSO) Por ser $A^{c}$ y $B^{c}$ independientes, tenemos que
      \[
        P(A^{c} | B^{c}) = P(A^{c}) = 1 - P(A) = \frac{3}{4} \neq \frac{1}{2}
      \] 

    \item (FALSO) Por ser $A$ y $B$ independientes, $A^{c}$ y $B^{c}$ son independientes. También lo son $ A$ y $B^{c}$ ya que
      \[ 
        P(A^{c} \cap B) = P(B \setminus (A \cap B)) = P(B) - P(A \cap B)
      \] 
      donde $A \cap B \subset B$, entonces
      \[ 
        P(A^{c} \cap B) = P(B)(1 - P(A)) = P(B) \cdot P(A^{c}) 
      \] 
      A partir de la independencia de estos sucesos,
      \[ 
        P(A | B) + P(A | B^{c}) = P(A) + P(A) = \frac{1}{4} + \frac{1}{4} = \frac{1}{2} \neq \frac{1}{2}.
      \] 
  \end{enumerate}
\end{sol}

\begin{ejr}[Ejercicio 6, Hoja 4]
  Sean $A_{1}, A_{2}, A_{3}, A_{4}, A_{5}$ sucesos independientes. Demostrar que:
  \begin{enumerate}[label=(\roman*)]
    \item $A_{1} \cup A_{2}$, $A_{3} \cap A_{4}$ y $A_{5}$ son independientes,
    \item $(A_{1} \cup A_{2}) \cap A_{3}$ y ${A_{4}}^c \cup {A_{5}}^c$ son independientes.
  \end{enumerate}
\end{ejr}

\begin{sol}
  Veamos primero que si $A, B$ y $C$ son independientes, entonces $A \cup B$ y $C$ también lo son.
  \[ 
    P((A \cup B) \cap C) = P((A \cap C) \cup (B \cap C))  
  \] 
  \[ 
    = P(A \cap C) + P(B \cap C) - P(A \cap B \cap C) 
  \]
  \[ 
    = P(A) \cdot P(C) + P(B) \cdot P(C) - P(A) \cdot P(B) \cdot P(C) 
  \] 
  \[ 
    = P(C) \cdot (P(A) + P(B) - P(A) * P(B)) 
  \] 
  \[ 
    = P(C)\cdot P(A \cup B) 
  \] 
  por tanto, $A \cup B$ y $C$ son independientes. \\

  Veamos ahora que si $A, B$ y $C$ son independientes, entonces $A \cap B$ y $C$ son independientes.
  \[ 
    P((A \cap B) \cap C) = P(A \cap B \cap C) 
  \] 
  \[ 
    = P(A) \cdot P(B) \cdot P(C) 
  \] 
  \[ 
    = P(A \cap B) \cdot P(C) 
  \] 
  por tanto, $A \cap B$ y $C$ son independientes.

  \begin{enumerate}[label=(\roman*)]
    \item Por lo que hemos visto antes tenemos que $A_{1} \cup A_{2}$ y $A_{5}$ son independientes y $A_{3} \cap A_{4}$ y $A_{5}$ son independientes. Vemos que $A_{1} \cup A_{2}$ y $A_{3} \cap A_{4}$ son independientes. 
      \[ 
        P((A_{1}\cup A_{2}) \cap (A_{3} \cap A_{4})) = P((A_{1} \cap A_{3} \cap A_{4}) \cup (A_{2} \cap A_{3} \cap A_{4})) 
      \] 
      \[ 
        = P(A_{1} \cap A_{3} \cap A_{4}) + P(A_{2} \cap A_{3} \cap A_{4}) - P((A_{1} \cap A_{3} \cap A_{4}) \cap (A_{2} \cap A_{3} \cap A_{4})) 
      \] 
      \[ 
        =P(A_{1}) \cdot P(A_{3} \cap A_{4}) + P(A_{2}) \cdot P(A_{3} \cap A_{4}) - P(A_{1} \cap A_{2} \cap A_{3} \cap A_{4})
      \] 
      \[ 
        =P(A_{1}) \cdot P(A_{3} \cap A_{4}) + P(A_{2}) \cdot P(A_{3} \cap A_{4}) - P(A_{1}) \cdot P(A_{2}) \cdot P(A_{3} \cap A_{4})
      \] 
      \[ 
        = P(A_{3} \cap A_{4}) \cdot (P(A_{1}) + P(A_{2}) + P(A_{1}) \cdot P(A_{2})) 
      \] 
      \[ 
        = P(A_{3} \cap A_{4}) \cdot P(A_{1} \cup A_{2}) 
      \] 
      Por tanto, $A_{1} \cup A_{2}$ y $A_{3} \cap A_{4}$ son independientes. 
    \item Vemos que $(A_{1} \cup A_{2}) \cap A_{3}$ y ${A_{4}}^c \cup {A_{5}}^c$ son independientes.

      Sean $A$ y $B$ sucesos independientes arbitrarios $\Rightarrow A$ y $B^{c}$ independientes, $A^{c}$ y $B^{c}$ independietes, y $A^{c}$ y $B$ independientes (demostrado en el ejercicio 2). Como $A_{4}^{c} \cup A_{5}^{c} = (A_{4} \cap A_{5})^{c}$ entonces, basta ver que $(A_{1} \cup A_{2}) \cap A_{3}$ y $A_{4} \cap A_{5}$ son independientes.

      \[ 
        P ((A_{1} \cup A_{2}) \cap A_{3}) \cap ({A_{4}} \cap {A_{5}})) = P \Big [ \big [ (A_{1} \cup A_{2}) \cap (A_{4} \cap A_{5}) \big ] \cap \big [ A_{3} \cap (A_{4} \cap A_{5}) \big ] \Big ]
      \] 
      \[ 
        = P((A_{1} \cup A_{2}) \cap (A_{4} \cap A_{5}) \cap A_{3})
      \] 
      donde $A_{1} \cup A_{2}$, $A_{4} \cap A_{5}$ y $A_{3}$ son independientes. Por tanto,
      \[ 
        P((A_{1} \cup A_{2}) \cap (A_{4} \cap A_{5}) \cap A_{3}) = P(A_{1} \cup A_{2}) \cdot P(A_{4} \cap A_{5}) \cdot P(A_{3})
      \] 
      \[ 
        = P((A_{1} \cup A_{2}) \cap A_{3}) \cdot P(A_{4} \cap A_{5}) 
      \] 
      entonces $(A_{1} \cup A_{2}) \cap A_{3}$ y $A_{4} \cap A_{5}$ son independientes $\Rightarrow$ $(A_{1} \cup A_{2}) \cap A_{3}$ y $A_{4}^{c} \cup A_{5}^{c}$ son independientes.
  \end{enumerate}
\end{sol}
