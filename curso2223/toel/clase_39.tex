\begin{defn}[Simplemente Conexo]
  Sea $X$ e.t.. Se dice que es simplemente conexo si es conexo por caminos y su grupo fundamental es trivial.
\end{defn}

\begin{prop}
  Si $X,Y$ e.t. homotópicamente equivalentes tal que $X$ es simplemente conexo entonces $Y$ es simplemente conexo.
\end{prop}

\begin{dem}
  Sea $X$ c.p.c.. Como $X$ es homotópicamente equivalente a $Y$, se tiene que $Y$ es c.p.c.. Y $\pi_{1}(X) \simeq 0 \xRightarrow[]{ \text{homo. equiv.} } \pi_{1}(Y) \simeq 0$.
\end{dem}

\begin{cor}
  El ser simplemente conexo es invariante topológico.
\end{cor}

\begin{dem}
  Esto se debe a que la propiedad de equivalencia homotópica es más fuerte que la de invariancia.
\end{dem}

\begin{prop}
  Todo e.t. contractil es simplemente conexo.
\end{prop}

\begin{dem}
  $X$ contractil, $\forall x, y \in X$ $\Rightarrow 1_{X} \simeq c_{x}$ y $1_{X} \simeq c_{y} \Rightarrow H$ homotopía de $c_{x}$ en $c_{y}$, entonces $\exists h : I \to X$ continua definida por 
  \[ 
    h(t) = H(x,t).
  \]  
  De manaera que
  \[ 
    h(0) = H(x,0) = c_{x}(x) = x 
  \] 
  \[ 
    h(1) = H(y, 1) = c_{y}(x) = x
  \] 
  Entonces, $X$ es c.p.c y $X$ tiene homotopía de un solo punto $\Rightarrow \pi_{1}(X) \simeq 0$.
\end{dem}

\begin{prop}
  Si $X, Y$ e.t.. Entonces, $X \times Y$ simplemente conexo $\Leftrightarrow X, Y$ simplemente conexo.
\end{prop}

\begin{dem}
  $X \times Y$ c.p.c. $\Leftrightarrow X, Y$ c.p.c. y $\pi_{1}(X \times U, (a, b)) \simeq \pi_{1}(X,a) \times \pi_{1}(Y, y)$.
  \begin{enumerate}[label=(\roman*)]
    \item [$(\Rightarrow)$] Ejercico
    \item [$(\Leftarrow)$] Ejercicio
  \end{enumerate}
\end{dem}

\begin{obs}
  El ser simplemente conexo no es propiedad hereditaria.
\end{obs}

\begin{ejm}
  $\mathbb{R}^{2}$ es contractil $\Rightarrow \mathbb{R}^{2}$ es simplemente conexo pero $\mathbb{S}^{1}$ no lo es.
\end{ejm}

\begin{obs}
  El ser simplemente conexo no se conserva por *.
\end{obs}

\begin{ejm}
  $I$ es contractil $\Rightarrow I$ es simplemente conexo pero $\mathbb{S}^{1}$ no lo es.
\end{ejm}
