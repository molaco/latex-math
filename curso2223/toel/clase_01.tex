\part{Topología General}
\chapter{Espacios Topológicos Arbitrarios}
\section{Espacios Topológicos}

\begin{defn}[Topología]
  Se llama topología sobre un conjunto $X$ a $\forall \tau \subset \mathcal{P}(X)$ que verifique:
  \begin{enumerate}[label=(\roman*)]
    \item [(G1)] $\emptyset,X \in \mathcal{T}$.
    \item [(G2)] $\forall A_{1}, A_{2} \in \mathcal{T} \Rightarrow A_{1}\cap A_{2} \in \mathcal{T}$
    \item [(G3)] $\forall \{ A_{j} \}_{j \in J} \subset \mathcal{T} \Rightarrow \bigcup_{j \in J} A_{j} \in \mathcal{T}$
  \end{enumerate}
\end{defn}

\begin{obs}
  Al par $\big( X, \mathcal{T} \big)$ se denomina \textit{espacio topológico} y los elementos de $X$ son puntos del espacio topológico.
\end{obs}

\begin{ejm}
  \begin{enumerate}[label=(\roman*)]
    \item   Sea $X$ un conjunto, entonces $\mathcal{P}(X) = \mathcal{T}_{D}$ es una topología y se llama topología discreta.
    \item  La colección $\mathcal{T} = \{ X, \emptyset \}$ es también una topología y la llamamos topología trivial.
    \item Sea $(X, d)$ un espacio métrico y sea $\mathcal{T}_{d} = \{ U \subset X : \forall x \in U, \exits \epsilon > 0 : B_{\epsilon} \subset U \}$ es una topología y la llamamos topología inducida por la métrica $d$.
  \end{enumerate}
\end{ejm}

\begin{obs}
  Toda métrica induce un espacio topológico pero no todo espacio topológico es inducido por una métrica.
\end{obs}

\begin{defn}[Espacio Metrizable]
  Sea $\big( X, \mathcal{T} \big)$ e.t., decimos que es un \textit{espacio matizable} si $\exits d$ métrica sobre $X$ tal que $\mathcal{T} = \mathcal{T}_{d}$.
\end{defn}

\begin{defn}[Conjunto Abierto]
  Sea $\big( x, \mathcal{T} \big)$ espacio topológico, decimos que $U \subset X$ es un \textit{conjunto abierto} si $U \in \mathcal{T}$.
\end{defn}

\begin{obs}
  Si $U$ es un conjunto abierto, entonces $X \setminus U$ es un conjunto cerrado.
\end{obs}

\begin{obs}
  Existen conjuntos que son abiertos y cerrados simultáneamente. Y existen conjuntos que no son ni abiertos ni cerrados.
\end{obs}

\begin{ejm}
  Sea el espacio topológico $\big( \mathbb{R}, \mathcal{T}_{u} \big)$ entonces $S = (0,1])$ no es ni abierto ni cerrado.
\end{ejm}

\begin{ejm}
  Sea el espacio topológico $\big( X, \mathcal{T}_{d} \big)$ donde $\mathcal{T}_{d}=\mathcal{P}(X)$ entonces $\forall S \subset X$, $S$ es abierto y cerrado simultáneamente.
\end{ejm}
