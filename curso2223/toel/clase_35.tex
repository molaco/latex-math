
\begin{defn}[Grupo Fundamental]
  Sea $X$ e.t., $x_{0} \in X$. Se llama grupo fundametal con base $x_{0}$ a $(\pi_{1}(X, x_{0}))$.
\end{defn}

\begin{obs}
  En la demostración anterior usamos caminos en lugar de lazos. Por tanto, tenemos lo siguiente.
  \begin{enumerate}[label=(\roman*)]
    \item Si $f,g,h$ caminos en $X$ e.t. tal que $f(1) = g(0)$ y $g(1) = h(0)$, entonces
      \[ 
        ((f*g) * h) \simeq_{\{ 0,1 \}} f * ( g* h)(s).
      \] 
    \item Si $f$ es camino en $X$ de origen $a$ y extremo $b$, entonces
      \[ 
        c_{a} * f \simeq_{\{ 0, 1 \}} f * c_{b}.
      \] 
      donde $c_{a}, c_{b}$ son caminos constantes $a, b$ respectivamente.
    \item Si $f$ es camino en $X$ de origen $a$ y extremo $b$, entonces
      \[ 
        f * f' \simeq_{\{ 0, 1 \}} c_{a}, \quad f * f' \simeq_{\{ 0, 1 \}} c_{b}
      \] 
  \end{enumerate}
\end{obs}

\begin{obs}
  Solo se conserva la estructura de grupo al cambiar de base si $X$ es conexo por caminos.
\end{obs}

\begin{obs}
  Sea $X$ e.t., $x, y \in X$, $h$ camino en $X$ tal que $h(0) = X$, $h(1) = y$ entonces, $\forall f$ lazo en $X$ con base $x$, $(h' * f) * h$ es lazo en $x$ con base $y$. \\

  Si $f,g$ son lazos en $X$ con base $x$ tal que $f \simeq_{\{ 0, 1 \}} g$ y $F$ es homotpía de $f$ en $g$ relativa a $\{ 0, 1 \}$ entonces, sea $H : I \times I \to x$ aplicación definida por
  \[ 
    H(s,t) =
    \begin{aligned}
      \begin{cases}
        h'(4s), \quad 0 \leq s \leq \frac{1}{4} \\
        F(4s - 1, t), \quad \frac{1}{4} \leq s \leq \frac{1}{2} \\
        h(2s - 1), \quad \frac{1}{2} \leq s \leq 1
      \end{cases}
    \end{aligned}
  \] 
  Esta aplicación esta bien definida y $C_{i}$ son cerrados de $I^{2}$ y lo recubren, entonces $H|_{C_{i}}$ continua $\Rightarrow H$ continua.
  \[ 
    H(s,0) = h' * (f * h)
  \] 
  \[ 
    H(s,1) = (h' * f) * h 
  \] 
  Por tanto, $H$ es homotpía de $(h' * f) * h$ en $(h' * g) * h$. Veamos el resultado por caminos
  \[ 
    h'(0) = ((h' * f) * h)(0) = ((h' * g) * h)(0) 
  \] 
  \[ 
    h'(1) = ((h' * f) * h)(1) = ((h' * g) * h)(1)
  \] 
\end{obs} 

\begin{prop}
  Sea $X$ e.t. conexo por caminos, entoces $\forall x, y \in X$, $\pi_{1}(X,x), \pi_{1}(X, y)$ son isomorfos.
\end{prop}
