\begin{lem}[Jones]
  Sea $( X, \mathcal{T} )$ e.t.. Si $\exists D$ denso en $ ( X, \mathcal{T} )$, $E$ cerrado en $( X, \mathcal{T} )$ tal que $( E, \mathcal{T}|_{E})$ es discreto y $\card(E) \geq 2^{\card(D)}$. Entonces, $( X, \mathcal{T} )$ no es normal.
\end{lem}

\begin{dem}
  Sea $( X, \mathcal{T} )$ normal, $\forall C \subset E, C$ y $E$ son disjuntos y cerrados en $( E, \mathcal{T}|_{E})$ y en $( X, \mathcal{T} )$ que es normal. Entonces, $\exists U_{C}, V_{C} \in \mathcal{T}: C \subset U_{C}, E \setminus C \subset V_{C} $. Ahora, sea $f: \mathcal{P}(E) \to \mathcal{P}(D): c \mapsto f(c) = U_{C} \cap D$ aplicación. Veamos que $f$ es inyectiva, $\forall C_{1}, C_{2} \in \mathcal{P}(E) : C_{1} \neq C_{2} \Rightarrow \exists x \in C_{1} : x \not \in C_{2} \Rightarrow C_{i} \subset U_{C_{i}}, E \setminus C_{i} \subset V_{C_{i}}, i \in \{ 1, 2 \} \Rightarrow x \in U_{C_{1}} \cap V_{C_{2}} \in \mathcal{T} \Rightarrow (\text{ $D$ denso }) y \in U_{C_{1}} \cap V_{C_{2}} \cap D \neq \emptyset \Rightarrow$
  \[ 
    \begin{cases}
      y \in U_{C_{1}} \cap D = f(C_{1}) \\ 
      y \not \in U_{C_{2}} \cap D = f(C_{2}) 
    \end{cases} 
  \] 
  $\Rightarrow f(C_{1}) \neq f(C_{2}) \Rightarrow  \card(E) < \card(\mathcal{P}(E)) \leq \card(\mathcal{P}(D)) = 2^{\card(D)} $
\end{dem}

\begin{prop}
  Sea $( X, \mathcal{T} )$ e.t. normal $( T_{4})$, $E \subset X,$ $E$ cerrado en $( X, \mathcal{T} )$. Entonces, $( E, \mathcal{T}|_{E})$ es normal $(T_{4})$.
\end{prop}

\begin{dem}
  $\forall C_{1}, C_{2}$ cerrados disjuntos en $( E, \mathcal{T}|_{E})$. Entonces, $C_{1}, C_{2}$ cerrados disjuntos de $ ( X, \mathcal{T} )$ que es normal $ \Rightarrow \exists U_{i} \in \mathcal{T}$ disjuntos tal que $C_{i} \subset U_{i},  i \in \{  1, 2 \} \Rightarrow U_{i} \cap E \in \mathcal{T}|_{E}$ disjuntos tal que $C_{i} \subset U_{i} \cap E, i \in \{ 1, 2 \}$.
\end{dem}

\begin{obs}
  El producto de e.t. normales no es necesariamente normal.
\end{obs}

\begin{prop}
  Sea $\{ ( X_{j}, \mathcal{T}_{j} ) \}_{j \in J}$ familia de e.t.. Entonces, $( \sum_{k \in J} X_{k}, \sum_{k \in J} \mathcal{T}_{k})$ es normal $(T_{4}) \Leftrightarrow \forall j \in J, ( X_{j}, \mathcal{T}_{j} )$ es normal $(T_{4})$.
\end{prop}

\begin{dem}
  \begin{enumerate}[label=(\roman*)]
    \item [($\Rightarrow$)] $\forall j_{0} \in J, X_{j_{0}} \simeq X_{j_{0}} \times \{ j_{0} \} \subset \sum_{ j \in J } X_{j}$ es cerrado.
    \item [($\Leftarrow$)] $\forall C_{1}, C_{2}$ cerrados disjuntos de $( \sum_{k \in J} X_{k}, \sum_{k \in J} \mathcal{T}_{k}) \Rightarrow \forall k \in J, j_{k}^{-1}(C_{1}), j_{k}^{-1}(C_{2})$ cerrados disjuntos de $( X_{k}, \mathcal{T}_{k} )$ que es normal $\Rightarrow \forall j \in J, \exists U_{k,1}, U_{k,2} \in \mathcal{T}_{k}$ disjuntos tal que $j_{k}^{-1}(C_{i}) \subset U_{k,i} \in \mathcal{T}_{k}, i \in \{ 1, 2 \}$. Entonces, $U_{1} = \bigcup_{k \in J} U_{k,1} \times \{ k \}, \bigcup_{k \in J} U_{k,1} \times \{ k \} \in \sum_{k \in J} \mathcal{T}_{k}$ son abiertos disjuntos y $C_{1} \subset U_{1}, C_{2} \subset U_{2}$.
  \end{enumerate}
\end{dem}

\begin{prop}
  Sea $( X, \mathcal{T} )$, $( Y, \mathcal{S} )$ e.t. tal que $ ( X, \mathcal{T} )$ es normal, $f: ( X, \mathcal{T} ) \to ( Y, \mathcal{S} )$ suprayectiva, continua y cerrada. Entonces, $( Y, \mathcal{S} )$ es normal $(T_{4})$.
\end{prop}

\begin{dem}
  %
$\forall C_{1}, C_{2}$ cerrado disjunto $( Y, \mathcal{T} ) \Rightarrow f^{-1}(C_{1}), f^{-1}(C_{2})$ cerrados disjuntos de $( X, \mathcal{T} ) \Rightarrow \exists U_{i} \in \mathcal{T}$ disjuntos tal que $f^{-1}(C_{i}) \subset U_{i}, i \in \{ 1, 2 \} \Rightarrow (f \text{ cerrada }) V_{i} = Y \setminus f(X \setminus U_{i}) \in \mathcal{S}, i \in \{ 1, 2 \} $ 
\[
  \Rightarrow V_{1} \cap V_{2} = ( Y \setminus f(X \setminus U_{1}) ) \cap ( Y \setminus f(X \setminus U_{2})) 
\]
\[
  = (Y \setminus (f(X \setminus U_{1}) \cup f(X \setminus U_{2})) )
\]
\[ 
  = Y \setminus f(X \setminus U_{1} \cup X \setminus U_{2}) 
\] 
\[ 
  = Y \setminus f(X \setminus (U_{1} \cap U_{2}))
\] 
\[ 
  Y \setminus f(X) = Y \setminus Y = \emptyset 
\] 
$\Rightarrow$ disjuntos. \\

¿$C_{i} \subset X \setminus f(X \setminus U_{i})$? $ \forall y \in C_{i} \Rightarrow f^{-1}(y) \subset f^{-1}(C_{i}) \subset U_{i} \Rightarrow X \setminus U_{i} \subset X \setminus f^{-1}(y) \Rightarrow f(X \setminus U_{i} ) \subset f(X \setminus f^{-1}(y)) \Rightarrow z \in Y \setminus f(X \setminus f^{-1}(y)) \subset Y \setminus f(X \setminus U_{i}) = V_{i} \Rightarrow z \not \in f(X \setminus f^{-1}(y)) \Rightarrow z = f(x') : x' \in X \setminus f^{-1}(y) \Rightarrow x' \in f^{-1}(y) \Rightarrow f(x') = y \Rightarrow z = y \Rightarrow \forall y \in C_{i}, y \in V_{i}$. \\

Para ver que es $T_{4}$:  $( X, \mathcal{T} )$ es $T_{4}$, $\forall y \in Y, \exists x \in X, f(x) = y \Rightarrow (T_{1}) \ \{ x \} \text{ cerrado } ( X, \mathcal{T} ) \Rightarrow (f \text{ cerrada }) \ f(\{ x \})$ es cerrado en $( Y, \mathcal{S} )$ y $ f(\{ x \}) = \{  y \}$.
%
\end{dem}

\begin{lem}[Urysohn]
  Sea $( X, \mathcal{T} )$ e.t.. Entonces, $( X, \mathcal{T} )$ normal $\Leftrightarrow \forall C_{1}, C_{2}$ cerrado disjunto, $\exists f: ( X, \mathcal{T} ) \to [0, 1]$ continua tal que $f(C_{1}) = \{  0 \}, f(C_{2}) = \{ 1 \}$.
\end{lem}

\begin{dem}
  \begin{enumerate}[label=(\roman*)]
    \item [($\Rightarrow$)] Sea 
      \[ 
        J = \bigcup_{n \in \mathbb{N} \cup \{ 0 \}} J_{n}, \; J_{n} = \{ \frac{k}{2^{n}} : k \in \{  0,1, \cdots , 2^{n} \} \}
      \] 
      Entonces, $\forall r \in J, \exists M_{r} \subset X$ tal que
      \begin{enumerate}
        \item $M_{0} = C_{1}, M_{1} = X \setminus C_{2}$
        \item $\forall r, r' \in J, r < r' \Rightarrow \overline{M}_{r} \subset \mathring{M}_{r}$
      \end{enumerate}

      Hacemos la demostración por inducción.  \\

      Sea $n = 0$ $\Rightarrow J_{0} = \{ 0, 1 \} \Rightarrow M_{0} = C_{1}, M_{1} = X \setminus C_{2} \Rightarrow \overline{M_{0}} = \overline{C_{1}} = C_{1} \subset X \setminus C_{2} = M_{1} = \mathring{M}_{1}$. \\

      Suponemos que es cierto para $ m = p$. Veamos que se cumple para $m = p + 1$. \\

      Si $m = p+1$, entonces $\forall r \in J, r = \frac{k}{2^{p+1}}$. Distinguimos $k$ par e impar.

      \begin{itemize}
        \item Si $k$ par, entonces $k = 2k', k' \in \mathbb{Z}^{+} \Rightarrow r = \frac{2k'}{2^{p+1}} = \frac{k'}{2^{p}} \in J_{p} \Rightarrow \exists U_{r} $ tal que cumple $(a)$ y $(b)$.
        \item Si $k$ es impar $\Rightarrow s = \frac{k - 1}{2^{p+1}}, t = \frac{k + 1}{2^{p+1}} \in J_{p} \Rightarrow \exists M_{s}, M_{t}$ tal que cumplen $(a)$ y $(b)$ dado que $ s < t , \overline{M}_{s} \subset \mathring{M}_{t}$ y como $( X, \mathcal{T} )$ es normal $\Rightarrow \exists M_{r} \in \mathcal{T}: \overline{M_{s}} \subset M_{r} \subset \overline{M}_{r} \subset \mathring{M}_{t}$.
      \end{itemize}

      Sea $f: X \to [0,1]$ tal que
      \[ 
        f(x) =
        \begin{cases}
          \inf \{ r \in J: x \in \overline{M}_{r} \}, \text{ si } x \not \in C_{2} \; (\Leftrightarrow x \in M_{1}) \\
          1, \text{ si } x \in C_{2}
        \end{cases} 
      \] 

      entonces, $f(C_{2}) = \{ 1 \}$ por definición. Y $\forall x \in C_{1} = M_{0} = \overline{M}_{0}$ cerrado $\Rightarrow f(x) = 0 \Rightarrow f(C_{1}) = \{ 0 \}$ \\

      Veamos que $f$ es continua

      \begin{itemize}
        \item \underline{ Si $0 < f(x) < 1$ }, $J$ denso en $[0,1] \Rightarrow \forall z \in [0,1], \forall \delta > 0, \exists m_{0} \in \mathbb{N} : \frac{1}{2^{m_{0}}} < \delta \Rightarrow \frac{k_{0}}{2^{m_{0}}} \in (z - \delta, z + \delta)$. Entonces, $\exists t, s \in J : f(x_{0}) - \epsilon < t < f(x_{0}) < s < f(x_{0}) + \epsilon$. \\
          \begin{itemize}
            \item Si $ t < f(x_{0}) \Rightarrow x_{0} \not \in \overline{M}_{t} (\text{ si } x_{0} \in \overline{M}_{t} \Rightarrow \forall j \in J: t < j, x_{0} \in \mathring{M}_{j} \subset \overline{M}_{j} \rightarrow f(x_{0}) \leq t )$.
            \item Si $t(x_{0}) < s \Rightarrow x_{0} \in \mathring{M}_{s} \; (f(x_{0}) = \inf \{  r \in J : x_{0} \in \overline{M}_{r} \} < s \Rightarrow \exists j \in J : j < s, x_{0} \in \overline{M}_{j} \subset \mathring{M}_{s})$
          \end{itemize}

      $\Rightarrow x \in (X \setminus \overline{M}_{t} \cap \mathring{M}_{s}) \in \mathcal{T}$ donde $X \setminus \overline{M}_{t} = V^{x_{0}}$. \\

      $x \in V^{x_{0}} \Rightarrow$
      \begin{itemize}
        \item Si $ x \not \in \overline{M}_{t} \Rightarrow f(x) \geq t \; (\text{ Si no }, f(x)<t, f(x) = \inf \{  r \in J : x \in \overline{M}_{r} \} \Rightarrow \exists j \in J: j< t, x \in \overline{M}_{j} \subset \mathring{M}_{t} \subset \overline{M}_{t} \text{ absurdo } )$ 
        \item Si $x \in \mathring{M}_{s} \subset \overline{M}_{s} \Rightarrow f(x) \leq s$
      \end{itemize}

      Entonces, $f(V^{x_{0}}) \subset [t, s] \subset (f(x_{0}) - \epsilon, f(x_{0}) + \epsilon)$

    \item \underline{ Si $f(x_{0}) = 1$}, $ \forall \epsilon >0, \exists t \in J :  f(x_{0}) - \epsilon = 1- \epsilon < t < f(x_{0}) = 1$. Entonces, $t < f(x_{0}) \rightarrow x_{0} \not \in \overline{M}_{t} \Rightarrow x_{0} \in X \setminus \overline{M}_{t} = V^{x_{0}}$. Por tanto, $\forall x \in V^{x_{0}} \Rightarrow x \not \in \overline{M}_{t} \Rightarrow f(x) \geq t$. Por tanto , $f(V^{x_{0}}) \subset [t, 1] \subset (f(x_{0}) - \epsilon, f(x_{0}) + \epsilon) = (1- \epsilon, 1)$.
    \item \underline{ Si $f(x_{0})$ = 1}, entonces $\forall \epsilon > 0, \exists s \in J: 0 = f(x_{0}) < s < f(x_{0}) + \epsilon = \epsilon$ donde $f(x_{0}) < s \Rightarrow x_{0} \in \mathring{M}_{s} = V^{x_{0}} \in \mathcal{T}$. Por tanto, $\forall x \in V^{x_{0}}, x \in \mathring{M}_{s} \subset \overline{M} \subset_{s} \Rightarrow f(x) \leq s \Rightarrow f(V^{x_{0}}) \subset [0, s] \subset [0, \epsilon] = [f(x_{0}), f(x_{0})+ \epsilon)$.
      \end{itemize}

    \item [($\Leftarrow$)] $\forall C_{1}, C_{2}$ cerradps disjuntos.
      \begin{itemize}
        \item Si $C_! = \emptyset$. Tomamos $ M_{1} = \emptyset, M_{2} = X$,
        \item Si $C_{1}, C_{2} \neq \emptyset \Rightarrow \exists f: ( X, \mathcal{T} ) \to [0,1]$ con tinua tal que $f(C_{1}) = \{  0 \}, f(C_{2}) = \{ 1 \} \Rightarrow f^{-1}([0, \frac{1}{2})), f^{-1}((\frac{1}{2},1]) \in \mathcal{T}$ disjuntos, donde $C_{1} \subset f^{-1}([0, \frac{1}{2})),  C_{2} \subset f^{-1}((\frac{1}{2},1]) \in \mathcal{T}$.
      \end{itemize}
  \end{enumerate}
\end{dem}

\begin{cor}
  $T_{4}$ es más fuerte que $T_{3a}$, $T_{4} \Rightarrow T_{3a}$.
\end{cor}

\begin{obs}
  Metrizable $\Rightarrow T_{4}, T_{3a}, T_{3}, T_{2}, T_{1}, T_{0}$.
\end{obs}

\begin{obs}
  $T_{3} \not \Rightarrow T_{4}$, $T_{3a}$ es multiplicativa y $T_{4}$ no.
\end{obs}
