\part{Curvas}
\chapter{Estudio Local}
\section{Curvas Parametrizadas}

\begin{defn}[Curva]
  Una curva en $\mathbb{R}^{3}$ es una función diferenciarle $\alpha: I \subset \mathbb{R} \to \mathbb{R}^{3}$.
\end{defn}

\begin{defn}[Vector tangente]
  Sea $\alpha: I \to \mathbb{R}^{3}$ una curva en $\mathbb{R}^{3}$ con $\alpha = (\alpha_{1}, \alpha^{2}, \alpha^{3})$. Entonces, $\forall t \in I$
  \[ 
    \alpha'(t) = \Big ( \frac{d{\alpha_{1}}}{d{t}}(t),\frac{d{\alpha_{2}}}{d{t}}(t),\frac{d{\alpha_{3}}}{d{t}}(t) \Big ).
  \] 
  \[ 
    = \lim_{h \to 0} \frac{\alpha(t+ h) - \alpha(t)}{h} 
  \] 
\end{defn}

\begin{obs}
  El vector tangente también se llama vector velocidad
\end{obs}

\begin{defn}[Reparametrización]
  Sea $\alpha: I \to \mathbb{R}^{3}$ una curva, $h: J \to I$ una función diferenciable. Entonces, la función $\beta: J \to \mathbb{R}^{3}$
  \[ 
    \beta(t) = \alpha(h(t)) 
  \] 
  es una reparametrización de $\alpha$ por $h$.
\end{defn}

\begin{ejm}
  Sea $\alpha(t) = \big ( t , t \sqrt{t}, 1-t \big )$ en $I = (0,4)$, $h(s) = s^{2}$ en $J = ( 0,2 )$. Entonces, la curva reparametrizada es $\beta(s) = \alpha(h(s)) = \alpha(s^{2}) = ( s, s^{3}, 1 -s^{2} )$.
\end{ejm}

\begin{lem}
  Si $\beta$ es una reparametrización de $\alpha$ por $h$, entonces
  \[ 
    \beta'(t) = h'(t)\cdot \alpha'(h(t)) 
  \] 
\end{lem}

\section{Curvas Regulares}

\begin{defn}[Curva Regular]
  Sea $\alpha: I \to \mathbb{R}^{3}$ una curva parametrizada. Entonces, si $\alpha'(t) \neq 0. \forall t \in I$ decimos que es regular.
\end{defn}


\begin{defn}[Longitud de Arco]
  Sea $\alpha : I \to \mathbb{R}^{3}, t_{0} \in I$. Definimos la función longitud de arco desde $t_{0}$ como $s: I \to \mathbb{R}$ donde
  \[ 
    s(t) = \int_{t_{0}}^{t}  ||\alpha ' (u)|| du. 
  \] 
\end{defn}

\begin{defn}[Curva Parametriza por Longitud de Arco]
  Sea $\alpha: I \to \mathbb{R}^{3}$ una curva diferenciable. Entonces, si $||\alpha'(t)|| = 1, \forall t \in I$ decimos que la curva está parametrizada por longitud de arco.
\end{defn}

\begin{theo}
  Sea $\alpha: I \to \mathbb{R}^{3}$ una curva regular. Entonces, $\exists \beta: J \to \mathbb{R}^{3}$ tal que $||\beta'(s)||=1, \forall s \in J$, es decir, $\beta$ tiene velocidad unitaria.
\end{theo}

\begin{obs}
  Una reparametrización $\alpha(h)$ preserva la orientación si $h' \geq 0$ y la invierte si $h' \leq 0$.
\end{obs}

\begin{obs}
  Por definición, una curva regular parametrizada por arco siempre conserva la orientación.
\end{obs}

%\begin{ejm}
%  Sea $\alpha(t) = ( a \cos(t), a \sen(t), bt)$.
%  \[ 
%    \alpha'(t) = ( -a \sen(t), a \cos(t), b) 
%  \]
%  Se tiene que la velocidad de $\alpha$ es constante dado que 
%  \[
%    ||\alpha'(t)||^{2} = ((\alpha'(t)\cdot \alpha'(t))^{\frac{1}{2}})^{2} = \alpha'(t) \cdot \alpha'(t) =
%  \]
%  \[ 
%    = ( -a \sen(t), a \cos(t), b) \cdot ( -a \sen(t), a \cos(t), b) =
%  \] 
%  \[ 
%     = a^{2} \sen(t)^{2} + a^{2} \cos(t)^{2} + b^{2} = a^{2} + b^{2}.
%  \]
%  que es constante. Sea $c = ||\alpha'|| = (a^{2} + b^{2})^{\frac{1}{2}}$. Entonces, la longitud de arco de $\alpha$ es
%  \[ 
%    s(t) = \int^{t}_{0} c du = ct. 
%  \] 
%  Cuya inversa es $t(s) = \frac{s}{c}$. Ahora, si componemos $\alpha$ con $t$ obtenemos un reparametrización de $\alpha$, con longitud de arco unitaria
%  \[ 
%    \beta(s) = \alpha(\frac{s}{c}) = \Big ( a \cos(\frac{s}{c}), a \sen(\frac{s}{c}), \frac{bs}{c} \Big ) .
%  \] 
%\end{ejm}

\section{Producto Vectorial}

\begin{defn}[Producto Vectorial]
  Sean $u, v \in \mathbb{R}^{3}$. El producto vectorial de $u, v$ es
  \[ 
    u \times v = 
    \begin{pmatrix}
       \vec{i} & \vec{j} & \vec{k}\\
       u_{1} & u_{2} & u_{3}\\
       v_{1} & v_{2} & v_{3}
    \end{pmatrix}
  \] 
\end{defn}

\begin{prop}[Propiedades Producto vectorial]
  Sean $u, v \in \mathbb{R}^{3}$. Entonces,
  \begin{enumerate}[label=(\roman*)]
    \item $u \times v = -v \times u$.
    \item $u \times v$ es lineal respecto de $u$ y $v$, es decir, para $w \in \mathbb{R}^{3}$ y $a, b \in \mathbb{R}, (au + bw) \times v = au \times v + bw \times v$.
    \item $u \times v = 0 \Leftrightarrow u, v$ son linealmente dependientes.
    \item $(u \times v) \cdot u = 0, (u \times v) \cdot v = 0$.
  \end{enumerate}
\end{prop}

\section{Fórmulas de Frenet}

\begin{defn}[Curvatura]
  Sea $\alpha : I \subset \mathbb{R} \to \mathbb{R}^{3}$ una curva regular p.p.a., $s \in I$. Entonces, $||\alpha''(s)|| = k(s)$ se llama curvatura de $\alpha$ en $s$.
\end{defn}
 
\begin{obs}
  $k(s)$ describe el cambio en la dirección de la curva en un instante.
\end{obs}

%\begin{ejm}
%  Sea $u, v \in \mathbb{R}(3)$, $\alpha(s) = us + v$. Entonces, $k(s) = 0, \forall s \in I$. Reciprocamente, $k = ||\alpha''(s)|| = 0$. Entonces, $ \int ( \int k ds) ds \Rightarrow \alpha(s) = us + v$.
%\end{ejm}

\begin{prop}
  Sea $\alpha  : I \subset \mathbb{R} \to \mathbb{R}^{3}$ una curva regular p.p.a.. Entonces, $\alpha''(s) \perp \alpha'(s), \forall s \in I$.
\end{prop}

\begin{dem}
  $||\alpha'(s)|| = 1, \forall s \in I \Rightarrow \alpha'(s) \cdot \alpha'(s) = 1 \Rightarrow 2 \alpha''(s) \cdot \alpha'(s) = 0 \Rightarrow \alpha''(s) \perp \alpha'(s), \forall s \in I$.
\end{dem}

\begin{prop}
  La curvatura se mantiene invariante ante un cambio de orientación. 
\end{prop}

\begin{dem}
  $\beta(-s) = \alpha(s) \Rightarrow \beta'(s) = -\alpha'(s) \Rightarrow \beta''(-s) = \alpha''(s) = k(s)$. 
\end{dem}

\begin{defn}[Vector Tangente Unitario]
  Sea $\alpha  : I \subset \mathbb{R} \to \mathbb{R}^{3}$ una curva regular p.p.a.. Entonces,
  \[
    T(s) = \alpha'(s) \]
  se llama vector tangente unitario a $\alpha$ en $s$.
\end{defn}

\begin{obs}
  $k(s) = ||T'(s)||$.
\end{obs}

\begin{note}
  Observamos que $\forall s \in I: k(s) > 0, \ k(s) = ||\alpha''(s)|| \Rightarrow \alpha''(s) = k(s) N(s)$ donde $N(s)$ es un vector unitario en la dirección de $\alpha''(s)$. Además, $\alpha''(s) \perp \alpha'(s) \Rightarrow N(s)$ es normal a $\alpha(s)$.
\end{note}

\begin{defn}[Vector Normal]
  Sea $\alpha  : I \subset \mathbb{R} \to \mathbb{R}^{3}$ curva regular p.p.a.. Entonces, 
  \[ 
    N(s) = \frac{T'(s)}{k(s)} 
  \] 
  se llama vector normal a $\alpha$ en $s$.
\end{defn}

\begin{obs}
  El vector normal $N$ es perpendicular al vector tangente unitario $T$ y normal a la curva $\alpha$ en $s$. Esto es, $\alpha'(s) \cdot \alpha''(s) = T(s) \cdot k(s)N(s) = 0$
\end{obs}

\begin{defn}[Plano Oscilador]
  Sea $\alpha  : I \subset \mathbb{R} \to \mathbb{R}^{3}$. Entonces, $T(s),N(s)$ determinan un plano en $\mathbb{R}^{3}$ y lo llamamos plano oscilador.
\end{defn}

\begin{obs}
  También se llama Referencia móvil de Frenet para curvas planas.
\end{obs}

\begin{defn}[Vector Binormal]
  Sea $\alpha  : I \subset \mathbb{R} \to \mathbb{R}^{3}$ una curva regular p.p.a.. Entonces, $B(s) = T(s) \times N(s)$ es el vector normal al plano oscilador en $s$ y se dice vector binormal en $s$.
\end{defn}

\begin{obs}
  $||B'(s)||$ mide la tasa de cambio del plano oscilador, es deicr, la rapidez con la que la curva se aleja del plano oscilador en $s$.
\end{obs}

\begin{note}
  $B' = T' \times N' + T \times N' = T \times N' \Rightarrow B'$ es normal a $T$ y $B'$ es paralelo a $N$. Entonces, escribimos $B' = \tau N$ para alguna función $\tau$.
\end{note}

\begin{defn}[Torsión]
  Sea $\alpha  : I \subset \mathbb{R} \to \mathbb{R}^{3}$ una curva p.p.a. tal que $\alpha''(s) \neq 0, s \in I$. Entonces, decimos que
  \[
    \tau(s) = \frac{B'(s)}{N(s)}
  \]
  es la torsión de $\alpha$ en $s$.
\end{defn}

\begin{obs}
  Si cambia la orientación entonces el signo del vector binormal cambia dado que $B = T \times N$. Por tanto, $B'(s)$ y la torsión se mantienen invariantes. 
\end{obs}

\begin{defn}[Tiedro de Frenet]
  Sea $\alpha  : I \subset \mathbb{R} \to \mathbb{R}^{3}$ una curva regular p.p.a. tal que $k>0$. Entonces, para cada valor $s \in I$, $\exists T(s), N(s), B(s)$ vectores unitarios mutuamente ortogonales y los llamamos el tiedro de Frenet en $\alpha$. Estos vectores vienen dados de la siguiente forma
  \[ 
    T(s) = \alpha'(s) \ \text{ vector tangente } ,
  \] 
  \[ 
    k(s) = ||T'(s)||  \text{ curvatura } ,
  \] 
  \[ 
    N(s) = \frac{1}{k(s)}T'(s)  \text{ vector normal } ,
  \]
  \[ 
    B = T \times N  \text{ vector binormal } ,
  \] 
  \[ 
    \tau(s) = \frac{B'(s)}{N(s)}  \text{ torsión } 
  \] 
  donde $T \cdot T = N \cdot N = B \cdot B = 1$ y cualquier otro producto escalar es $0$.
\end{defn}

DIBUJO

\begin{defn}[Fórmulas de Frenet]
  Sea $\alpha  : I \subset \mathbb{R} \to \mathbb{R}^{3}$ curva regualar p.p.a con $k>0$ y torsión $\tau$. Entonces, 
  \[ 
    T' = kN, 
  \] 
  \[ 
    N' = -kT + \tau B,
  \] 
  \[ 
    B' = -\tau N,
  \] 
\end{defn}

\begin{prop}
  $\tau = 0$ si y solo si $\alpha$ es una curva en el plano.
\end{prop}

\begin{dem}
  \begin{enumerate}[label=(\roman*)]
    \item [($\Rightarrow$)] Sea $\alpha  : I \subset \mathbb{R} \to \mathbb{R}^{3}$ una curva plana p.p.a.. Entonces, $\exists p \in \mathbb{R}, q \in \mathbb{R}^{3}$ tal que $(\alpha(s) - p) \cdot q = 0, \forall s \in I$. Derivando,
  \[ 
    \alpha'(s) \cdot q = \alpha''(s) \cdot q = 0, \; \forall s \in I.
  \] 
  Por tanto, $q$ es ortogonal a $T$ y $N \Rightarrow B = \frac{q}{||q||} \Rightarrow B' = 0 \Rightarrow \tau = 0$.
    \item [($\Leftarrow$)] Sea $\tau = 0 \Rightarrow B' = 0 \Rightarrow B' \ || \ B$. Queremos ver que $\alpha$ es ortogonal a $B$ en $0$. Sea 
      \[ 
        f(s) = (\alpha(s) - \alpha(0)) \cdot B, \forall s \in I .
      \] 
      Entonces,
      \[
        \frac{\partial{f}}{\partial{s}} = \alpha' \cdot B = T \cdot B = 0
      \]
      donde $f(0) = 0 \Rightarrow (\alpha(s) - \alpha(0)) \cdot B = 0, \; s \in I$. Por tanto, $\alpha$ permanece en el plano ortogonal a $B$.
  \end{enumerate}
\end{dem}

\begin{prop}
  Sea $\alpha  : I \subset \mathbb{R} \to \mathbb{R}^{3}$ una curva regular p.p.a. con curvatura constante $k>0$ y $\tau = 0$. Entonces $\alpha$ es parte de un circulo de radio $\frac{1}{k}$.
\end{prop}

\begin{dem}
  $\tau = 0 \Rightarrow \alpha$ es una curva en plano. Sea $\gamma = \alpha + \frac{1}{k}N$ entonces,
  \[ 
    \gamma' = \alpha' + \frac{1}{k_{\alpha}}N'_{\alpha} = T_{\alpha} - \frac{1}{k_{\alpha}}k_{\alpha}T_{\alpha} = 0.
  \] 
  Como $T_{\gamma} = 0 \Rightarrow k_{\gamma} =0 \Rightarrow \gamma$ es una recta horizontal. Sea $\gamma = c \in \mathbb{R}$
  \[ 
    \gamma(s) = \alpha(s) + \frac{1}{k_{\alpha}(s)}N(s) = c, \; \forall s  \in I
  \] 
  \[ 
    \Rightarrow d(c, \alpha(s)) = ||c - \alpha(s)|| = ||\frac{1}{k}N(s)|| = \frac{1}{k}.
  \] 
  Luego, $\alpha$ es una curva que en todo punto se mantiene a distancia $\frac{1}{k}$ de un punto fijo $c$, el centro de la circunferencia.
\end{dem}

\section{Curvas Arbitrarias}

\begin{prop}
  Sea $\alpha  : I \subset \mathbb{R} \to \mathbb{R}^{3}$ una curva regular con $k>0$ y $\beta: J \to \mathbb{R}^{3}$ su reparametrización por arco tal que $\beta(t) = \alpha(s(t))$ donde $s(t)$ es la longitud de arco. Entonces, 
  \[ 
    T' = kvN 
  \] 
  \[ 
    N' = -kvT + \tau v B 
  \]
  \[ 
    B' = -\tau v N 
  \]
\end{prop}

\begin{dem}
  $\frac{d{T(s(t))}}{d{t}} = T'(s(t)) \cdot s'(t) = k(s(t)) N(s(t)) v(t) = k(s) N(s) v$.
\end{dem}

\begin{prop}
  Sea $\alpha  : I \subset \mathbb{R} \to \mathbb{R}^{3}$ una curva regular con $k>0$ y $\beta: J \to \mathbb{R}^{3}$ su reparametrización por arco tal que $\beta(t) = \alpha(s(t))$ donde $s(t)$ es la longitud de arco. Entonces, 
  \[ 
    \frac{d{\alpha}}{d{t}} = \alpha'(s) \frac{d{s}}{d{t}} = v T(s),
  \] 
  \[ 
    \frac{d{\alpha'}}{d{t}} = \frac{d{v}}{d{t}}T + vT' = v'T(s) + kv^{2}N 
  \] 
  son la velocidad y aceleración de $\alpha$ en $s(t)$.
\end{prop}

DIBUJO

\begin{theo}
  Sea $\alpha  : I \subset \mathbb{R} \to \mathbb{R}^{3}$ una curva regular. Entonces,
  \[ 
    T = \frac{\alpha'}{||\alpha'||}, \ k = \frac{||\alpha' \times \alpha''||}{||\alpha'||^{3}},
  \] 
  \[ 
    N = B \times T , \ B = \frac{\alpha' \times \alpha''}{||\alpha' \times \alpha''||},
  \] 
  \[ 
    \tau = (\alpha' \times \alpha'') \cdot \frac{\alpha'''}{||\alpha' \times \alpha'''||^{2}} .
  \] 
\end{theo}

\begin{defn}[Hélice Cilíndrica]
  Sea $\alpha  : I \subset \mathbb{R} \to \mathbb{R}^{3}$ una curva regular tal que $ T(t) \cdot u = \cos(\varphi), \forall t \in I$. Entonces, $\alpha$ es una hélice cilíndrica.
\end{defn}

\begin{theo}
  Sea $\alpha  : I \subset \mathbb{R} \to \mathbb{R}^{3}$ curva regula con $k>0$. Entonces, $\alpha$ es una hélice cilíndrica si y solo si $\frac{\tau}{k}$ es constante.
\end{theo}

\begin{dem}
  \begin{enumerate}[label=(\roman*)]
    \item []
    \item [($\Rightarrow$)] Sea $\alpha  : I \subset \mathbb{R} \to \mathbb{R}^{3}$ curva regular p.p.a con $k>0$. Entonces, si $\alpha$ es una hélice cilíndrica $T(t) \cdot u = \cos(\varphi), \; \forall t \in I \Rightarrow$
      \[ 
        0 = (T \cdot u)' = T' \cdot u = kN \cdot u
      \] 
      donde $k>0 \Rightarrow N \cdot u = 0$. Por tanto, $\forall t \in I,$ $u$ está en el plano determinado por $T(t)$ y $B(t)$. Es decir,
      \[ 
        u = \cos(\varphi)T + \sen(\varphi)B .
      \] 
      Usando las fórmulas de Frenet
      \[ 
        0 = (k \cos(\varphi) + \tau \sen(\varphi))N 
      \] 
      \[ 
        \Rightarrow \frac{\tau}{k} = \frac{\cos(\varphi)}{sin(\varphi)}.
      \] 
    \item [($\Leftarrow$)] Si $\frac{\tau(t)}{k(t)} = \cos(\varphi), \forall t \in I$. Entonces, eligiendo $\cot(\varphi) = \frac{\tau}{k}$, si
      \[ 
        U = \cos(\varphi)T + \sen(\varphi)B 
      \] 
      tenemos que
      \[ 
        U' = (k \cos(\varphi) - \tau \sen(\varphi))N = 0 
      \] 
      determina un vector unitario $u$ tal que $T \cdot u = \cos(\varphi) \Rightarrow \alpha$ es una hélice cilíndrica.
  \end{enumerate}
\end{dem}

\begin{theo}[Fundamental de la Teoría Local de Curvas]
  Sean $k,\tau: I \subset \mathbb{R} \to \mathbb{R}$ funciones diferenciables con $k(s) > 0, \tau(s)$. Entonces, $\exists \alpha  : I \subset \mathbb{R} \to \mathbb{R}^{3}$ curva tal que $s$ es la longitud de arco, $k(s)$ es la curvatura, y $\tau(s)$ es la torsión de $\alpha$.

  Además, cualquier otra curva $\overline{\alpha}$ difiere de $\alpha$ por un movimiento rígido, es decir, $\exists \gamma: I \to \mathbb{R}³$ aplicación lineal ortogonal con $\det \gamma > 0 $ y $c \in \mathbb{R}^{3}: \overline{\alpha} = (\overline{\alpha} \circ \gamma) + c$.
\end{theo}

\begin{dem}
  content
\end{dem}

\chapter{Estudio Global}
