\chapter{Orientabilidad}

\section{Campos}

\begin{defn}[Campo]
  Sea $S \subset \mathbb{R}^{3}$ superficie. Un espacio vectorial diferenciable en $S$ es una aplicación diferenciable $V : S \to \mathbb{R}^{3}$. 
  \begin{itemize}
    \item Si $V(p) \in T_{p}(S), \forall p \in S$, decimos que $V$ es un campo tangente a $S$.
    \item Si $V(p) \perp T_{p}(S), \forall p \in S$, decimos que $V$ es un campo normal a $S$. 
  \end{itemize}
  Además, si $| V(p) | = 1, \forall p \in S$, decimos que $V$ es el campo unitario.
\end{defn}

\begin{obs}
  $\forall p \in S$, hay dos vectores unitarios de $\mathbb{R}^{3}$ perpendiculares al plano tangente $T_{p}(S)$.
\end{obs}

\begin{prop}
  Sea $S \subset \mathbb{R}^{3}$ superficie, $X : U \to \mathbb{R}^{3}$ parametrización de $S$. Entonces, $\exists N \subset V = X(U)$ campo normal unitario.
\end{prop}

\begin{dem}
  content
\end{dem}

\begin{defn}[Supercicies Orientable]
  Sea $S \subset \mathbb{R}^{3}$ superficie. Decimos que $S$ es orientable si admite un campo normal unitario global $N : S \to \mathbb{S}^{2} \subset \mathbb{R}^{3}$. Decimos que $N$ es una orientación de $S$. Cada superficie $S$ tiene dos orientaciones. Fijada $N$ decimos que $S$ está orientada.
\end{defn}

\begin{obs}
  el campo vectorial unitario global $N$ se conoce como aplicación de Gauss.
\end{obs}

\begin{prop}
  Sea $S \subset \mathbb{R}^{3}$ superficie. Entoces, $S$ es orientable $\Leftrightarrow \exists N : S \to \mathbb{R}^{3}$ en $S$.
\end{prop}

\begin{dem}
  content
\end{dem}
