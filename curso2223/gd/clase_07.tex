\chapter{Curvatura}

\section{Resumen Orientación}

\begin{note}
  Dada una parametrización $X : U \subset \mathbb{R}^{2} \to S$ de una superficie regular $S$ en un punto $p \in S$, podemos elegir un vector normal unitario en cada punto de $X(U)$
  \[ 
    N(q) = \frac{X_{u} \times X_{v}}{| X_{u} \times X_{v} |} (q), \quad q \in X(U).
  \] 
  Entonces, tenemos una aplicación diferencial $N : X(U) \to \mathbb{R}^{3}$ que asocia a cada $ q \in X(U)$un vector normal unitaro. En general, si $V \subset S$ es un abierto de $S$ y $N : V \to \mathbb{R}^{3}$ es una aplicación diferencial que asocia $\forall q \in V$ un vector normal unitario en $q$, entonces decimos que $N$ es un campo normal unitaro diferenciale en $V$.
\end{note}

\begin{obs}
  No todas las superficies admiten un campo normal unitario diferenciable. Por ejemplo, la band de Mobius.
\end{obs}

\begin{defn}
  Decimos que $S$ superficie es orientable si admite un campo normal unitario diferenciable en todo $S$. A este campo lo llamamos orientación de $S$.
\end{defn}

\begin{obs}
  Toda superficie cubierta por un solo sistema de coordenadas es trivialmente orientable. Por ejemplo, la superficies representadas por grafos de funciones diferenciables.
\end{obs}

\begin{prop}
  Una orientación $N$ en $S$ induce una orientación en el espacio tangenete $T_{p}(S)$. Sea $p \in S$. Definimos $\{ v, w \} \subset T_{p}(S)$ como base positiva si $(v \times w) \cdot N > 0$. Entonces, el conjunto de todas las bases positivas de $T_{p}(S)$ es una orientación para $T_{p}(S)$.
\end{prop}

\begin{nota}
  Una superficie $S$ orientable tiene orientación $N$, donde $N$ es un campo normal unitario diferenciable.
\end{nota}

\section{Segunda Forma Fundamental}

\begin{defn}[Aplicación de Gauss]
  Sea $S \subset \mathbb{R}^{3}$ una superficie con orientación $N$. La aplicación $N : S \to \mathbb{R}^{3}$ toma valores en la esfera unidad
  \[ 
    \mathbb{S}^{2} = \{ (x, y, z) \in \mathbb{R}^{3} : x^{2} + y^{2} + z^{2} = 1 \} 
  \] 
  La aplicación $N : S \to \mathbb{S}^{2}$ se llama aplicación de Gauss de $S$.
\end{defn}

\begin{obs}
  La aplicación de Gauss es diferenciable tal que $(d N)_{p} : T_{p}(S) \to T_{N(p)}(\mathbb{S}^{2}) = T_{p}(S)$ es una aplicación lineal. 
\end{obs}

\begin{prop}
  La diferencial $(d N)_{p} : T_{p}(S) \to T_{p}(S)$ de la aplicación de Gauss es una aplicación lineal auto-adjunta, es decir,
  \[ 
    \langle (d N)_{p}(v){ , }w \rangle = \langle w{ , }(d N)_{p}(v) \rangle, \quad \forall v, w \in T_{p}(S)
  \] 
\end{prop}

\begin{dem}
  content
\end{dem}

\begin{defn}[Segunda Forma Fundamental]
  La forma bilineal $II_{p} : T_{p}(S) \times T_{p}(S) \to \mathbb{R}$ definida por 
  \[
    II_{p}(v,w) = - \langle (d N)_{p}(v){ , }w \rangle, \quad v, w \in T_{p}(S)
  \] 
  se llama segunda forma fundamental de $S$ en $p$.
\end{defn}

\begin{prop}
  La segunda fomra fundamental es simétrica, es decir,
  \[ 
    II_{p}(v,w) = II_{p}(w,v), \quad \forall v, w \in T_{p}(S)
  \] 
\end{prop}

\begin{dem}
  content
\end{dem}

\section{Meusnier}

\begin{defn}[Sección Normal]
  La sección normal en $p \in S$ por $v \in T_{p}(S)$ es el plano formado por $v$ y $N(p)$ en $p$.
\end{defn}

\begin{defn}[Curvatura Normal]
  Sea $P$ la sección normal en $p \in S$ por $v \in T_{p}(S)$, $\alpha$ la curva forma por la intersección de $P$ y $S$. Entonces, la curvatura de $\alpha$ se llama curvatura normal y se denota $k_{n}$.
\end{defn}

\begin{prop}
  Sea $P$ la sección normal en $p \in S$ por $v \in T_{p}(S)$. Entonces,
  \[ 
    k_{n} = \langle \alpha''(0){ , }N(p) \rangle = II_{p}(v, v)
  \] 
  es la curvatura normal.
\end{prop}

\begin{dem}
  content
\end{dem}

\begin{theo}[Meusnier]
  Sea $P$ la sección normal en $p$ por $v \in T_{p}(S)$ y sea $P_{\theta}$ el plano que contiene $v$ y forma un ángulo $\theta$ con $P$. Entonces, La curva $c_{\theta}$ forma por la intersección de $S$ y $P_{\theta}$ tiene curvatura $k_{\theta}$ y viene dada por
  \[ 
    k_{\theta} \cdot \cos(\theta) = k_{n} 
  \] 
  donde $k_{n}$ es la curvatura normal.
\end{theo}

\begin{dem}
  content
\end{dem}

\begin{prop}
  Todas las curvas en $S$ que tienen la misma tangente en $p$ tienen la misma curvatura normal.
\end{prop}

\section{Euler}

\begin{theo}[Euler]
  Las curvaturas normales $k_{n}$ tienen un mínimo $k_{1}$ en una dirección $e_{1}$ y un máximo $k_{2}$ en una dirección perpendicular $e_{2}$. Sea $v \in T_{p}(S)$ dirección formando un ángulo $\theta$ con $e_{1}$. Entonces,
  \[ 
    k_{n} = k_{1} \cos^{2}(\theta) + k_{2} \sen^{2}(\theta)
  \] 
\end{theo}

\begin{defn}[Curvaturas Principales]
  La curvatura normal máxima $k_{1}$ y la curvatura normal mínima se llaman curvaturas principales en $p$.
\end{defn}

\begin{defn}[Direcciones Principales]
   Las direcciones $e_{1}$ y $e_{2}$ se llaman direcciones principales en $p$.
\end{defn}

\section{Coordenadas Locales}

\begin{note}[Coordenadas Locales]
  Sea $X : U \subset \mathbb{R}^{2} \to S$ parametrización de $S$ en $p \in S$ compatible con una orientación $N : S \to \mathbb{R}^{3}$ entonces,
  \[ 
    N^{X}(u,v) = \frac{X_{u} \times X_{v}}{| X_{u} \times X_{v} |}(u,v)
  \] 
  Sea $\alpha(t) = (X \circ \beta)(t) = X(u(t),v(t))$ con $\alpha(0) = p$. Entonces,
  \[ 
    \alpha' = X_{u} u' + X_{v} v' 
  \] 
  y podemos expresar
  \[ 
    (d N)_{p}(\alpha') = (N \circ X \circ \beta)' = (N^{X} \circ \beta)= N_{u} u' + N_{v} v' .
  \] 
  Ahora, como $N_{u}, N_{v} \in T_{p}(S)$ tenemos que
  \[ 
    N_{u} = a_{11} X_{u} + a_{21} X_{v} 
  \] 
  \[ 
    N_{v} = a_{12} X_{u} + a_{22} X_{v} 
  \] 
  Por tanto, la matriz de $(d N)_{p}(\alpha')$ en la base $\{ X_{u}, X_{v} \}$ es
  \[ 
    \begin{pmatrix}
       a_{11} & a_{12} \\
       a_{21} & a_{22}
    \end{pmatrix}
  \] 
  Por otro lado, la segunda forma fundamental en la base $\{ X_{u}, X_{v} \}$ es 
  \[ 
    II_{p}(\alpha', \alpha') = - \langle (d N)_{p}(\alpha'){ , }\alpha' \rangle
  \] 
  \[ 
    = \langle N_{u}u' + N_{v} v'{ , }X_{u} u' + X_{v} v' \rangle 
  \] 
  \[ 
    = e(u')^{2} + 2 f u' v' + g(v')^{2} 
  \] 
  donde $e$, $f$, $g$ son los coefiecientes de la segunda forma fundamental dados por
  \[ 
    e = \langle N{ , }X_{uu} \rangle, 
  \] 
  \[ 
    f = \langle N{ , }X_{uv} \rangle,
  \] 
  \[ 
    g = \langle N{ , }X_{vv} \rangle.
  \] 
  Resoloviendo el sistema obtenemos los coeficientes $a_{ij}$ y expresamos la curvatura en función de estos
  \[ 
    K = \det((d N)_{p}(\alpha')) = \det(a_{ij}) = \frac{eg - f^{2}}{EG - F^{2}}
  \] 
\end{note}

\section{Puntos Curvatura}

\begin{note}
  Considerando la aplicación $dN_{p}$ en la base $\{ e_{1}, e_{2} \}$ donde $e_{1}$ y $e_{2}$ son la direcciones principales tenemos que
  \[ 
    K = k_{1} k_{2}, \quad H = \frac{k_{1} + k_{2}}{2}
  \] 
\end{note}

\begin{defn}
  Un punto $p \in S$ superficie se llama
  \begin{enumerate}[label=(\roman*)]
    \item Elíptico si $K(p) > 0$,
    \item Hiperbólico si $K(p) < 0$,
    \item Parabólico si $K(p) = 0$ con $dN_{p} \neq 0$,
    \item Plano si $K(p) = 0$.
  \end{enumerate}
\end{defn}

\begin{obs}[Putno Elíptico]
  Si $p \in S$ es elíptico
  \[
    (eg -f^{2}) > 0,
  \]
  entonces $K(p) > 0$ y ambas $k_{1}, k_{2} > 0$ o $k_{1}, k_{2} < 0$. Por tanto, toda curva $\alpha(t_{0}) \subset S$ tal que $\exists s_{0} \in I : \alpha(s_{0}) = p$  tiene sus vectores normales apuntando hacia el mismo lado del plano tangente.
\end{obs}

\begin{obs}[Punto Hiperbólico]
  Si $p \in S$ es hiperbólico
  \[
    (eg -f^{2}) < 0,
  \]
  entonces $K(p) < 0$ y $k_{1} < 0, k_{2} > 0$ ó $k_{1} > 0, k_{2} < 0$. Por tanto, hay curvas en $p$ cuyos vectores normales apuntan hacia ambos lados del plano tangente.
\end{obs}

\begin{obs}[Punto Parabólico]
  Si $p \in S$ es parabólico
  \[
    (eg -f^{2}) = 0, \quad eg = f^{2} \neq 0,
  \]
  entonces $K(p) = 0$ y $k_{1} \neq 0$ o $k_{2} \emptyset 0$.
\end{obs}

\begin{obs}[Punto Plano]
  Si $p \in S$ es plano
  \[ 
  (eg - f^{2}) = 0, \quad eg = f = 0 
  \]
  entonces $k_{1} = 0$ y $k_{2} = 0$.
\end{obs}

\begin{defn}[Punto Umbílico]
  Si $p \in S$, $k_{1} = k_{2}$, entonces $p$ se llama punto umbílico.
\end{defn}

\begin{obs}
  Los puntos planos, es decir, $p \in S$ tal que $k_{1} = k_{2} = 0$ son puntos umbílicos.
\end{obs}

\begin{prop}
  Si todos los puntos de una superficie conexa $S$ son umbílicos, entonces $S$ está contenida en una esfera o en un plano.
\end{prop}

\begin{dem}
  content
\end{dem}

\section{Curvas Asintóticas}

\begin{defn}[Dirección Asintótica]
  Sea $P$ la sección normal en $p \in S$ por $v \in T_{p}(S)$. Si $k_{n} = 0$, entonces $v$ es una dirección asintótica.
\end{defn}

\begin{defn}[Curva Asintótica]
  Sea $\alpha$ curva en $S$ tal que $\alpha'(t)$ es dirección asintótica $\forall t \in I$. Entonces, $\alpha$ es una curva asintótica. 
\end{defn}

\begin{obs}
  $\forall \alpha(t) = X(u(t),v(t))$, entonces $\alpha$ es curva asintótica si y solo si $II_{p}(\alpha'(t)) = 0, \forall t \in I$.
\end{obs}

\begin{lem}
  Sea $p \in S$ elíptico, entonces no hay direcciones asintóricas.
\end{lem}

\begin{lem}
  Sea $p \in S$ hiperbólico, entonces hay exactamente $2$ direcciones asintóticas.
\end{lem}

\begin{obs}
  En cada punto parabólico hay exactamente $1$ dirección asintótica.
\end{obs}

\begin{obs}
  Si $p \in S$ no es umbílico, cada valor de $II_{p}(v)$ se alcanza en exactamente 2 direcciones distintas.
\end{obs}

\begin{obs}
  Si $\alpha : I \to S$ es una recta, entonces es línea asintótica y también es sección normal. 
\end{obs}

\begin{prop}
  Una curva $\alpha$ dada por $\alpha(t) = (X \circ \beta)(t) = (X(u(t),v(t))$ es curva asintótica si y solo si
  \[ 
    e(u')^{2} + 2 f u' v' + g (v')^{2} = 0, \quad t \in I .
  \] 
  Esta ecuación se llama ecuación diferencial de las curvas asitóticas.
\end{prop}

\begin{lem}
  Las curvas coordenadas de una parametrización $X$ en punto hiperbólico $(eg - f^{2} < 0)$ son curvas asíntoticas si y solo si
  \[ 
    e=g=0 
  \] 
\end{lem}

\section{Líneas de Curvatura}

\begin{defn}[Linea de Curvatura]
  Sea $\alpha$ una curva en $S$ tal que $\alpha'(t)$ es una dirección principal $\forall t \in I$. Entonces, $\alpha$ es una línea de curvatura.
\end{defn}

\begin{obs}
  La curva $\alpha$ es línea de curvatura si y solo si $(d N)_{p}(\alpha'(t)) = \lambda(t) \alpha'(t)$.
\end{obs}

\begin{prop}
  Una curva $\alpha$ dada por una parametrización $X$ es una línea de curvatura si y solo si
  \[ 
    (fE - eF)(u')^{2} + (gE - eG)u'v' + (gF - fG)(v')^{2} = 0 .
  \] 
  Esta ecuación se llama ecuación diferencial de las líneas de curvatura.
\end{prop}

\begin{lem}
  Las líneas coordenadas en un entorno de un punto no umbílico son líneas de curvatura si y solo si $F = f = 0$.
\end{lem}

\section{Ecuaciones de Compatibilidad}

Do Carmon Sección 4.3 -> Formulario

\begin{nota}[Símbolos de Christoffel]
  \[ 
    \begin{aligned}
      \begin{cases}
        \Gamma_{11}^{1} E + \Gamma_{11}^{2} F = \langle X_{uu}{ , }X_{u} \rangle = \frac{1}{2} E_{u} \\
        \Gamma_{11}^{1} F + \Gamma_{11}^{2} G = \langle X_{uu}{ , }X_{v} \rangle = F_{u} - \frac{1}{2} E_{v} \\
      \end{cases}
    \end{aligned} 
  \] 
  \[ 
    \begin{aligned}
      \begin{cases}
        \Gamma_{12}^{1} E + \Gamma_{12}^{2} F = \langle X_{uv}{ , }X_{u} \rangle = \frac{1}{2} E_{v} \\
        \Gamma_{11}^{1} F + \Gamma_{11}^{2} G = \langle X_{uv}{ , }X_{v} \rangle = \frac{1}{2}G_{u} \\
      \end{cases}
    \end{aligned} 
  \] 
  \[ 
    \begin{aligned}
      \begin{cases}
        \Gamma_{11}^{1} E + \Gamma_{11}^{2} F = \langle X_{vv}{ , }X_{u} \rangle = F_{v} - \frac{1}{2} G_{u} \\
        \Gamma_{11}^{1} F + \Gamma_{11}^{2} G = \langle X_{vv}{ , }X_{v} \rangle = \frac{1}{2} G_{v} \\
      \end{cases}
    \end{aligned} 
  \] 
\end{nota}


\section{Teorema Egregium}

\begin{theo}(Egregium)
  La curvatura de una superficie es invariante invariante por isometrías locales.
\end{theo}

\begin{dem}
  La curvatura se puede expresar en términos de $E, F, G$. Por tanto, por el teorema de isometrías se tiene que dos superficies isométricas tienen
  \[
    E = \overline{E},  \quad F = \overline{F}, \quad G = \overline{G}
  \]
  Entonces, tienen la misma curvatura.
\end{dem}

\begin{obs}
  Entre el plano y un cilindro existe una isometría local. La curvatura del plano es $0$. Por tanto, el cilindro tiene curvatura $0$.
\end{obs}

\begin{ejm}
  Sea $Z \subset \mathbb{R}^{3}$ un cilindro
  \[ 
    Z = \{ (x,y,z) : x^{2} + y^{2} = a^{2} \} 
  \] 
  Entonces, una isometría local es $f : \mathbb{R}^{2} \to Z$ definida por
  \[
    f(s,t) = (a \cos(\frac{s}{a}), a \sen(\frac{s}{a}), t).
  \] 
  Vemos que $f$ es isometría. Sea $p \in \mathbb{R}^{2}$ y $w_{1}, w_{2} \in T_{p}(\mathbb{R}^{2})$. Entonces,
  \[ 
    \langle w_{1}{ , }w_{2} \rangle = \langle (d f)_{p}(w_{1}){ , }(d f)_{p}(w_{2}) \rangle
  \] 
  dado que
  \[ 
    (d f)_{p}(w) =
    \begin{pmatrix}
      -\sen(s) & \cos(s) & 0 \\
      0 & 0 & 1
    \end{pmatrix} 
  \] ??
\end{ejm}

\begin{theo}[Bonnet]
  Sea $S \subset \mathbb{R}^{3}$ superficie y $X, \overline{X} : U \to \mathbb{R}^{3}$ dos parametrizaciones de $S$ con 
  \[  
    E = \overline{E}, \quad F = \overline{F}, \quad G = \overline{G},
  \] 
  \[ 
    e = \overline{e}, \quad f = \overline{f} , \quad g = \overline{g}.
  \] 
  Entonces, existe $\phi : \mathbb{R}^{3} \to \mathbb{R}^{3}$ movimiento rígido tal que $\phi \circ X = \overline{X}$.
\end{theo}

\section{Curvatura Superficies Compactas}

\begin{defn}[Punto límite]
  Sea $A \subset \mathbb{R}^{3}$. Deciomos que $p \in \mathbb{R}^{3}$ es un punto límite de $A$ si $\forall U^{p}$ entorno de $p$ en $ \mathbb{R}^{3}$, $A \cap setminus U^{p} \setminus \{ p \} \neq \emptyset$.
\end{defn}

\begin{defn}[Conjunto Cerrado]
  $A$ es cerrado si contiene todos sus puntos límite.
\end{defn}

\begin{defn}[Conjunto Acontado]
  $A$ es acotado si está contenido en alguna bola.
\end{defn}

\begin{defn}[Superfice Compacta]
  Sea $S \subset \mathbb{R}^{3}$ superfices. Si $S$ es cerrado y acotado, entonces $S$ es compacto. 
\end{defn}

\begin{prop}
  No hay superficies compactas con curvatura negativa en todos sus puntos.
\end{prop}

\begin{dem}
\end{dem}

\begin{prop}
  Toda superficie compacta $S$ tiene un punto de curvatura positivo.
\end{prop}

\begin{prop}
  Sea $S$ superficie compacta. Sea $S_{r}$ una esfera de radio $r$ y $B_{r}$ una bola compacta de radio $r$ que contiene a $S_{r}$. Sea $ p \in S \cap S_{r}$ y $ S \subset B_{r}$. Entonces, $S$ tiene curvatura positiva en $p$.
\end{prop}

\section{Definición no rigurosa de Curvatura}

\begin{nota}
  Sea $S \subset \mathbb{R}^{3}$ superficie, $p \in S$ la definición no rigurosa de la curvatura en $p$ es
  \[ 
    K(p) = \lim_{A \to p} \frac{\text{área } N(A)}{\text{área } A}
  \] 
  donde $A$ es un entorno de $p$ y $N$ es el campo norma inducido por una parametrización $X : U \subset \mathbb{R}^{2} \to S$.
\end{nota}

\chapter{Geodésicas}
%
%\section{Campos Vectoriales}
%
%\begin{defn}[Campo Vectorial]
%  Un campo vectorial en un conjunto $U \subset \mathbb{R}^{2}$ es una aplicación que asigna a cada $q \in U$ un vector $w(q) \in \mathbb{R}^{2}$. Es decir,
%  \[ 
%    w : U \to \mathbb{R}^{2} : q \mapsto w(q).
%  \]
%\end{defn}
%
%\begin{defn}[Campo Vectorial Diferenciable]
%  El campo vectorial $w : U \to \mathbb{R}^{2}$ es diferenciable si para $q = (x,y)$ y
%  \[ 
%    w(q) = w(x, y) = (a(x,y), b(x,y)) 
%  \] 
%  se tiene que las funciones $a$ y $b$ son diferenciables en $U$.
%\end{defn}
%
%\begin{obs}
%  En ecuaciones diferenciales decimos que el campo vectorial $w$ determina un sistema de ecuciones diferenciales
%  \[ 
%    \frac{d{x}}{d{t}} = a(x, y) 
%  \] 
%  \[ 
%    \frac{d{y}}{d{t}} = b(x, y) 
%  \] 
%  y un trayectoria de $w$ es una solución del sistema.
%\end{obs}
%
%\begin{theo}
%  Sea $w$ un campo vectorial en un conjunto abierto $U \subset \mathbb{R}^{2}$. Dado $p \in U$, $\exists \alpha : I \to U$ trayectoria de $w$, es decir,
%  \[
%    \alpha'(t) = w(\alpha(t)), \quad t \in I.
%  \]
%  con $\alpha(0) = p$. Además, esta trayectoria es única.
%\end{theo}
%
%\begin{defn}[Campo Vectorial En Superficie]
%  Sea $S \subset \mathbb{R}^{3}$ superficie y $p \in S$. Un campo vectorial en $U \subset S$ con $p \in U$ es una aplicación
%  \[
%    w : U \to T_{p}(S).
%  \]
%  La aplicación es diferenciable en $p$ y dada una parametrización $X(u,v)$ en $p$ se expresa
%  \[ 
%    w(p) = a(u, v) X_{u} + b(u, v) X_{v} 
%  \] 
%  con $a$ y $b$ diferenciables.
%\end{defn}
%
%\begin{obs}
%  La aplicación $w (p) $ no depende de $X$.
%\end{obs}
%
%\begin{ejm}
%  content
%\end{ejm}
%
%\begin{theo}
%  content
%\end{theo}
%
%\begin{cor}
%  Sea $S \subset \mathbb{R}^{3}$ superfice. Entonces, $\forall p \in S$ existe un parametrización $X(u,v)$ en un entorno $ V$ de $p$ tal que las curvas coordenadas $u = u_{1}$ y $ v = v_{1}$ donde $u_{1}, v_{1} \in \mathbb{R}$ se intersecan ortogonalmente $\forall q \in V$. Esta parametrización se llama parametrización ortogonal.
%\end{cor}
%
%\begin{cor}
%  Sea $p \in S$ un punto hiperbólico de $S$. Entonces, es posible parametrizar un entorno de $p$ de manera que la curvas coordenadas de esta parametrización sean curvas asintóticas de $ S$.
%\end{cor}
%
%\begin{ejm}
%  Una parametrización del hiperboloide parabólico $z = x^{2} - y^{2}$ es 
%  \[ 
%    X(u, v) = (u, v, u^{2} - v^{2}).
%  \] 
%  Las curvas asintóticas son solución de 
%  \[ 
%    e(u')^{2} + 2 f u' v' + g(v')^{2} = 0 
%  \] 
%  que se puede factorizar como
%  \[ 
%    (Au' + Bv')(Au' + Dv') = 0
%  \] 
%  donde $A^{2} = e$, $A(B + D) = 2f$, $BD = g$. Este sistema da lugar a las ecuaciones
%  \[
%    \begin{aligned}
%      \begin{cases}
%        Au' + Bv' = 0 \\
%        Au' + Dv' = 0 
%      \end{cases}
%    \end{aligned}
%  \] 
%  En este caso, los coeficientes son
%  \[ 
%    e = \frac{2}{(1 + 4 u ^{2} + 4 v^{2})^{\frac{1}{2}}}, \quad  f =0, \quad g = - \frac{2}{(1 + 4 u ^{2} + 4 v^{2})^{\frac{1}{2}}}.
%  \] 
%  Por tanto, la ecuación de la curvas asintóticas viene dada por
%  \[ 
%    \frac{2}{(1 + 4 u ^{2} + 4 v^{2})^{\frac{1}{2}}}((u')^{2} - (v')^{2}) = 0
%  \] 
%  cuye sistema asociado es
%  \[ 
%    \begin{aligned}
%      \begin{cases}
%        u' + v' = 0 \\
%        u' - v' = 0
%      \end{cases}
%    \end{aligned} 
%  \] 
%  Las líneas integrales vienen dadas por
%  \[ 
%    \begin{aligned}
%      \begin{cases}
%        u' + v' = \cte \\
%        u' - v' = \cte
%      \end{cases}
%    \end{aligned} 
%  \] 
%  Por tanto, para
%  \[ 
%    \overline{u} = u + v, \quad \overline{v} = u - v 
%  \] 
%  obtenemos una nueva parametrización $\overline{X} = X(\overline{u}\overline{vl})$ donde las curvas coordenadas son las curvas asintóticas.
%\end{ejm}
%
%\begin{cor}
%  Sea $p \in S$ un punto no umbílico. Entonces, existe una parametrización para $V$ entorno de $p$ de manera que las curvas de esta parametrización son lineas de curvatura de $S$.
%\end{cor}
%
%\section{Derivada Covariante}
%
%\begin{defn}[Derivada Covariante]
%  Sea $w$ un campo vectorial diferenciable en $U \subset S$ abierto y $ p \in U$. Sea $y \in T_{p}(S)$. Consideramos la curva parametrizada 
%  \[ 
%    \alpha : I \to U 
%  \] 
%  donde $\alpha(0) = p$ y $\alpha'(0) = y$. Sea $w(t), t \in I$ la restricción del campo vectorial $w$ a la curva $\alpha$. Entonces, el vector obtenido por la proyección de 
%  \[ 
%    \frac{d{w}}{d{t}}(0) 
%  \] 
%  en el plano $T_{p}(S)$ se llama derivada covariante del campo vectorial $w$ relativo al vector $y$ y se denota 
%  \[ 
%    \frac{D w}{d t}(0)
%  \] 
%\end{defn}
%
%\begin{obs}
%  La derivada covariante es un concepto de geometría intrínseca y no depende de la elección de curva $\alpha$. Sea $X(u,v)$ parametrización de $S$ en $p$. Sea
%  \[
%    (X \circ \beta) (t) =  X(u(t),v(t)) = \alpha(t)
%  \]
%  y sea
%  \[ 
%    w(t) = a(u(t),v(t)) X_{u} + b(u(t),v(t)) X_{v}
%  \] 
%  \[ 
%    = a(t) X_{u} + b(t) X_{v}
%  \] 
%  Entonces,
%  \[ 
%    \frac{d{w}}{d{t}} = a (X_{uu} u' + X_{uv} v') + b (X_{vu} u' + X_{vv} v') + a' X_{u} + b' X_{v}
%  \] 
%  Entonces, sustituyendo $X_{uu}, X_{uv}, X_{vv}$ tenemos
%  \[ 
%    \frac{D w}{dt} = (a' + \Gamma_{11}^{1}au' + \Gamma_{12}^{1}bu' + \Gamma_{22}^{1}bv' )X_{u}
%  \] 
%  \[ 
%    (b' + \Gamma_{11}^{2}au' + \Gamma_{12}^{2}bu' + \Gamma_{22}^{2}bv' )X_{v} 
%  \] 
%  Por tanto, la derivada covariante depende solo de $y = (u', v')$
%\end{obs}
%
%\begin{defn}[Campo Vectorial Diferenciable]
%  Sea $\alpha : I \to S$ una curva parametrizada en $S$. Un campo vectorial $w$ a lo largo de $\alpha$ es una correspondencia que asigna a cada $t \in I$ un vector
%  \[ 
%     w(t) \in T_{\alpha(t)}(S).
%  \] 
%  El campo vectorial $w$ es diferencialbe en $t_{0} \in I$ si para alguna parametrización $X(u,v)$ e $\alpha(t_{0})$ la componentes $a(t), b(t)$ de $w(t) = a X_{u} + b X_{v}$ son funciones diferenciables de $t$ en $t_{0}$.
%\end{defn}
%
%\begin{ejm}
%  Un campo vectorial diferenciable a lo largo de $\alpha$ es el campo $\alpha'(t)$ de vectores tangentes a $\alpha$.
%\end{ejm}
%
%\begin{defn}
%  Sea un campo vectorial diferenciable $w$ a lo largo de $\alpha : I \to S$. Entonces,
%    \[ 
%    \frac{D w}{dt}(t) = (a' + \Gamma_{11}^{1}au' + \Gamma_{12}^{1}bu' + \Gamma_{22}^{1}bv' )X_{u}
%  \] 
%  \[ 
%    (b' + \Gamma_{11}^{2}au' + \Gamma_{12}^{2}bu' + \Gamma_{22}^{2}bv' )X_{v} 
%  \] 
%  está bien definida y es la derivada covariante de $w$ en $t$.
%\end{defn}
%
%\begin{obs}
%  Cuando dos superficies son tangentes a lo largo de una curva $\alpha$, entonces la derivada de un campo $w$ a lo largo de $\alpha$ es la misma para ambas superficies.
%\end{obs}
%
%\begin{obs}
%  La derivada covariante $\frac{D{\alpha'}}{d{t}}$ de $\alpha'(t)$ es la componente tangencial de la aceleración $\alpha''(t)$. Intuitivamente $\frac{D{\alpha'}}{d{t}}$ es la aceleración del punto $\alpha(t)$ visto desde la superficie $S$.
%\end{obs}
%
%\begin{defn}[Campo Vectorial Paralelo]
%  Un campo vectorial a lo largo de una curva parametrizada $\alpha : I \to S$ es paralelo si $\frac{D w}{d t} = 0, \forall t \in I$.
%\end{defn}
%
%\begin{prop}
%  Sean $w, v$ campos vectoriales paralelos a lo largo de $\alpha : I \to S$. Entonces,
%  \[ 
%    \langle w(t){ , }v(t) \rangle 
%  \] 
%  es constante. En particular, $| w(t) |, | v(t) |$ son constantes y el ángulo entre $v(t)$ y $w(t)$ es constante.
%\end{prop}
%
%\begin{dem}
%  Si $w$ es campo vectorial paralelo a lo largo de $\alpha$, entonces $\frac{d{w}}{d{t}}$ es normal al plano que es tangente a la superficie en $\alpha(t)$, es decir,
%  \[ 
%    \langle v(t){ , }w'(t) \rangle = 0, \quad t \in I
%  \] 
%\end{dem}
%
%\begin{prop}
%  Sea $\alpha : I \to S$ una curva parametrizada en $S$ y $w_{0} \in T_{\alpha(t_{0})}(S), t_{0} \in I$. Entonces, existe un campo vectorial paralelo $w(t)$ a lo largo de $\alpha(t)$, con $w(t_{0} = w_{0}$.
%\end{prop}
%
%\begin{defn}[Transporte Paralelo]
%  Sea $\alpha : I \to S$ una curva parametrizada y $w_{0} \in T_{\alpha(t_{0})}(S), t_{0} \in I$. Se $ w$ un campo vectorial paralelo a lo largo de $\alpha$ con $w(t_{0}) = w_{0}$. El vextor $w(t_{1}), t_{1} \in I$ se llama transporte paralelo de $w_{0}$ a lo largo del punto $t_{1}$.
%\end{defn}
%
%\begin{obs}
%  Si $\alpha$ es una curva regular, entonces el transporte paralelo no depende de de la parametrización de $\alpha(I)$.
%\end{obs}
%
%\begin{ejm}
%  content
%\end{ejm}
%
%\begin{defn}[Geodésica]
%  Sea $\gamma : I \to S$ una curva parametrizada no constante. Si el campo de sus vectores tangentes $\gamma'(t)$ es paralelo a lo largo de $\gamma$ en $t$, es decir,
%  \[ 
%    \frac{D \gamma'(t)}{d t} = 0
%  \]
%  entonces decimos que $\gamma$ es una geodésica. 
%\end{defn}
%
%\begin{obs}
%  $\gamma$ es una geodésica parametrizada si es una geodésica $\forall t \in I$.
%\end{obs}
%
%\begin{obs}[Definición alternativa] 
%  Sea $\alpha : I \to S$ curva regular conexa p.p.a en $S$. Si $\alpha'(s)$ es paralela a el campo vectorial paralelo a lo largo de $\alpha(s)$, entonces $\alpha$ es una geodésica.
%\end{obs}
%
%\begin{ejm}
%  Toda linea recta contenida en la superficie es una geodésica.
%\end{ejm}
%
%\begin{ejm}
%  Si consideramos el cilindro dado por $x^{2} + y^{2} = 1$, entonces la intersección del cilindro con planos perpendiculares a su eje principal son geodésicas. Las líneas rectas del cilindor también son geodésicas. Para ver el resto de geodésicas del cilindro $C$ consideramos la parametrización
%  \[ 
%    \overline{X}(u, v) = (\cos(u), \sen(u), v)
%  \]  
%  en $p \in C$ con $p = \overline{X}(0,0)$. Ahora, una parametrización del plano es
%  \[
%    X = p_{0} + u w_{1} + v w_{2}
%  \] 
%  donde $w_{1}$ y $w_{2}$ son vectores ortonormales que pasan por $p_{0}$. Sea $\phi = X \circ \overline{X}^{-1}$. Como $E = \overline{E}, F = \overline{F}, G = \overline{G}$, entonces $\phi : C \to P$ es una isometría local en un entorno $V$ de $p \in \overline{X}(U) \subset C$.
% 
%  De forma análoga, la aplicación $\phi^{-1} = \overline{X} \circ X^{-1} : P \to C$ es una isometría local del plano al cilindro tal que
%  \[ 
%    \phi^{-1}(X(u(t),v(t))) = \overline{X}(u(t),v(t))
%  \]
%  Ahora, las geodésicas son invariantes por isometrías. En concreto, las geodésicas del plano son líneas rectas. Por tanto, para una curva p.p.a dada por $\alpha(s) = \overline{X}(u(s), v(s))$ que no es un círculo o una recta, es una geodésica si y solo si es de la forma 
%  \[ 
%    (\cos(as), \sen(as), b s) 
%  \] 
%  que es una hélice.
%
%\end{ejm}
