\chapter{Primera Forma Fundamental}

\section{Primera Forma Fundamental}

\begin{prop}
  Sea $S \subset \mathbb{R}^{3}$ superficie. Entonces, el producto interior natural de $\mathbb{R}^{3}$ induce en cada plano tangente $T_{p}(S)$ un producto interior $\langle  { , } \rangle_{p}$, cuya forma quadrática es $I_{p} : T_{p}(S) \to \mathbb{R}$ definida por
  \[ 
    I_{p}(w) = \langle w{ , }w \rangle_{p} = | w |^{w} \geq 0.
  \] 
\end{prop}

\begin{defn}
  Sea $S \subset \mathbb{R}^{3}$ superficie. Decimos que la primera forma fundamental de $S$ es la restricción del producto escalar de $\mathbb{R}^{3}$ a cada plano tangente $T_{p}(S)$. Es decir, la forma quadrática $I_{p}$ en $T_{p}(S)$ es la primera forma fundamental de $S$ en $p$.
\end{defn}

\begin{obs}
  La primera forma fundamental nos permite tomar medidas sobre la superficie sin referirnos al espacio $\mathbb{R}^{3}$.
\end{obs}

\begin{note}[Expresión de la Primera Forma Fundamental]
  Sea $S \subset \mathbb{R}^{3}$ superfice, $X : U \subset \mathbb{R}^{2} \to S$ parametrización. Entonces, $\exists \{ X_{u}, X_{v} \}$ base de $X$ en $p \in S$. Como el vector tangete $w \in T_{p}(S)$ es tangete a la curva $\alpha(t) = X(u(t), v(t))$ con $t \in (-\epsilon, \epsilon)$ y $p = \alpha(0) = X(u_{0}, v_{0})$ tenemos que
  \[ 
    I_{p}(\alpha'(0)) = \langle \alpha'(0){ , }\alpha'(0) \rangle_{p}
  \] 
  \[ 
    \langle X_{u}u' + X_{v}v'{ , } X_{u}u' + X_{v}v' \rangle_{p}
  \] 
  \[ 
    \langle X_{u}{ , }X_{u} \rangle_{p}(u')^{2} + 2 \langle X_{u}{ , }X_{v} \rangle_{p} u' v' + \langle X_{v}{ , }X_{v} \rangle_{p}(v')^{2} 
  \]
  \[ 
    E(u')^{2} + 2 F u' v' + G(v')^{2} 
  \] 
  donde $t = 0$ y 
  \[ 
    E(u_{0}, v_{0}) = \langle X_{u}{ , }X_{u} \rangle_{p},
  \] 
  \[ 
    F(u_{0}, v_{0}) = \langle X_{u}{ , }X_{v} \rangle_{p}, 
  \] 
  \[ 
    G(u_{0}, v_{0}) = \langle X_{v}{ , }X_{v} \rangle_{p} 
  \] 
\end{note}

\begin{ejm}
  Un sistema de coordenadas para un plano $P \subset \mathbb{R}^{3}$ que pasa por el punto $p_{0} = ( x_{0}, y_{0}, z_{0} )$ y que contiene los vectores ortonormales $w_{1} = (a_{1}, a_{2}, a_{3})$, $w_{2} = (b_{1}, b_{2}, b_{3})$ viene dado por
  \[ 
    X(u,v) = p_{0} + u w_{1} + v w_{2}, \quad (u, v) \in \mathbb{R}^{2}, 
  \] 
  Para calcular la primera forma fundamental en un punto arbitrario de $P$ observamos que $X_{u} = w_{1}, X_{v} = w_{2}$. Dado que $w_{1}$ y $w_{2}$ son vectores unitarios ortogonales, la funciones $E, F, G$ son constantes y vienen dadas por
  \[ 
    E = 1, \quad F = 1, \quad G = 1. 
  \] 
\end{ejm}

\begin{ejm}
  El cilindro sobre el círculo $x^{2} + y^{2} = 1$ admite como parametrización $X : U \to \mathbb{R}^{3}$ definida por
  \[ 
    X(u, v) = (\cos(u), \sen(u), v),
  \] 
  \[ 
    U = \{ (u, v) \in \mathbb{R}^{2} : u \in (0, 2 \pi), v \in (- \infty, + \infty) \}. 
  \] 
  Para calcular la primera forma fundamental, calculamos las derivadas paraciales
  \[ 
    X_{u} = (- \sen(u), \cos(v), 0), \quad X_{v} = (0, 0, 1),
  \] 
  entonces,
  \[ 
    E = 1, \quad F = 0, \quad G = 1. 
  \] 
\end{ejm}

\section{Isometrías}

\begin{defn}[Superficies Isométricas]
  Un difeomorfismo $\varphi : S \to S'$ es un isometría si $\forall p \in S, \forall w_{1}, w_{2} \in T_{p}(S)$ se tiene que
  \[ 
  {\langle w_{1}{ , }w_{2} \rangle}_{p} = \langle d \varphi_{p}(w_{1}){ , }d \varphi_{p}(w_{2}) \rangle_{\varphi(p)}
  \] 
  Las superficies $S$ y $S'$ se dice que son isométricas.
\end{defn}

\begin{obs}
  Si $\forall p \in S, (d \varphi)_{p} : T_{p}(S) \to T_{p}(S')$ conserva el producto escalar, entonces $\varphi$ es una isometría.
\end{obs}

\begin{defn}
  Sea $\varphi : V \to \overline{S}$ aplicación donde $V$ es un entorno de $p$ en $S \subset \mathbb{R}^{3}$ superficie. Entonces, $\varphi$ es una isometría local si $\exists \overline{V}$ entorno de $\varphi(p)$ en $\overline{S}$ tal que $\varphi : V \to \overline{V}$ es una isometría.

  \begin{itemize}
    \item Si $\forall p \in S$, existe una isometría local en $\overline{S}$ entonces, decimos que $S$ es localmente isométrica a $\overline{S}$.
    \item Decimos que $S$ y $\overline{S}$ son isométricas si $S$ es localmente isométrica a $\overline{S}$ y $\overline{S}$ es localmente isométrica a $S$.
  \end{itemize}
\end{defn}

\begin{obs}
  Si $\varphi : S \to \overline{S}$ es difeomorfismo y isometría local $\forall p \in S$, entonces $\varphi$ es isometría global.
\end{obs}

\begin{ejm}
  Consideramos el cilindro del ejemplo anterior. Sea $\varphi : \mathbb{R}^{2} \to C$ definida por
  \[ 
    \varphi(u, v) = (\cos(u), \sen(u), v).
  \] 
  De esta manera  $\varphi$ es difeomorfismo local e isometría local ya que es $\varphi$ diferenciable, biyectiva, $(d \varphi)_{q}$ es isomorfismo $\forall q \in V$ entonor de $p$ en $S$, $\varphi^{-1}$ es diferenciable y $\langle (d \varphi)_{p}(w){ , }(d X)_{p}(w) \rangle_{p} = ||w||^{2}$.
\end{ejm}

\begin{prop} 
  Sea $f : S \to S'$ difeomorfismo local. Entonces, $ f$ es isometría local $\Leftrightarrow$ f preserva longitudes de curvas.
\end{prop}

\begin{dem}
  content
\end{dem}

\begin{prop}
  Sea $\phi : S \to \overline{S}$ una isometría entre superficies, $X : U \to S$ parametrización de $S$. Entonces, $\overline{X} = \phi \circ X$ es una parametrización de $\phi(X(U)) \subset \overline{S}$ y
  \[ 
    E = \overline{E}, \quad F = \overline{F}, \quad G = \overline{G}. 
  \] 
\end{prop}

\begin{dem}
  content
\end{dem}

\begin{cor}
  Una propiedad local $P$ de una superficie en términos de $E, F, G$ es invariante por isometrías. Si $\phi : S \to S'$ es isometría, $S$ satisface $P$ si y solo si $S'$ satisface $P$.
\end{cor}

\begin{prop}
  Sean $X : U \to S$ y $X' : U \to S'$ parametrizaciones tal que $\forall (u, v) \in U,$
  \[ 
    E(u, v) = E'(u, v), \quad F(u, v) = F'(u, v), \quad G(u, v) = G'(u, v),
  \] 
  Entonces, $\varphi = X' \circ X^{-1} : X(U) \to S'$ es isometría local.
\end{prop}

\begin{dem}
  content
\end{dem}

\section{Area de una Superficie}

\begin{defn}
  Sea $S \subset \mathbb{R}^{3}$ superficie, $X : U \to S$ parametrización. Llamamos área de $R = X(Q)$ una región acotada de $S$ a
  \[ 
    A(R) = \iint_{Q}^{} ||X_{u} \times X_{v}|| du \ dv 
  \] 
  \[ 
    = \iint_{Q}^{} \sqrt{EG - F^{2}} \ du \ dv 
  \] 
\end{defn}

\begin{obs}
  El área no depende de la parametrización que escojamos.
\end{obs}

\begin{dem}
  content
\end{dem}

\section{Ángulo de Aplicaciones Conformes}

\begin{defn}
  Sean $\alpha : I \to S, \beta : J \to S$ curvas diferenciables en $S$ que se cortan en $p = \alpha(0) = \beta(0)$. Decimos que se cortan con ángulo $\theta$ donde
  \[ 
    \cos(\theta) = \frac{\alpha'(0) \cdot \beta'(0)}{||\alpha'(0)||\cdot ||\beta'(0)||}.
  \] 
\end{defn}

\begin{defn}
  Sea $\varphi : S \to S'$ diferenciable. Decimos que $\varphi$ preserva ángulos en $p \in S$ si $\forall v, w \in T_{p}(S)$, el ángulo entre $v$ y $w$ es el mismo que el ángulo entre $(d \varphi)_{p}(v)$ y $(d \varphi)_{p}(w)$. Decimos que $\varphi$ preserva ángulos si $\varphi$ preserva ángulos $\forall p \in S$.
\end{defn}

\begin{prop}
  Las isometrías locales preservan ángulos.
\end{prop}

\begin{dem}
  content
\end{dem}

\begin{prop}
  Sea $f : \mathbb{R}^{2} \to \mathbb{R}^{2}$ aplicación lineal. Son equivalentes,
  \begin{enumerate}[label=(\roman*)]
    \item $f$ preserva ángulos
      \[ 
        \frac{f(u) \cdot f(v)}{||f(u)|| ||f(v)||} = \frac{u \cdot v}{||u|| ||v||}, \quad \forall u, v \in \mathbb{R}^{2}.
      \] 
    \item $\exists \lambda > 0 : f(v) \cdot f(w) = \lambda^{2} (v \cdot w), \quad \forall v, w \in \mathbb{R}^{2}$.
    \item $\exists \lambda > 0 : || f(v)|| = \lambda ||v||, \quad \forall v \in \mathbb{R}^{2}$.
    \item Dada base ortonormal $\{ u_{1}, u_{2} \}, \exists \lambda > 0 : ||f(u_{i})|| = \lambda ||u_{i}||$ y $(f(u_{1}) \cdot f(u_{2})) = 0$.
  \end{enumerate}
\end{prop}

\begin{dem}
  content
\end{dem}

\begin{defn}
  Un difeomorfismo $\varphi : S \to \overline{S}$ es una aplicación conforme si $\forall p \in S, \forall v_{1}, v_{2} \in T_{p}(S)$ 
  \[ 
    (d \varphi)_{p}(v_{1}) \cdot (d \varphi)_{p}(v_{2}) = \lambda^{2}(p) \langle v_{1}{ , }v_{2} \rangle_{p},
  \] 
  donde $\lambda^{2}> 0$ es una aplicación diferenciable en $S$.
\end{defn}
