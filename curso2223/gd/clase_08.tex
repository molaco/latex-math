\section{Geodésicas}

\begin{obs}[Idea Geodésicas]
  Una geodésica generaliza rectas en $\mathbb{R}^{2}$ y $\mathbb{R}^{3}$.
\end{obs}

\begin{obs}[Rectas]
  Una recta $r$ en $\mathbb{R}^{2}$ o $\mathbb{R}^{3}$ tiene las siguientes propiedades
  \begin{enumerate}[label=(\roman*)]
    \item $\forall p, q \in r$, $r$ es la curvar de menor longitud entre $p$ y $q$.
    \item Si $\alpha(t) = p + t \vec{v}$ p.p.a entonces $\alpha''(t) = 0, \forall t$.
    \item Los vectores tangentes a $\alpha'(t)$ son todos paralelos.
  \end{enumerate}
\end{obs}

\begin{obs}[Propiedades Geodésicas]
  Sea $\gamma$ geodésica en $S$ entonces
  \begin{enumerate}[label=(\roman*)]
    \item $\gamma$ minimiza la longitud localmente. Es decir, $\forall p \in \gamma, \exists U $ entorno de $p$ tal que $\forall q \in \gamma \cap U$ se tiene que $\gamma$ es la curva de longitud mínima que une $p$ y $q$.
    \item $\gamma$ p.p.a será geodésica si la proyección  de $\gamma''(t)$ sobre $T_{\gamma(t)}(S)$ es $\vec{0}$.
    \item $\gamma'(t)$ se percibe desde $S$ como transporte paralelo de vectores.
  \end{enumerate}
\end{obs}

\begin{defn}[Geodésica]
  Una curva regular $\alpha : I \to S$ es una geodésica si 
  \[ 
    \alpha''(t) \perp T_{\alpha(t)}(S), \quad \forall t \in I
  \] 
\end{defn}

\begin{lem}
  Sea $\alpha$ geodésica, entonces $||\alpha'(t)||$ es constante.
\end{lem}

\begin{dem}
  Sea $\alpha$ geodésica, entonces
  \[ 
    \frac{d{}}{d{t}}|| \alpha'(t) ||^{2} = \frac{d{}}{d{t}} \langle \alpha'(t){ , }\alpha'(t) \rangle
  \] 
  \[ 
    = \langle \alpha''(t){ , }\alpha'(t) \rangle  = 0
  \] 
\end{dem}

\begin{obs}
  Es decir, el parámetro $t$ de una geodésica $\alpha(t)$ es proporcional al parámetro $s$ longitud de arco $\Rightarrow \exists c \neq 0 : t = c - s$.
\end{obs}

\begin{dem}
  Sea $\alpha$ geodésica, entonces
  \[ 
    s = \int_{0}^{t} ||\alpha'(t)|| dt = \int_{0}^{t} c \ dt = c \cdot t, \quad \forall t \in I.
  \] 
\end{dem}

\begin{obs}
  Si $\alpha$ es geodésica, entonces $||\alpha'(t)|| = c, \forall t \in I$. Sea $\beta(s)$ la p.p.a de $\alpha$, entonces $\alpha(t) = \beta(c \cdot s)$.
\end{obs}

\begin{defn}[Curvatura Geodésica]
  Sea $\alpham : I \to S$ p.p.a., $N : S \to \mathbb{R}^{3}$ campo vectorial unitario a $S$. Entonces,
  \[
    \alpha''(s) = k_{n}(s) N_{\alpha(s)} + k_{g}(s) N(\alpha(s)) \times \alpha'(s)
  \]
  donde $k_{n}$ es la cuvatura normal y $k_{g}(s)$ es la curvatura geodésica de $\alpha$ en $s$.
\end{defn}

\begin{obs}
  Si $||\alpha'(s)|| = 1$, entonces $\alpha''(s) \perp \alpha'(s)$.
\end{obs}

\begin{defn}[Curvatura Geodésica de curva regular]
  Si $\alpha$ es una curva regular, entonces su curvatura geodésica es la de su reparametrización por arco.
\end{defn}

\begin{obs}
  Si $\alpha$ es p.p.a., entonces $\alpha$ es geodésica $\Leftrightarrow k_{g}(s) = 0 \forall s$.
\end{obs}

\begin{ejm}[Plano]
  Sea $P = \{ x : \langle x-p{ , }a \rangle = 0 \}$, $\alpha : I \to P$ geodésica p.p.a.. Entonces,
  \[ 
    \langle \alpha(t) - p{ , }a \rangle = 0
  \] 
  donde derivando tenemos que
  \[ 
    \langle \alpha''(t){ , }a \rangle = 0 
  \] 
  como $a \neq 0$, entonces $\alpha''(t) = 0, \forall t \in I$. Por tanto,
  \[ 
    \alpha(t) = p + t \cdot \vec{v},
  \] 
  es decir, la geodésica es parametrización afín de una recta.
\end{ejm}

\begin{ejm}[Esfera]
  Los círculos máximos p.p.a son geodésicas. Sea $\alpha : I \to \mathbb{S}^{2}$ geodésica p.p.a.. Entonces, $||\alpha'(t)|| = 1$ y derivando $||\alpha'(t)||^{2} = 1$
  \[ 
    \langle \alpha'(t){ , }\alpha(t) \rangle = 0
  \] 
  \[ 
    \langle \alpha''(t){ , }\alpha(t) \rangle + \langle \alpha'(t){ , }\alpha(t) \rangle = 0
  \] 
  \[ 
    \langle \alpha''(t){ , }\alpha(t) \rangle + ||\alpha'(t)||^{2} = 0
  \] 
  \[ 
    \Rightarrow \langle \alpha''(t){ , }\alpha(t) \rangle = -1
  \] 
  Por tanto, si $\alpha$ es geodésica, entonces $\alpha''(t) = - \alpha(t)$. Luego,
  \[ 
    \alpha(t) = (a_{1}\sen(t) + a_{2}\cos(t), b_{1}\sen(t) + b_{2}\cos(t), c_{1}\sen(t) + c_{2} \cos(t)), \quad t \in ( 0, 2 \pi ).
  \] 
  Si $\alpha(0) = p$, $\alpha'(0) = v$, $||\alpha(t)|| = 1$, $||\alpha'(t)|| = 1$ entonces
  \[ 
    \alpha(t) = p \cos(t) + v \sen(t) 
  \] 
  que es parametrización de círculo máximo.
\end{ejm}

\begin{ejm}[Cilindro]
  Sea $C = \{ (x,y,z) : x^{2} + y^{2} = 1 \}$, $\alpha : I \to C$ geodésica p.p.a. con 
  \[ 
    \alpha(t) = (x(t), y(t), z(t)), \quad t \in [0, 1].
  \] 
  Como $\alpha(t) \in C$, entonces $x^{2}(t) + y^{2}(t) = 1$ y derivando dos veces tenemos
  \[ 
    2x(t)x'(t) + 2y(t)y'(t) = 0 
  \] 
  \[ 
    2x'(t)x'(t) + 2x(t)x''(t) + 2y'(t)y'(t) + 2y(t)y''(t) = 0
  \] 
  Ahora, $||\alpha'(t)|| = 1$ entonces,
  \[ 
    ||(x'(t), y'(t), z'(t))|| = \sqrt{x'(t)^{2} + y'(t)^{2} + z'(t)^{2}} = 1
  \] 
  \[ 
    \Rightarrow x'(t)^{2} + y'(t)^{2} + z'(t)^{2} = 1
  \] 
  Por tanto, si sustituimos en la ecuación anterior tenemos
  \[ 
    x'(t)^{2} + y'(t)^{2} + x(t)x''(t) + y(t)y''(t) = 0 
  \] 
  \[ 
    1 - z'(t)^{2} + x(t)x''(t) + y(t)y''(t) = 0 
  \] 
  Dado que $\alpha''(t) \perp \alpha'(t)$, tenemos que $\alpha''(t) || \alpha(t)$, entonces
  \[ 
    (x'', y'') = \lambda(t) (x, y).
  \] 
  Luego,
  \[ 
    \lambda(t) = -1 + z'(t)^{2} 
  \] 
  Resolviendo el sistema asociado
  \[ 
    \begin{aligned}
      \begin{cases}
        z''(t) = 0 \\
        x''(t) = (-1 + z'(t)^{2}) x(t) \\
        y''(t) = (-1 + z'(t)^{2}) y(t)
      \end{cases}
    \end{aligned} 
  \] 
  con $z(t) = t b, \alpha(0) = (1,0,0), \alpha'(0) = (0, a, b), a^{2} + b^{2} = 1$ obtenemos
  \[ 
    \alpha(t) = (\cos(at), \sen(at), bt)
  \] 
  que es una hélice o una recta si $a = 0$.
\end{ejm}

\begin{ejm}[Geodésicas en coordenadas locales]
  Sea $X : U \to X(U) \subset S$ parametrización se $S$ y $\alpha : I \to S$ curva p.p.a con $\alpha(0) \in X(U)$. Sea
  \[ 
    \beta(t) = X^{-1}(\alpha(t)) = (u(t), v(t))
  \] 
  entonces,
  \[ 
    \alpha'(t) = u'(t) X_{u} + v'(t) X_{v} 
  \] 
  donde derivando de nuevo tenemos
  \[ 
    \alpha''(t) = u'(t) X_{u} + u'(t)[X_{uu} u'(t) + X_{uv}v'(t)]
  \] 
  \[ 
    + v''(t) X_{v} + v'(t)[X_{vu} u'(t) + X_{vv}v'(t)] 
  \] 
  donde
  \[ 
    X_{uu} = \Gamma_{11}^{1} X_{u} + \Gamma_{11}^{2} X_{v} + e N^{X}
  \] 
  \[ 
    X_{uv} = \Gamma_{12}^{1} X_{u} + \Gamma_{12}^{2} X_{v} + f N^{X}
  \] 
  \[ 
    X_{uu} = \Gamma_{22}^{1} X_{u} + \Gamma_{22}^{2} X_{v} + g N^{X}
  \] 
  y como $\alpha''(t) \perp T_{\alpha(t)}(S)$ tenemos
  \[ 
    \langle \alpha''(t){ , }\alpha'(t) \rangle = 0
  \] 
  Por tanto, el sistema de ecuaciones
  \[ 
    \begin{aligned}
      \begin{cases}
        u'' + \Gamma_{11}^{1}(u')^{2} + 2\Gamma_{12}^{1}u' v' + \Gamma_{22}^{1}(v')^{2} = 0 \\
        v'' + \Gamma_{11}^{2}(u')^{2} + 2\Gamma_{12}^{2}u' v' + \Gamma_{22}^{2}(v')^{2} = 0 \\
      \end{cases}
    \end{aligned} 
  \] 
  nos permite encontrar geodésicas en coordenadas locales.
\end{ejm}

\begin{nota}[Símbolos de Christoffel]
  \[ 
    \begin{aligned}
      \begin{cases}
        \Gamma_{11}^{1} E + \Gamma_{11}^{2} F = \langle X_{uu}{ , }X_{u} \rangle = \frac{1}{2} E_{u} \\
        \Gamma_{11}^{1} F + \Gamma_{11}^{2} G = \langle X_{uu}{ , }X_{v} \rangle = F_{u} - \frac{1}{2} E_{v} \\
      \end{cases}
    \end{aligned} 
  \] 
  \[ 
    \begin{aligned}
      \begin{cases}
        \Gamma_{12}^{1} E + \Gamma_{12}^{2} F = \langle X_{uv}{ , }X_{u} \rangle = \frac{1}{2} E_{v} \\
        \Gamma_{11}^{1} F + \Gamma_{11}^{2} G = \langle X_{uv}{ , }X_{v} \rangle = \frac{1}{2}G_{u} \\
      \end{cases}
    \end{aligned} 
  \] 
  \[ 
    \begin{aligned}
      \begin{cases}
        \Gamma_{11}^{1} E + \Gamma_{11}^{2} F = \langle X_{vv}{ , }X_{u} \rangle = F_{v} - \frac{1}{2} G_{u} \\
        \Gamma_{11}^{1} F + \Gamma_{11}^{2} G = \langle X_{vv}{ , }X_{v} \rangle = \frac{1}{2} G_{v} \\
      \end{cases}
    \end{aligned} 
  \] 
\end{nota}

\begin{ejm}[Geodésicas de Superficie de Revolución]
  Sea la parametrización 
  \[ 
    X(u,v) = (f(v) \cos(u), f(v) \sen(u), g(v)), \quad f(v) \neq 0
  \] 
  se tiene que
  \[ 
    E = f(v)^{2}, \quad F = 0, \quad G = f'(v)^{2} + g'(v)^{2}
  \] 
  Primero obtenemos los símbolos de Christoffel derivando los coeficientes de la primera forma fundamental
  \[ 
    E_{u} = 0, \quad E_{v} = 2 f f' 
  \] 
  \[ 
    F_{u} 0, \quad F_{v} = 0, 
  \] 
  \[ 
    G_{u} = 0, \quad G_{v} = 2(f' f'' + g' g'') 
  \] 
  Las dos primeras ecuaciones dan
  \[ 
    \Gamma_{11}^{1} = 0, \quad \Gamma_{11}^{2} = - \frac{f f'}{(f')^{2} + (g')^{2}} 
  \] 
  Las dos segundas dan
  \[ 
    \Gamma_{12}^{1} = \frac{f f'}{f^{2}}, \quad \Gamma_{12}^{2} = 0
  \] 
  Y Las dos últimas dan
  \[ 
    \Gamma_{22}^{1} = 0, \quad \Gamma_{22}^{2} = \frac{f' f'' + g' g''}{f'^{2} + g'^{2}}.
  \] 
  Con estos valores, el sistema de ecuaciones diferenciales resultante es
  \[ 
    \begin{aligned}
      \begin{cases}
        u'' + \frac{2 f f'}{f^{2}}u' v' = 0 \\
        v'' - \frac{f f'}{(f')^{2} + (g')^{2}}(u')^{2} + \frac{f' f'' + g' g''}{f'^{2} + g'^{2}}(v')^{2} = 0
      \end{cases}
    \end{aligned} 
  \]   
  \begin{enumerate}[label=(\roman*)]
    \item   Vemos que $u = \cte$ y $v = v(s)$ son geodésicas. La primera ecuación se satisface si $u = \cte$. En este caso, $u' = 0$ y $u'' = 0$
  \[ 
    0 + \frac{2 f f'}{f^{2}} \cdot 0 \cdot v' = 0 
  \] 
  y la segunda ecuación resulta
  \[ 
    v'' + \frac{f' f'' + g' g''}{(f')^{2} + (g')^{2}} (v')^{2} = 0 
  \] 
  Por tanto,
  \[ 
    v'' = - \frac{f' f'' + g' g''}{(f')^{2} + (g')^{2}} (v')^{2}
  \] 
    \item Vemos que paralelos $v = \cte$, $u = u(s)$ son geodésicas. La primera de las ecuaciones es
      \[ 
        u'' + \frac{2 f f'}{f^{2}}u' \cdot 0 = 0
      \] 
      \[ 
        u'' = 0 
      \] 
      y la segunda es
      \[ 
        \frac{f f'}{(f')^{2} + (g')^{2}}(u')^{2} = 0
      \] 
      Por tanto, para que el paralelo $v = \cte$, $u = u(s)$ sea geodésica se debe tener que $u' \neq 0$. Como $f'^{2} + g'^{2} \neq 0$ y $ f \neq 0$, entonces $f' = 0$.
    \item La primera ecuación se puede escribir como 
      \[ 
        (f^{2} u')' = f^{2} u'' + 2 f f' u' v' = 0.
      \] 
      Por tanto, 
      \[ 
        f^{2} u' = \cte = c 
      \] 
      Por otro lado, el ángulo $\theta$ de una geodésica con un paralelo que la interseca es 
      \[ 
        \cos(\theta) = \frac{| \langle X_{u}{ , }X_{u} u' + X_{v} v' \rangle |}{| X-u |} = | f u' | 
      \] 
      donde $f = r$, entonces
      \[  
        r \cos(\theta) = \cte = | c |
      \] 

  \end{enumerate}
\end{ejm}

\begin{prop}[Invariancia Geodésica Por Isometrías]
  El sistema de ecuciones diferenciales de las geodésicas solo depende de los Símbolos de Christoffel, es decir, solo depende de los coeficientes $E, F, G$ de la primera forma fundamental. Por tanto, si $f : S \to S'$ es una isometría local y $\alpha : I \to S$ es geodésica entonces $(f \circ \alpha) : I \to S'$ es geodésica.
\end{prop}

\begin{ejm}
  Sea $P$ un plano y $C$ un cilindro. Como $P$ y $C$ son isométricos, $\exists f$ isometría local y por tanto cualquier geodésica en $P$es geodésica en $C$ por la invariancia de las geodésica por isometrías.
\end{ejm}

\begin{prop}
  $\forall p \in S$, $\forall v \in T_{p}(S)$, $\exists! \alpha_{p, v} : (-\epsilon, \epsilon) \to S$  geodésica con $\alpha(0) = p$ y $\alpha'(0) = v$.
\end{prop}

\section{Aplicación Exponencial}

\begin{defn}
  La aplicación exponencial en un entorno de $p \in S$ es
  \[ 
    \exp_{\phi} : B_{\epsilon_{0}} \to  S
  \] 
  \[ 
    v \mapsto \alpha_{p, v} 
  \] 
\end{defn}

\begin{prop}
  \begin{enumerate}[label=(\roman*)]
    \item La exponencial es continua y diferenciable.
    \item Es un difeomorfismo en $0$.
    \item La imagen del rango $\{ \lambda v \}_{\lambda \in [0,1]}$ es una geodésica.
    \item La imagen de las circinferencias concentricas son ortogonales a las geodésicas radiales.
    \item Tomando coordenadas polares se tiene 
      \[ 
        X(r,\theta) = \exp_{p} (r \cos(\theta), r \sen(\theta)), \quad r \in (0, \epsilon_{0}), \theta \in (0, 2 \pi)
      \] 
      satisfa
      \[ 
        E = 1, \quad F = 0, \quad \lim_{r \to 0} G = 0, \quad \lim_{r \to 0} \sqrt{G} = 1,
      \] 
    \item La curvatura geodésica en las coordenas polares de $\exp_{p}(r,\theta)$ es 
      \[ 
        K \circ X = -\frac{\sqrt{G}_{r r}}{\sqrt{G}} 
      \] 
    \item Las ecuaciones en estas coordenadas locales son
      \[ 
        \begin{aligned}
          \begin{cases}
            r'' = \frac{1}{2}G_{r}(\theta')^{2} \\
            \theta'' = - \frac{G_{r}}{G} \theta' r' - \frac{1}{2} \frac{G_{\theta}}{G}(\theta')^{2}
          \end{cases}
        \end{aligned} 
      \] 
  \end{enumerate}
\end{prop}
